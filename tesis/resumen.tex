\documentclass[letterpaper, 11pt]{article}
%\usepackage{fullpage} % changes the margin
\usepackage[spanish]{babel}
\usepackage{enumerate}
\usepackage[utf8]{inputenc}
\usepackage{hyperref}
\usepackage[margin=1.5cm, includefoot]{geometry}
\pagestyle{empty}
\begin{document}
\noindent
\large
\textbf{Verificación Formal de Arboles Rojinegros} \hfill 
\textbf{David Felipe Hern\'andez Chiapa} \\
\normalsize Resumen de Proyecto de Tesis       \hfill 312321329\\
                                               \hfill 2019-20-10 \\
\newcommand{\coq}{\textbf{Coq}}

\section*{Resumen del Proyecto de Tesis}
El propósito de este trabajo es presentar la verificación formal de una 
estructura de datos implementada en \textbf{Haskell}, los Arboles Rojinegros 
(ARN), usando un sistema de verificación 
% menos susceptible a errores
como lo es el asistente de pruebas \textbf{Coq}.
Esto se realizará con la ayuda de una herramienta de traducción de 
\textbf{Haskell} a \textbf{Coq} (\textsc{Hs-to-Coq}) [4]. 
Se verificar\'a dicha estructura de datos con tipos y operaciones
directamente traducidos de las bibliotecas de \textbf{Haskell}.

Se buscar\'a mostrar las ventajas y desventajas que se presentan al verificar 
formalmente una estructura de datos y los retos encontrados al tratar de 
verificar programas directamente traducidos de \textbf{Haskell}.

\section*{Índice tentativo}
\begin{enumerate}[I.]
  \item Motivación
    \begin{enumerate}[1.]
     \item Arboles Rojinegros
     \item Traducción de Haskell a {\coq} (Hs-to-Coq)
    \end{enumerate}

  \item Implementación de Arboles Rojinegros en {\coq}
    \begin{enumerate}[1.]
      \item Traduciendo implementaciones
      \item Inserción de elementos en un ARN
      \item Eliminación de elementos de un ARN
    \end{enumerate}

  \item Verificación formal de ARN
    \begin{enumerate}[1.]
      \item Capturando los invariantes de los ARN
      \item Verificaci\'on de la operaci\'on de inserci\'on 
      \item Verificaci\'on de la operaci\'on de eliminaci\'on
    \end{enumerate}
    
  \item Conclusiones
\end{enumerate}

\section*{Bibliografía básica}
\begin{enumerate}
  \item Graciela L\'opez Campos, \emph{Implementaciones funcionales de arboles roji-negros}. 2015., 
Disponible en l\'inea \url{	http://132.248.9.195/ptd2015/septiembre/0735684/Index.html}, 2019.
  \item  Andrew W. Appel, Pierre Letouzey. \emph{Library Coq.MSets.MSetRBT}. 2011. \url{https://coq.inria.fr/library/Coq.MSets.MSetRBT.html}
  \item Pierce, Benjamin C. et al., \emph{Software Foundations}. 2017. \url{http://www.cis.upenn.edu/~bcpierce/sf/current/index.html} 
  \item Joachim Breitner, Antal Spector-Zabusky, Yao Li, Christine Rizkallah, John Wiegley, Stephanie Weirich, \emph{Ready, Set, Verify!}. 2018., 
Disponible en l\'inea \url{	https://arxiv.org/pdf/1803.06960v2.pdf}, 2019.
\end{enumerate}

\end{document}
