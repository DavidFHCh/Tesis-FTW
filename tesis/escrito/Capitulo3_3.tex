\subsection{Verificación de la operación de eliminación}
Al igual que en la funci\'on de inserci\'on se enuncian lemas para ayudarnos a llegar al resultado
de verificar la operación de eliminación. Estos lemas giran en torno a las funciones auxiliares
que se usaron para poder demostrar la operación, como $append$ y $del$.

A continuación se describe el razonamiento usado para poder verificar dichas funciones.

\subsubsection{Primer Lema}
\begin{figure}[!ht]
\centering
\captionsetup{justification=centering}
\begin{minted}{coq}
Lemma append_arb_rb {a} `{GHC.Base.Ord a} (n:nat) (l r: RB a) :
is_redblack n l -> is_redblack n r ->
(nearly_redblack n (append l r)) /\
(notred l -> notred r -> is_redblack n (append l r)).
\end{minted}
\caption{Lema $append\_arb\_rb$.}
\label{lema_4}
\end{figure}

La funci\'on mas importante para la operaci\'on de eliminaci\'on es $append$, la cual concatena
dos subarboles. Estos dos subarboles son el resultado de buscar, encontrar y eliminar un nodo. En
este primer lema se enuncia lo antes descrito: que para cualesquiera dos \'arboles si estos
cumplen con la definici\'on inductiva de $is\_redblack$, ambos con altura $n$, el resultado de
concatenar es casi un {\arn}, en otras palabras, la concatenaci\'on cumple con la definicion de
$nearly\_redblack$. Pero si los \'arboles que se van a concatenar ademas de cumplir con
$is\_redblack$, tambien cumplen con $notred$, es decir, las raices de dichos arboles no son rojas,
el resultado de concatenar respeta tambien la definici\'on $is\_redblack$. La demostracio\'on de
este lema en coq se describe en seguida:

\begin{minted}{coq}
______________________________________(1/1)
Forall (r : RB a) (n : nat),
  is_redblack n l
  -> is_redblack n r
    -> nearly_redblack n (append l r)
      /\ (notred l -> notred r -> is_redblack n (append l r))
\end{minted}

En este primera etapa de la demostraci\'on podemos ver lo que se describio en el parrafo anterior.
Se decidio proseguir con esta demostraci\'on usando inducci\'on, primero sobre el arbol $l$ y
posteriormente sobre $r$. Los casos base de estas inducciones consisten en simplificacion de las
expresiones y facilmente se llega a una hipotesis o a una contradicci\'on. Estos casos no se
trataran mas a fondo en este trabajo, nos pasaremos directamente a los casos mas interesantes.

\begin{figure}[!ht]
\centering
\captionsetup{justification=centering}
\begin{minted}{coq}
______________________________________(1/4)
forall n : nat,
is_redblack n (T R ll lx lr)
-> is_redblack n (T R rl rx rr)
-> nearly_redblack n (append (T R ll lx lr) (T R rl rx rr))
/\ (notred (T R ll lx lr)
-> notred (T R rl rx rr)
-> is_redblack n (append (T R ll lx lr) (T R rl rx rr)))
______________________________________(2/4)
forall n : nat,
is_redblack n (T R ll lx lr)
-> is_redblack n (T B rl rx rr)
-> nearly_redblack n (append (T R ll lx lr) (T B rl rx rr))
/\ (notred (T R ll lx lr)
-> notred (T B rl rx rr)
-> is_redblack n (append (T R ll lx lr) (T B rl rx rr)))
______________________________________(3/4)
forall n : nat,
is_redblack n (T B ll lx lr)
-> is_redblack n (T R rl rx rr)
-> nearly_redblack n (append (T B ll lx lr) (T R rl rx rr))
/\ (notred (T B ll lx lr)
-> notred (T R rl rx rr)
-> is_redblack n (append (T B ll lx lr) (T R rl rx rr)))
______________________________________(4/4)
forall n : nat,
is_redblack n (T B ll lx lr)
-> is_redblack n (T B rl rx rr)
-> nearly_redblack n (append (T B ll lx lr) (T B rl rx rr))
/\ (notred (T B ll lx lr)
-> notred (T B rl rx rr)
-> is_redblack n (append (T B ll lx lr) (T B rl rx rr)))
\end{minted}
\caption{Casos del lema $append\_arb\_rb$.}
\label{casos_append}
\end{figure}

Esta doble inducci\'on nos deja con cuatro casos, expuestos en la figura \ref{casos_append}, estos
son los siguientes:
\begin{itemize}
    \item Los arboles a concatenar son rojos.
    \item El arbol que se concatenar a la izquierda es rojo y el derecho es negro.
    \item El arbol que se concatenar a la izquierda es negro y el derecho es rojo.
    \item Los arboles a concatenar son negros.
\end{itemize}

En estos cuatro casos se tiene que cuidar demasiado el hecho de no desbalancear el \'arbol, en
especial en los casos donde se manejan nodos negros, ya que estos son los unicos nodos
considerados para el balanceo.

En las siguientes subsecciones explicamos mas a fondo los pasos usados para probar estos casos.


\paragraph{Concatenaci\'on de dos \'arboles rojos.}

Este primer caso es la concatenaci\'on de dos \'arboles con raices rojas, en el siguiente
fragmento de la salida del asistente de pruebas se observa como la meta es una conjunci\'on.

\begin{minted}{coq}
IHlr : forall (r : RB a) (n : nat),
    is_redblack n lr
     -> is_redblack n r
       -> nearly_redblack n (append lr r)
         /\ (notred lr -> notred r -> is_redblack n (append lr r))
IHrl : forall n : nat,
     is_redblack n (T R ll lx lr)
     -> is_redblack n rl
       -> nearly_redblack n (append (T R ll lx lr) rl)
         /\ (notred (T R ll lx lr)
            -> notred rl -> is_redblack n (append (T R ll lx lr) rl))
______________________________________(1/1)
forall n : nat,
  is_redblack n (T R ll lx lr)
  -> is_redblack n (T R rl rx rr)
   -> nearly_redblack n (append (T R ll lx lr) (T R rl rx rr))
     /\ (notred (T R ll lx lr)
       -> notred (T R rl rx rr)
        -> is_redblack n (append (T R ll lx lr) (T R rl rx rr)))
\end{minted}

Podemos observar que la segunda parte de la conjunci\'on es una contradicci\'on, ya que al
introducir los antecedentes de la meta tendriamos lo siguiente:

\begin{minted}{coq}
H21 : notred (T R ll lx lr)
H22 : notred (T R rl rx rr)
______________________________________(1/1)
is_redblack n (append (T R ll lx lr) (T R rl rx rr))
\end{minted}

Evidentemente las dos funciones $notred$ de las hip\'otesis $H21$ y $H22$ se evaluan a falso y por
esto es una contradicci\'on.

Nos queda por demostrar la primera parte de la conjunci\'on, la meta de este caso, como se ve en
seguida, es que al ser concatenados un par de arboles rojos el arbol resultante cumple con la
definici\'on de ser de $nearly\_redblack$, es decir, que la raiz del \'arbol puede tener dos nodos
rojos consecutivos.

\begin{minted}{coq}
______________________________________(1/2)
nearly_redblack n (append (T R ll lx lr) (T R rl rx rr))
\end{minted}

El siguiente paso seria simplificar esta expresion, la cual cae en el caso de dos nodos rojos de la funcion $append$ y nos resulta en la siguiente meta:

\begin{minted}{coq}
______________________________________(1/1)
redred_tree n
  match append lr rl with
  | T R lr' x rl' => T R (T R ll lx lr') x (T R rl' rx rr)
  | _ => T R ll lx (T R (append lr rl) rx rr)
  end
\end{minted}

Podemos ver que la caza de patrones depende de la evaluaci\'on de la expresi\'on $append(lr,rl)$,
sea $rbt$, esto nos daria dos casos:

\begin{itemize}
    \item El primer caso, como se ve en seguida, se puede resolver usando las definiciones
    inductivas de $redred\_tree$ e $is\_redblack$, y las metas resultantes son resultados directos
    de aplicar las hipotesis que se muestran.
    \begin{minted}{coq}
        H8 : notred lr
        H9 : is_redblack n ll
        H14 : notred rl
        H16 : notred rr
        H18 : is_redblack n rr
        H19 : nearly_redblack n E
        H20 : notred lr -> notred rl -> is_redblack n E
        H21 : redred_tree n E
        ______________________________________(1/2)
        redred_tree n (T R ll lx (T R E rx rr))
    \end{minted}
    \item El segundo caso es un poco mas complejo, pues se tienen que ver los casos en que $rbt$,
    resulta en un \'arbol con raiz roja y negra:
    \begin{itemize}
        \item En caso de que el \'arbol sea rojo, se aplica de igual manera las definiciones
        inductivas mencionados en el caso anterior y las metas resultantes son implicaciones
        directas de las hipotesis que se muestran.
    \begin{minted}{coq}
    H6 : notred ll
    H9 : is_redblack n ll
    H14 : notred rl
    H16 : notred rr
    H20 : notred lr ->
          notred rl -> is_redblack n (T R r1_1 a0 r1_2)
    ______________________________________(1/1)
    redred_tree n (T R (T R ll lx r1_1) a0 (T R r1_2 rx rr))
    \end{minted}
        \item El caso donde el \'arbol es negro, al igual que en el caso pasado se hacen uso de
        las defincionces inductivas ya mencionadas y se siguen directamente de las siguientes
        hipotesis.
    \begin{minted}{coq}
    H6 : notred ll
    H8 : notred lr
    H9 : is_redblack n ll
    H10 : is_redblack n lr
    H14 : notred rl
    H16 : notred rr
    H17 : is_redblack n rl
    H18 : is_redblack n rr
    H19 : nearly_redblack n (T B r1_1 a0 r1_2)
    H20 : notred lr ->
          notred rl -> is_redblack n (T B r1_1 a0 r1_2)
    ______________________________________(1/1)
    redred_tree n (T R ll lx (T R (T B r1_1 a0 r1_2) rx rr))
    \end{minted}
    \end{itemize}
\end{itemize}

Con este procedimiento queda demostrado este caso de juntar dos \'aboles rojos con la funci\'on
$append$, se puede apreciar como los pasos de la demostraci\'on tienden a repetirse, esto puede
significar que existan una serie de comandos del asistente de pruebas que nos ayuden a acortar
esta prueba, sin embargo, en este trabajo se esta tomando el camino mas extenso para mostrar la
simplificaci\'on de la linea de pensamiento al demostrar estructuras complejas.

\paragraph{Concatenaci\'on de un \'arbol rojo y uno negro.}

Ahora es turno de analizar la demonstraci\'on del caso donde se concatena un \'arbol rojo y uno negro, es decir, $append(r,b)$, donde r es un \'arbol rojo y b es uno negro.

\begin{minted}{coq}
IHlr : forall (r : RB a) (n : nat),
     is_redblack n lr
     -> is_redblack n r
       -> nearly_redblack n (append lr r)
         /\ (notred lr -> notred r -> is_redblack n (append lr r))
______________________________________(1/1)
forall n : nat,
  is_redblack n (T R ll lx lr)
  -> is_redblack n (T B rl rx rr)
   -> nearly_redblack n (append (T R ll lx lr) (T B rl rx rr))
     /\ (notred (T R ll lx lr)
       -> notred (T B rl rx rr)
        -> is_redblack n (append (T R ll lx lr) (T B rl rx rr)))
\end{minted}

En este segundo caso la conjunci\'on tambien contiene una contradicci\'on en la mitad derecha de
esta, ya que se tiene \textit{notred (T R ll lx lr)}, entonces al igual que el caso pasado solo
resolveremos la primera mitad de la conjunci\'on. Para esta demostracion tenemos que guiar al asistente de pruebas un poco mas de lo normal, pues le tenemos que decirle que $r$ y el arbol \textit{(T B rl rx rr)} son el mismo.

\begin{minted}{coq}
r := T B rl rx rr : RB a
IHlr : forall n : nat,
     is_redblack n lr
     -> is_redblack n r
       -> nearly_redblack n (append lr r)
         /\ (notred lr -> notred r -> is_redblack n (append lr r))
n : nat
H1 : is_redblack n (T R ll lx lr)
H2 : is_redblack n r
______________________________________(1/1)
nearly_redblack n (T R ll lx (append lr r))
\end{minted}

Podemos observar que se realizo una simplificaci\'on de la meta, donde se desarrollo lo mas
posible la funci\'on append y se intodujeron los antecedentes a las hip\'otesis. Para poder
demostrar esta nueva meta tenemos que destruir la hipotesis `IHlr', lo cual nos introduciria los
dos antecedendetes de la misma como metas. Se destruye esta hipotesis para poder obtener su consecuente como hip\'otesis.

\begin{minted}{coq}
r := T B rl rx rr : RB a
IHlr : forall n : nat,
     is_redblack n lr
     -> is_redblack n r
       -> nearly_redblack n (append lr r)
         /\ (notred lr -> notred r -> is_redblack n (append lr r))
IHrl : forall n : nat,
     is_redblack n (T R ll lx lr)
     -> is_redblack n rl
       -> nearly_redblack n (append (T R ll lx lr) rl)
         /\ (notred (T R ll lx lr)
            -> notred rl -> is_redblack n (append (T R ll lx lr) rl))
n : nat
H1 : is_redblack n (T R ll lx lr)
H2 : is_redblack n r
______________________________________(1/3)
is_redblack n lr
______________________________________(2/3)
is_redblack n r
______________________________________(3/3)
nearly_redblack n (T R ll lx (append lr r))
\end{minted}

Para poder demostrar los dos primeros casos basta con aplicar la definici\'on inductiva
$is\_redblack$, lo cual nos introduce las hipotesis necesarias para poder cumplir las metas.
Para el \'ultimo caso nos basta de igual manera con aplicar la misma definici\'on inductiva a H1 y
a la meta aplicar las definiciones de $nearly\_redblack$ y $redred\_tree$ y esto nos da metas, que
gracias a las nuevas hip\'otesis integradas por H1, se pueden probar sin mayor problema.

Con esto demostrado este caso esta completo.

\paragraph{Concatenaci\'on de un \'arbol negro y uno rojo.}

En este caso se invierten los colores con respecto al caso anterior; el \'arbol derecho es rojo y
el izquierdo es negro. Este caso, al igual que el pasado, requiere de una pequeña ayuda al
asistente de pruebas, la cual explicaremos mas adelante.

\begin{minted}{coq}
IHlr : forall (r : RB a) (n : nat),
     is_redblack n lr
     -> is_redblack n r
       -> nearly_redblack n (append lr r)
         /\ (notred lr -> notred r -> is_redblack n (append lr r))
IHrl : forall n : nat,
     is_redblack n (T B ll lx lr)
     -> is_redblack n rl
       -> nearly_redblack n (append (T B ll lx lr) rl)
         /\ (notred (T B ll lx lr)
            -> notred rl -> is_redblack n (append (T B ll lx lr) rl))
______________________________________(1/1)
forall n : nat,
  is_redblack n (T B ll lx lr)
  -> is_redblack n (T R rl rx rr)
    -> nearly_redblack n (append (T B ll lx lr) (T R rl rx rr))
      /\ (notred (T B ll lx lr)
         -> notred (T R rl rx rr)
           -> is_redblack n (append (T B ll lx lr) (T R rl rx rr)))
\end{minted}

En este caso al igual que los dos pasados, como uno\footnote{o los dos.} de los arboles es de
color rojo, la segunda parte de la conjunci\'on vuelve a ser una contradicci\'on, por la expresion
$notred$.

Entonces solo nos quedamos con la primera mitad de la conjunci\'on:

\begin{minted}{coq}
l := T B ll lx lr : RB a
IHrl : forall n : nat,
     is_redblack n l
     -> is_redblack n rl
       -> nearly_redblack n (append l rl)
         /\ (notred l -> notred rl -> is_redblack n (append l rl))
n : nat
H1 : is_redblack n l
H2 : is_redblack n (T R rl rx rr)
______________________________________(1/2)
nearly_redblack n (T R (append l rl) rx rr)
\end{minted}

Podemos apreciar que este caso es el caso espejo del caso pasdo, entonces el procedimiento a usar
para demostrar esta meta es el mismo, lo \'unico que cambia es cuales hip\'otesis se usan para
lograr esto. En el caso pasado se destruyo la hip\'otesis IHlr, en este caso se usa su contraparte
IHrl y el resto de la demostraci\'on se sigue directamente de las nuevas metas introducidas y del
uso de las definiciones inductivas mencionadas en el caso pasado.

\paragraph{Concatenaci\'on de dos \'arboles negros.}

Este \'ultimo caso es el \'unico que no incluye una contradicci\'on ya que la contradicci'on se
daba al tener un \'arbol rojo como uno de los dos \'arboles que se pasan a la funci\'on $append$,
pero en este caso los dos \'arboles a concatenar son negros, entonces la conjunci\'on completa
ser\'a probada.

\begin{minted}{coq}
IHlr : forall (r : RB a) (n : nat),
     is_redblack n lr
     -> is_redblack n r
       -> nearly_redblack n (append lr r)
         /\ (notred lr -> notred r -> is_redblack n (append lr r))
IHrl : forall n : nat,
     is_redblack n (T B ll lx lr)
     -> is_redblack n rl
       -> nearly_redblack n (append (T B ll lx lr) rl)
         /\ (notred (T B ll lx lr)
            -> notred rl -> is_redblack n (append (T B ll lx lr) rl))
______________________________________(1/1)
forall n : nat,
  is_redblack n (T B ll lx lr)
  -> is_redblack n (T B rl rx rr)
    -> nearly_redblack n (append (T B ll lx lr) (T B rl rx rr))
      /\ (notred (T B ll lx lr)
         -> notred (T B rl rx rr)
           -> is_redblack n (append (T B ll lx lr) (T B rl rx rr)))
\end{minted}

Esta demostraci\'on se inicia con una inducci\'on sobre $n$, lo cual nos da el caso base con $n=0$
, seguido de la separaci\'on de la conjunci\'on y esto nos da 2 casos base, como se muestra en
seguida:

\begin{minted}{coq}
IHlr : forall n : nat,
     is_redblack n lr
     -> is_redblack n rl
       -> nearly_redblack n (append lr rl)
         /\ (notred lr -> notred rl -> is_redblack n (append lr rl))
IHrl : forall n : nat,
     is_redblack n (T B ll lx lr)
     -> is_redblack n rl
       -> nearly_redblack n (append (T B ll lx lr) rl)
         /\ (notred (T B ll lx lr)
            -> notred rl -> is_redblack n (append (T B ll lx lr) rl))
H1 : is_redblack 0 (T B ll lx lr)
H2 : is_redblack 0 (T B rl rx rr)
______________________________________(1/2)
nearly_redblack 0 (append (T B ll lx lr) (T B rl rx rr))
______________________________________(2/2)
notred (T B ll lx lr)
-> notred (T B rl rx rr)
  -> is_redblack 0 (append (T B ll lx lr) (T B rl rx rr))
\end{minted}

Estos casos base se resuelven aplicando las definiciones inductivas correspondientes tanto a las
metas como a las hip\'otesis H1 y H2, esto nos da las hipotesis necesarias para probar estos dos
casos.

Nos queda por probar el paso inductivo, en seguida podemos ver que la hipotesis de inducci\'on
`IH' es parte de IHlr, la cual se obtuvo de destruir esa hipotesis, en el siguiente paso se explica porque se decidio destruir esta hip\'otesis y no su contraparte IHrl.

\begin{minted}{coq}
IHlr : forall n : nat,
     is_redblack n lr
     -> is_redblack n rl
       -> nearly_redblack n (append lr rl)
         /\ (notred lr -> notred rl -> is_redblack n (append lr rl))
IHrl : forall n : nat,
     is_redblack n (T B ll lx lr)
     -> is_redblack n rl
       -> nearly_redblack n (append (T B ll lx lr) rl)
         /\ (notred (T B ll lx lr)
            -> notred rl -> is_redblack n (append (T B ll lx lr) rl))
n : nat
H1 : is_redblack (S n) (T B ll lx lr)
H2 : is_redblack (S n) (T B rl rx rr)
IH : nearly_redblack n (append lr rl)
______________________________________(1/1)
nearly_redblack (S n) (append (T B ll lx lr) (T B rl rx rr))
/\ (notred (T B ll lx lr)
   -> notred (T B rl rx rr)
     -> is_redblack (S n) (append (T B ll lx lr) (T B rl rx rr)))
\end{minted}

Proseguimos con la separaci\'on de la conjunci\'on, lo cual nos da dos casos que trabajaremos por
separado:

\paragraph{Primera mitad de conjunci\'on}

\begin{minted}{coq}
______________________________________(1/1)
nearly_redblack (S n) (append (T B ll lx lr) (T B rl rx rr))
\end{minted}

Despues de simplificar la meta de este caso, nos queda una meta que depende del resultado de una
llamada recursiva a $append$ de los subarboles $lr$ y $rl$, lo cual nos genera otros dos casos:

\begin{minted}{coq}
______________________________________(1/2)
is_redblack (S n) (lbalS ll lx (T B E rx rr))
______________________________________(2/2)
is_redblack (S n)
  match c with
  | R => T R (T B ll lx r1) a0 (T B r2 rx rr)
  | B => lbalS ll lx (T B (T c r1 a0 r2) rx rr)
  end
\end{minted}

Como podemos ver en ambas metas, tenemos una funci\'on nueva, $lbalS$, esta es una funci\'on de
balanceo,la cual extiende a las funciones que ya se habian usado con anterioridad en la funci\'on
de inserci\'on, como lo son $rbal'$, $rbal$ y $lbal$.

Para poder resolver esta parte de la demostraci\'on nos apoyaremos de otro lema, figura
\ref{lema_5}, el cual ilustra una propiedad de la operaci\'on $lbalS$.

\begin{figure}[!ht]
\centering
\captionsetup{justification=centering}
\begin{minted}{coq}
Lemma lbalS_rb {a} `{GHC.Base.Ord a}
(n : nat) (l : RB a) (x : a ) (r : RB a) :
nearly_redblack n l -> is_redblack (S n) r ->
              notred r -> is_redblack (S n) (lbalS l x r).
\end{minted}
\caption{Lema $lbalS\_rb$.}
\label{lema_5}
\end{figure}


Lo que el lema, arriba escrito en sintaxis de coq, quiere decir es que si tenemos un par de
arboles, sean l y r, un n\'umero natural n y un elemento x, si el \'arbol $l$ cumple con la
definici\'on inductiva $nearly\_redblack$ y r no es de color rojo y cumple con la definici\'on
inductiva $is\_redblack$, entonces balancear estos dos arboles con $lbalS$ resulta en un {\arn}
que cumple con la definici\'on $is_redblack$.

Ls demostraci\'on de este lema se convierte en un analisis de casos en el cual solamente es
necesario simplificar, aplicar las definiciones inductivas y las metas que se generan son
consecuencias directas de las hipotesis generadas, la inducci\'on no es necesaria.

Regresando a las dos metas generadas por destruir la func\'on append, si nos fijamos en la primera
, podemos ver que se puede aplicar directamente el lema $lbalS_rb$, lo cual nos genera 3 nuevas metas:

\begin{minted}{coq}
H1 : is_redblack (S n) (T B ll lx lr)
H2 : is_redblack (S n) (T B rl rx rr)
IH : nearly_redblack n E
______________________________________(1/3)
nearly_redblack n ll
______________________________________(2/3)
is_redblack (S n) (T B E rx rr)
______________________________________(3/3)
notred (T B E rx rr)
\end{minted}

Estas metas de nuevo caen en el caso de simplificar y aplicar las respectivas definiciones
inductivas para obtener las metas deseadas, de esta manera el primer caso queda resuelto.

Ahora nos vamos al segundo caso generado al destruir la funci\'on $append$ el cual nos dice que
tenemos que hacer un analisis de casos sobre el color del nodo:

\begin{minted}{coq}
H1 : is_redblack (S n) (T B ll lx lr)
H2 : is_redblack (S n) (T B rl rx rr)
IH : nearly_redblack n (T R r1 a0 r2)
______________________________________(1/2)
is_redblack (S n) (T R (T B ll lx r1) a0 (T B r2 rx rr))
______________________________________(2/2)
is_redblack (S n) (lbalS ll lx (T B (T B r1 a0 r2) rx rr))
\end{minted}

Ese analisis de casos nos da dos metas nuevas, una por color, el primer caso solamente requiere
simplificaci\'on y aplicaci\'on de definiciones inductivas para obtener las metas deseadas. El
segundo caso sigue los mismos pasos con la unica diferencia se volver a aplicar el lema $lbalS$.

De esta manera queda demostrada la primera mitad de la conjunci\'on.

\paragraph{Segunda mitad de conjunci\'on}

\begin{minted}{coq}
______________________________________(1/1)
notred (T B ll lx lr)
-> notred (T B rl rx rr)
  -> is_redblack (S n) (append (T B ll lx lr) (T B rl rx rr))
\end{minted}

Esta segunda mitad sigue exactamente el mismo procedimiento antes descrito con la \'unica
diferencia de que se agregan hip\'otesis nuevas:

\begin{minted}{coq}
H1 : is_redblack (S n) (T B ll lx lr)
H2 : is_redblack (S n) (T B rl rx rr)
IH : nearly_redblack n (append lr rl)
H3 : notred (T B ll lx lr)
H4 : notred (T B rl rx rr)
______________________________________(1/1)
is_redblack (S n)
  match append lr rl with
  | T R lr' x rl' => T R (T B ll lx lr') x (T B rl' rx rr)
  | _ => lbalS ll lx (T B (append lr rl) rx rr)
  end
\end{minted}

Al hacer el analisis de casos destruyendo la funci\'on $append$ con los parametros $lr$ y $rl$,
obtenemos exactamente las mismas metas que en la primera parte de la conjunci\'on y al tener mas
hipotesis la demostraci\'on se acorta por un par de pasos pero el procedimiento es el mismo.

De esta manera el primer lema queda demostrado, con esta demostraci\'on, a pesar de ser larga y
tediosa, se puede observar el poder del asistente de pruebas, ya que las demostraciones se reducen
a álgebra ecuacional, es decir, tratar de igualar la meta con lo que se tiene como hip\'otesis,
esto se seguira viendo en las siguientes pruebas.

\subsubsection{Segundo Lema}
Este siguiente lema utiliza una palabra especial en el lenguaje de {\coq}, `with', esta palabra
es un truco para demostrar dos lemas simultanemente, el cual es el caso como se ve en la figura
\ref{lema_6}.

\begin{figure}[!ht]
\centering
\captionsetup{justification=centering}
\begin{minted}{coq}
Lemma del_arb {a} `{GHC.Base.Ord a} (s:RB a) (x:a) (n:nat) :
      is_redblack (S n) s ->
      isblack s -> nearly_redblack n (del x s)
 with del_rb  {a} `{GHC.Base.Ord a} (s:RB a) (x:a) (n:nat) :
      is_redblack n s ->
      notblack s -> is_redblack n (del x s).
\end{minted}
\caption{Lema $del\_arb$}
\label{lema_6}
\end{figure}



Como podemos ver, el lema en si define dos lemas, esto se hace de esta manera porque la
demostraci\'on de uno de estos lemas depende del otro. De esta manera podemos definir ambos lemas y
solo usar una sola prueba.

\begin{minted}{coq}
del_arb : forall (a : Type) (H : Base.Eq_ a) (H0 : Base.Ord a)
        (s : RB a) (x : a) (n : nat),
        is_redblack (S n) s ->
        isblack s -> nearly_redblack n (del x s)
del_rb : forall (a : Type) (H : Base.Eq_ a) (H0 : Base.Ord a)
        (s : RB a) (x : a) (n : nat),
        is_redblack n s ->
        notblack s -> is_redblack n (del x s)
______________________________________(1/2)
is_redblack (S n) s -> isblack s -> nearly_redblack n (del x s)
______________________________________(2/2)
is_redblack n s -> notblack s -> is_redblack n (del x s)
\end{minted}

Lo que el asistente de pruebas esta haciendo es que nos esta integrando al ambiente de hip\'otesis
los dos lemas, de esta manera podemos realizar suposiciones con estos y asi ayudarnos a demostrar
los lemas. Realizaremos las pruebas de estos lemas por separado.

\paragraph{Prueba de $del\_arb$}

\begin{minted}{coq}
______________________________________(1/1)
is_redblack (S n) s -> isblack s -> nearly_redblack n (del x s)
\end{minted}

Este lema enuncia lo siguiente: Sea un \'arbol $s$, un elemento $x$ y un n\'umero natural $n$, si
$s$ cumple con la definicion inductiva $is\_redblack$ y $s$ es negro, entonces $s$ cumple con la
definición de $nearly\_redblack$ despues de eliminar el elemento $x$. En otras palabras, si tenemos
un \'arbol con la raiz de color negro, el resultado de eliminar un elemento ser\'a un \'arbol casi
 rojinegro.

La prueba empieza con una inducci\'on sobre $s$ lo cual nos da las dos metas siguientes:
\begin{minted}{coq}
______________________________________(1/2)
forall n : nat, is_redblack (S n) E ->
                isblack E -> nearly_redblack n (del x E)
______________________________________(2/2)
forall n : nat,
  is_redblack (S n) (T c s1 a0 s2) ->
  isblack (T c s1 a0 s2) ->
  nearly_redblack n (del x (T c s1 a0 s2))
\end{minted}

A primera vista podemos apreciar que la primera meta contiene un antecedente falso, el \'arbol
vacio $E$ no puede ser negro, entonces esta meta es una contradicci\'on. La segunda meta podemos
ver que si analizamos los dos casos del color del \'arbol, el caso rojo es igualmente una
contradicci\'on por el mismo antecedente $isblack$. Esto solo nos deja con el caso negro de la
segunda meta.

\begin{minted}{coq}
IHs1 : forall n : nat, is_redblack (S n) s1 ->
                       isblack s1 -> nearly_redblack n (del x s1)
IHs2 : forall n : nat, is_redblack (S n) s2 ->
                       isblack s2 -> nearly_redblack n (del x s2)
______________________________________(1/1)
forall n : nat,
is_redblack (S n) (T B s1 a0 s2)
-> isblack (T B s1 a0 s2) ->
nearly_redblack n (del x (T B s1 a0 s2))
\end{minted}

Despues de introducir los antecedentes y simplificar la meta, se hace un analisis de casos sobre la
operaci\'on $del$, primero se ve el caso si el nodo a eliminar esta en el subarbol derecho y
despues en el izquierdo.

\begin{minted}{coq}
IHs1 : forall n : nat, is_redblack (S n) s1 ->
                       isblack s1 -> nearly_redblack n (del x s1)
IHs2 : forall n : nat, is_redblack (S n) s2 ->
                       isblack s2 -> nearly_redblack n (del x s2)
H1 : is_redblack (S n) (T B s1 a0 s2)
H2 : isblack (T B s1 a0 s2)
H6 : is_redblack n s1
H8 : is_redblack n s2
______________________________________(1/2)
nearly_redblack n
match s1 with
| T B _ _ _ => lbalS (del x s1) a0 s2
| _ => T R (del x s1) a0 s2
end
______________________________________(2/2)
nearly_redblack n
(if _GHC.Base.>_ x a0
 then
  match s2 with
  | T B _ _ _ => rbalS s1 a0 (del x s2)
  | _ => T R s1 a0 (del x s2)
  end
 else append s1 s2)
\end{minted}

Seguimos con el analisis de los dos casos:
\begin{itemize}
  \item En el primer caso seguimos con la destrucci\'on del \'arbol $s1$ para tener un analisis de
  casos, si simplificamos con las definciones inductivas estas metas, eventualmente encontramos
  metas de las siguientes formas:
\begin{minted}{coq}
is_redblack n (del x E)

is_redblack n (del x (T R s1_1 a1 s1_2))
\end{minted}
  Estos casos son casos particulares del lema $del\_rb$, para solucionar esto usamos una tactica de
  coq llamada $assert$, la cual nos deja agregar hip\'otesis, las cuales despues tendremos que
  demostrar, en este caso esta quedara demostrada al terminar de demostrar todo este lema.
  Entonces como en este caso estamos destruyendo $s1$, la hip\'otesis a agregar seria:
\begin{minted}{coq}
IHl' : forall n : nat, is_redblack n s1 ->
                  notblack s1 -> is_redblack n (del x s1)
\end{minted}
  Al agregarla al inicio de la prueba podemos solamente aplicarla cuando lleguemos a los casos
  arriba mencionados.
  \item El segundo caso se divide en dos, la primera parte es el caso espejo al pasado, se realiza lo mismo pero para el \'arbol $s2$, lo cual nos da la siguiente hipotesis a agregar:
\begin{minted}{coq}
IHr' : forall n : nat, is_redblack n s2 ->
                  notblack s2 -> is_redblack n (del x s2)
\end{minted}
  Se aplica de la misma manera y llegamos a la segunda parte donde nos resulta la siguiente meta:
\begin{minted}{coq}
______________________________________(1/1)
nearly_redblack n (append s1 s2)
\end{minted}
  La cual es un caso particular del lema antes demostrado $append\_arb\_rb$.
\end{itemize}

\paragraph{Prueba de $del\_rb$}

\begin{minted}{coq}
______________________________________(1/1)
is_redblack n s -> notblack s -> is_redblack n (del x s)
\end{minted}

Este segundo lema enuncia lo siguiente: Sea un \'arbol $s$, un elemento $x$ y un n\'umero natural
$n$, si $s$ cumple con la definicion inductiva $is\_redblack$ y $s$ no es negro, entonces $s$
cumple con la definición de $is\_redblack$ despues de eliminar el elemento $x$. En otras palabras,
si tenemos un \'arbol con la raiz de color rojo, el resultado de eliminar un elemento ser\'a un
\'arbol casi rojinegro.

El enunciado con respecto al anterior busca que el resultado sea mas especifico, pues la propiedad
de ser $is\_redblack$ es la que buscamos que las operaciones cumplan. Sin embargo la demostraci\'on
en terminos de {\coq} no es muy distinta; iniciamos con inducci\'on sobre s.

\begin{minted}{coq}
______________________________________(1/2)
forall n : nat, is_redblack n E ->
           notblack E ->
           is_redblack n (del x E)
______________________________________(2/2)
forall n : nat, is_redblack n (T c s1 a0 s2) ->
           notblack (T c s1 a0 s2) ->
           is_redblack n (del x (T c s1 a0 s2))
\end{minted}

La primera meta se soluciona simplificando hasta obtener la meta deseada, mientras que la segunda
meta se le hace un analisis sobre el color $c$, el cual arroja dos casos, rojo y negro. El caso de
que $c$ sea negro es una contradicci\'on porque no cumple con la meta de ser $notblack$, solamente
nos enfocaremos en el caso en que $c$ es rojo.

\begin{minted}{coq}
IHs1 : forall n : nat,
         is_redblack n s1 ->
         notblack s1 -> is_redblack n (del x s1)
IHs2 : forall n : nat,
         is_redblack n s2 ->
         notblack s2 -> is_redblack n (del x s2)
H1 : is_redblack n (T R s1 a0 s2)
H2 : notblack (T R s1 a0 s2)
______________________________________(1/1)
is_redblack n (del x (T R s1 a0 s2))
\end{minted}

Al igual que en $del\_arb$ se hacen los casos de si el elemento a eliminar esta en el subarbol
derecho o izquierdo. Otra similitud que esta prueba tiene con respecto con la pasada es que tambien
tenemos que agregar una hipotesis extra, en este caso de $del\_arb$. De aqui en adelente la prueba
es muy similar a la anterior, simplificar, aplicar definiciones inductivas hasta llegar a
contradicciones o a las metas deseadas.

En este lema se usan lemas auxiliares muy similares a $lbalS\_rb$\footnote{descrito en la prueba de
la  funci\'on $append$}, como su espejo $rbalS\_rb$ o sus contrapartes $rbalS\_arb$ y $lbalS\_arb$.
Estos son lemas de balanceo sencillos de demostrar pero muy largos, tediosos y repetitivos, por lo
tanto no se incluiran en este trabajo.

Hasta este momento solo se han demostrado partes de la operacion total de eliminación, como juntar
dos subarboles despu\'es de eliminar su ra\'iz, que pasa si eliminamos de un arbol con raiz roja o
de uno con raiz negra. En seguida uniremos todos estos lemas en uno.

\subsubsection{Instancia de la funci\'on de eliminaci\'on}

\begin{figure}[!ht]
\centering
\captionsetup{justification=centering}
\begin{minted}{coq}
Instance remove_rb s x : redblack s -> redblack (remove x s).
\end{minted}
\caption{Instancia de eliminaci\'on de la clase $redblack$.}
\label{instance_del}
\end{figure}


Al igual que en la funci'on de inserción, terminamos la operación de eliminaci\'on generando una
instancia de la clase $redblack$, figura \ref{instance_del}, al igual que en la operación opuesta,
requerimos de un lema auxiliar con respecto a la clase $redblack$, figura \ref{lema_7}.

\begin{figure}[!ht]
\centering
\captionsetup{justification=centering}
\begin{minted}{coq}
Lemma makeBlack_rb {a} `{GHC.Base.Ord a} n t :
nearly_redblack n t -> redblack (makeBlack t).
\end{minted}
\caption{Lema $makeBlack\_rb$.}
\label{lema_7}
\end{figure}


Lo que este enunciado describe es la propiedad de que si un \'arbol $t$ cumple con la definici\'on
inductiva de ser $nearly\_redblack$, pintar su raiz de color negro lo convierte en una instancia de
la clase $redblack$. La desmostraci\'on de este lema es muy simple gracias al asistente de pruebas,
ya que solo basta con hacer un analisis de casos sobre el \'arbol $t$:

\begin{itemize}
  \item El \'arbol vacio $E$, la meta a demostrar para este caso es:
\begin{minted}{coq}
nearly_redblack n E -> redblack (makeBlack E)
\end{minted}
        Como la clase $redblack$ esconde un existencial en su definici\'on, para poder demostrar
        este caso basta con decir que existe $n$ con valor 0, esto nos da un caso trivial al ser la
        misma definici\'on inductiva de $is\_redblack$.
  \item El segundo caso se reduce a los dos casos en los que puede caer la definici\'on inductiva
  de $nearly\_redblack$.
\begin{minted}{coq}
H1 : nearly_redblack n (T c t1 a0 t2)
H2 : is_redblack n (T c t1 a0 t2)
______________________________________(1/2)
redblack (makeBlack (T c t1 a0 t2))
\end{minted}
        Este primer caso se reduce a hacer un analisis de casos sobre el color $c$, la soluci\'on
        de ambos colores consiste en, decir que existe $n'$ tal que su valor es \textit{S(n)},
        despuésde esto se simplifican las expresiones hasta obtener que las metas cumplan con las
        hip\'otesis.
\begin{minted}{coq}
H1 : nearly_redblack n (T c t1 a0 t2)
H2 : redred_tree n (T c t1 a0 t2)
______________________________________(2/2)
redblack (makeBlack (T c t1 a0 t2))
\end{minted}
        El segundo caso es mas corto que el primero, ya que al hacer el analisis de los colores,
        podemos ver que la definición de $redred\_tree$ no esta definida para \'arboles negros,
        entonces solo nos queda demostrar para \'arbles rojos, sin embargo, los pasos a seguir para
        este caso son los mismos que para el color rojo del caso anterior.
\end{itemize}

Con este lema demostrado ya contamos con todas las herramientas para poder demostrar que si tenemos
un \'arbol que es instancia de la clase $redblack$ y eliminamos un elemento de el, este sigue
siendo instancia de la clase, esta demostraci\'on comienza con un analisis de casos sobre el
\'arbol $s$:

\begin{minted}{coq}
H1 : is_redblack n E
______________________________________(1/2)
redblack (makeBlack (del x E))
______________________________________(2/2)
redblack (makeBlack (del x (T c s1 c0 s2)))
\end{minted}

Podemos ver que se hace uso de la funci\'on $makeBlack$. En la primera meta basta con aplicar el
lema $makeBlack\_rb$, simplificar y esta se soluciona. En la segunda meta se tiene que hacer otro
an\'alisis de casos, esta vez sobre el color:

\begin{minted}{coq}
H1 : is_redblack n (T R s1 c0 s2)
______________________________________(1/2)
redblack (makeBlack (del x (T R s1 c0 s2)))

\end{minted}

En la primera meta, color rojo, comenzamos por aplicar $makeBlack\_rb$, el cual despu\'es de
simplificar con la definci\'on inductiva nos resulta en la siguiente meta:

\begin{minted}{coq}
H1 : is_redblack n (T R s1 c0 s2)
______________________________________(1/1)
is_redblack n (del x (T R s1 c0 s2))
\end{minted}

La cual es un caso particular del lema $del\_rb$, nos basta con aplicarlo, simplificar y esta meta
queda resuelta. La \'unica meta que nos quedaria por demostrar seria el caso de de la raiz negra:

\begin{minted}{coq}
H1 : is_redblack n (T B s1 c0 s2)
______________________________________(1/1)
redblack (makeBlack (del x (T B s1 c0 s2)))
\end{minted}

En este caso hacemos un analisis sobre $n$, los dos casos serian 0 y \textit{S(n)}. Para 0 basta
con simplificar y la meta se resuelve, pero para \textit{S(n)}, es necesario volver a aplicar el
lema $makeBlack\_rb$, una vez que hacemos esto, nos queda la siguiente meta:

\begin{minted}{coq}
H1 : is_redblack (S n) (T B s1 c0 s2)
______________________________________(1/1)
nearly_redblack n (del x (T B s1 c0 s2))
\end{minted}

La cual resulta ser un caso particular del lema $del\_arb$, al aplicar este lema y simplificar
nuevamente, las metas resultantes quedan resueltas. Podemos ver que las demostraciones cubiertas en
este trabajo, en especial en la operaci\'on de eliminación, son muy repetitivas, y solo buscamos
hacer que las metas empaten con las hip\'otesis que tenemos.

Con esta operaci\'on desmostrada podemos decir que tenemos una estructura correcta y completa que
cumple con los invariantes descritos por las definiciones inductivas de $is\_redblack$ con las
operaciones de borrado e inserción y de la misma manera que estas dos operaciones son metodos de la
clase $redblack$, por lo cual podemos hacer estas operaciones cuantas veces queramos y el resultado
seguira siendo instancia de esta clase.
