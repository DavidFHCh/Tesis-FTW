\appendix
\chapter{Conceptos básicos de {\sc Coq}}

Para una mejor comprensi\'on de las pruebas realizadas en este trabajo, daremos una breve explicación
del asistente de pruebas {\coq}. Nos enfocaremos solamente en las herramientas que se usaron en este
trabajo.

Lo primero que se hace al querer probar un programa en Coq es escribirlo en su propio lenguaje. Este
contiene varias palabras reservadas para el desarrollo de funciones y de tipos especiales para el uso
de las mismas, en seguida explicaremos las que se usaron:

\begin{itemize}
  \item \textbf{\textit{Inductive}} - Esta palabra reservada se refiere a un tipo al cual se le puede aplicar inducc\'on
    estructural. Al usar esta palabra para definir a un tipo, se generan los constructores necesarios, y en algunos casos,
    destructores\cite{IndAndRec}. En el c\'odigo \ref{ind_example} presentamos un ejemplo de uso, definiendo los 
    n\'umeros naturales.
    \begin{listing}[!ht]
    \centering
    \captionsetup{justification=centering}
    \begin{minted}{coq}
      Inductive nat : Set :=
      | O : nat
      | S : nat -> nat. 
    \end{minted}
    \caption{Tipo Inductivo.}
    \label{ind_example}
    \end{listing}
  \item \textbf{\textit{Fixpoint}} - Esta palabra reservada se refiere a la definici\'on de una funci\'on recursiva, aplicada a, al menos, 
    un tipo inductivo definido con la palabra reservada \textbf{\textit{Inductive}}\cite{IndAndRec}. El c\'odigo \ref{fix_example} presentamos una funci\'on recursiva usando el tipo definido arriba. 
    \begin{listing}[!ht]
      \centering
      \captionsetup{justification=centering}
      \begin{minted}{coq}
        Fixpoint add (n m:nat) {struct n} : nat :=
        match n with
        | O => m
        | S p => S (add p m)
        end. 
      \end{minted}
      \caption{Funci\'on \textit{Fixpoint}.}
      \label{fix_example}
      \end{listing}
  \item \textbf{\textit{Definition}} - Esta palabra reservada b\'asicamente se refiere a una sustituci\'on textual de 
    una expresión por otra \cite{coqdefs}. En el c\'odigo \ref{def_example} se hace una sustituci\'on textual para aumentar en uno el valor de $n$.
    \begin{listing}[!ht]
      \centering
      \captionsetup{justification=centering}
      \begin{minted}{coq}
        Definition add_one (n :nat) {struct n} : nat :=
        match n with
        | O => S (O)
        | S p => S (S (p))
        end. 
      \end{minted}
      \caption{Sustituci\'on textual con \textit{Definition}.}
      \label{def_example}
      \end{listing}
\end{itemize}


Los tipos inductivos tambien los usamos para definir las invariantes o especificación formal del programa 
que se busca verificar. En el c\'odigo \hyperref[inductive_isRB]{$isRB$} definimos la especificación 
de los {\arns}.

Una vez escrito el programa y su especificación, procedemos a escribir lemas que nos ayuden a verificar
la correcci\'on con respecto a la especificación dada. Esto lo logramos con la palabra reservada 
\textbf{\textit{Lemma}} (ver \hyperref[lema_1]{\textit{ins\_rr\_rb}}). Al usar esta palabra Coq entra en modo 
prueba, con lo cual se pueden usar las \textit{t\'acticas}.

Las t\'acticas\cite{TACTICS} (ver c\'odigo \ref{tacticas_example}) son palabras como: \textit{intros}, \textit{destruct}, \textit{apply}, \textit{constructor}, etc. Estas 
t\'acticas nos ayudan a transformar el estado actual de una prueba, denominado \textit{meta}. Una
sucesión de t\'acticas eventualmente generan una prueba completa, la cual siempre empieza con la palabra
\textit{Proof} y termina con \textit{Qed}.

\begin{listing}[!ht]
\centering
\captionsetup{justification=centering}
\begin{minted}{coq}
(* lema que prueba que un arbol con al menos un elemento no es vacio.
 *)
Lemma T_neq_E {a} `{GHC.Base.Ord a}:
  forall (c:Color) (l: RB a) (k: a) (r: RB a), T c l k r <> E.
Proof.
intros. intro Hx. inversion Hx.
Qed.
\end{minted}
\caption{Ejemplo del uso de t\'acticas y lemas.}
\label{tacticas_example}
\end{listing}

Las metas (ver c\'odigo \ref{meta_example}) consisten de dos partes: la conclusi\'on y un contexto. El segundo contiene las hip\'otesis, definiciones y variables que podemos 
usar para demostrar la conclusi\'on con el uso de las t\'acticas.\cite{GOALS}. 
\begin{listing}[!ht]
\centering
\captionsetup{justification=centering}
\begin{minted}{coq}
H1_ : is_redblack n l
H1_0 : is_redblack n r
IHis_redblack1 :
    ifred l (redred_tree n (ins x l)) (is_redblack n (ins x l))
IHis_redblack2 :
    ifred r (redred_tree n (ins x r)) (is_redblack n (ins x r))
______________________________________(1/2)
is_redblack (S n) (lbal (ins x l) k r)
______________________________________(2/2)
is_redblack (S n) (rbal l k (ins x r))
\end{minted}
\caption{Ejemplo de una \textit{meta}.}
\label{meta_example}
\end{listing}
En este trabajo se habla en alto nivel como se usaron las t\'acticas para verificar los {\arns}.



