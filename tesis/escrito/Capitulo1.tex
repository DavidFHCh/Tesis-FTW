\chapter{Introducción}

\section{Motivación}
\label{motivation}
Hoy en día en el desarrollo de software existe un conjunto de normas a las cuales se les 
denomina \textit{buenas pr\'acticas de programaci\'on}. Algunos ejemplos de estas practicas ser\'ian: elegir si usaremos espacios o tabuladores para realizar la 
indentaci\'on, la documentaci\'on del c\'odigo, respetar las convenciones del lenguaje que estamos 
usado y verificar que nuestro programa se comporta de la manera deseada. En espec\'ifico, se pueden 
realizar pruebas unitarias.

Las \textit{pruebas unitarias} nos ayudan a saber si un código tiene el comportamiento
que buscamos, pero esto s\'olo nos sirve hasta cierto punto; por ejemplo, si tenemos una funci\'on
que recibe un par de números naturales, para poder verificar que la funci\'on es correcta\footnote{Podemos decir que una funci\'on es correcta si para toda entrada, la funci\'on regresa la salida esperada respecto a una especificaci\'on.} se 
tendrían que probar todos los casos de entradas, es decir, todas las combinaciones de pares de 
números naturales que existan, sin embargo, estas combinaciones son infinitas y se necesitaría la 
misma cantidad de memoria y de tiempo para poder ejecutar una prueba unitaria exhaustiva. 
Teóricamente esto es posible, pero en la pr\'actica simplemente no contamos con los recursos 
suficientes.

Siendo as\'i, escribir una prueba unitaria exhaustiva no es factible, en tal caso ¿que podríamos
hacer para verificar una implementaci\'on? Escribir una prueba unitaria que itere sobre un conjunto
representativo de los datos que la funci\'on puede recibir como argumentos es una opci\'on. Sin embargo, ¿qu\'e se 
puede esperar si la prueba unitaria misma es errónea? No hay una respuesta clara para esto y la 
industria hoy en día utiliza métodos como el expuesto anteriormente para probar c\'odigo pero, 
ciertamente, esto no es suficiente para decidir si el programa es correcto respecto a una 
especificaci\'on.

La única manera en que se puede demostrar que una función o programa es correcto respecto a una 
especificación es mediante una prueba matemática formal. El problema con este método es que es muy 
complejo, enredoso y a veces tardado para usarse en la industria o en el día a día. A lo largo del 
tiempo se ha buscado la manera de hacer este proceso mas amigable para el programador. Un ejemplo de 
esto son los lenguajes de programación funcionales, como lo ser\'ia Haskell. Este paradigma lleva a 
los programas a un contexto donde la notaci\'on es muy parecida a lo que se usar\'ia en las 
matemáticas tradicionales, es decir, funciones que van de un conjunto de datos a otro. Damos como ejemplo la 
funci\'on que calcula el factorial de un n\'umero natural. Esta la escribiremos tanto en Haskell 
como en la notaci\'on que suele usarse en cursos de matemáticas tradicionales (ver figuras 
\ref{func_fact_hask} y \ref{func_fact_math}). 
Podemos apreciar c\'omo las definiciones son casi idénticas, ya que ambas cumplen los siguientes 
puntos:
\begin{itemize}
    \item Ambas definen el tipo de sus variables, o en otras palabras, el conjunto al que las 
    variables pertenecen.
    \item Ambas definiciones establecen de que tipo de dato toman sus argumentos, es decir, la 
    entrada de la funci\'on es un n\'umero natural al igual que el resultado.
    \item Podemos intercambiar `f' por `fac' en cualquiera de las dos definiciones y el significado 
    no cambiaría.
\end{itemize}
%agregar ejemplo de java
\begin{listing}[!ht]
\centering
\captionsetup{justification=centering}
\begin{minted}{haskell}
fac :: (Integral n) => n -> n
fac 0 = 1
fac n = n * fac (n - 1)
\end{minted}
\caption{Funci\'on factorial, Haskell.}
\label{func_fact_hask}
\end{listing}

\begin{listing}[!ht]
\centering
\captionsetup{justification=centering}
\begin{minted}[escapeinside=||,mathescape=true]{text}
Sea n |$\in$ $\mathbb{N}$| y |$f(n)$| la función factorial definida como sigue:
|$f: \mathbb{N} \xrightarrow{} \mathbb{N}$|
|$f(0) = 1$|
|$f(n) = n \times f(n-1)$|
\end{minted}
\caption{Funci\'on factorial.}
\label{func_fact_math}
\end{listing}

La gran similitud entre estas definiciones 
facilita la demostraci\'on formal de los programas escritos en estos tipos de lenguajes, sin
embargo, estas demostraciones son realizadas de manera tradicional usando l\'apiz y papel.
No obstante, la \'unica manera en que estemos seguros acerca de la correcci\'on de la
demostraci\'on es que otra persona lea, entienda y valide la misma. Todos estos procesos as\'i como 
la creaci\'on y revisi\'on de la demostraci\'on, están hechos por humanos, por lo tanto son 
susceptibles a errores.

En los últimos años han comenzado su desarrollando diversos programas que nos ayudan a solucionar este
tipo de problemas, como son los \textit{demostradores autom\'aticos} y los \textit{asistentes de
prueba}. Estas herramientas son sistemas o programas que realizan o gestionan demostraciones. 
Nosotros nos enfocaremos en el segundo uso; en particular en este trabajo usaremos el asistente 
interactivo de pruebas llamado {\coq}\footnote{Como en este trabajo no se profundizara sobre el 
funcionamiento de la herramienta, presentamos la pagina de referencia del mismo: 
\url{https://coq.inria.fr/}.}. Este nos guiar\'a por la prueba, llevando un control de 
los casos que nos falten por demostrar y las hip\'otesis disponibles para cada caso, todo esto sobre 
programas escritos en el lenguaje funcional del mismo asistente.

Sin embargo, el uso de este asistente genera otro problema: no podemos probar cualquier programa
escrito en un lenguaje funcional, primero tenemos que traducir este programa al lenguaje de {\coq}
para poder comenzar con las demostraciones. Aquí tenemos tres opciones: traducir a mano, o adaptar 
una implementación al lenguaje de {\coq}, o utilizar una herramienta que nos ayude a traducir. En 
este trabajo usaremos la segunda opci\'on, una herramienta llamada \textit{hs-to-coq} \cite{thrc}, 
para probar formalmente la correcci\'on\footnote{Del ingl\'es \textit{correctness}.} de una estructura 
de datos como lo son los {\arns}\footnote{M\'as adelante se definir\'a esta clase de \'arboles.}.

Tomando como referencia el trabajo \cite{tesisG}, en donde se realizaron diversas implementaciones 
de {\arns} usando el lenguaje de programaci\'on Haskell, por ejemplo, se desarrollaron 
implementaciones usando constructores inteligentes y otra implementaci\'on mas compleja usando tipos 
anidados. La principal motivación de elegir esta estructura de datos no trivial para este trabajo fue
realizar una verificaci\'on formal de la implementaci\'on de constructores inteligentes. Sin embargo,
por motivos que se tocaran en secciones posteriores, realizar una traducci\'on del c\'odigo de dicha tesis
resulto no factible. Lo cual nos obligo a tomar otro camino: tomar la implementación de Coq de {\arns}
con pequeñas adecuaciones sacadas del c\'odigo de la tesis antes mencionada, entre otras cosas que
se explicaran mas adelante.

\section{\Arns}
Los {\arns} son una estructura de datos donde sus operaciones de inserci\'on, eliminaci\'on y
búsqueda se efectúan en tiempo logarítmico, es decir, la complejidad de estas operaciones es
$O(\log(n))$. Los {\arns} son una subclase de los \'arboles binarios de búsqueda, en los cuales la
complejidad de dichas operaciones crece hasta $O(n)$, como si estos fueran una lista simple o
doblemente ligada. Esta mejora se obtiene gracias a la introducción de colores a los nodos del
\'arbol (rojo y negro, de ah\'i rojinegros) y a invariantes relacionados con estos, los
cuales describiremos en la siguiente definici\'on.

\begin{defn}{({\Arns})}
\label{defn_arn}
Un \'arbol binario de búsqueda es un {\arn}\cite{canekED} si satisface lo siguiente:
\begin{enumerate}
    \item Todos sus nodos son rojos o negros.
    \item La ra\'iz es negra 
    \item Las siguientes invariantes se tienen que cumplir:
    \begin{itemize}
        \item Un nodo rojo debe tener hijos negros.
        \item Todos los caminos de la raíz a las hojas deben tener la misma cantidad de nodos
        negros.
        \item Todas las hojas del \'arbol son vacías y de color negro.
    \end{itemize}
\end{enumerate}
Dec\'imos que un \'arbol es negro o rojo si la ra\'iz es de ese color. Tambien podemos decir que la
altura negra de un nodo es el n\'umero de nodos negros que aparecen en el camino de 
ese nodo a la raíz.
\end{defn}

\begin{figure}[!ht]
\centering
\captionsetup{justification=centering}
 \begin{tikzpicture}[-,level/.style={sibling distance = 5cm/#1,level distance = 1cm}]
%\draw[style=help lines] (-5,-7) grid (5,0);
\node [arn_n] at (-2,0) {$6_0$}
        child{ node [arn_r] {$2$}
            child{ node [arn_n] {$1_1$}
                child{ node [arn_xx] {$x_2$}} %for a named pointer
                child{ node [arn_xx] {$x_2$}}
            }
            child{ node [arn_n] {$4_1$}
                child{ node [arn_r] {$3$}
                    child{ node [arn_xx] {$x_2$}}
                    child{ node [arn_xx] {$x_2$}}
                }
                child{ node [arn_r] {$5$}
                    child{ node [arn_xx] {$x_2$}}
                    child{ node [arn_xx] {$x_2$}}
                }
            }
        }
        child{ node [arn_n] {$8_1$}
            child{ node [arn_xx] {$x_2$}}
            child{ node [arn_r] {$9$}
                child{ node [arn_xx] {$x_2$}}
                child{ node [arn_xx] {$x_2$}}
            }
        }
;

\end{tikzpicture}
\caption {\Arn}
\label{arbolRB_1}
\end{figure}

En la figura \ref{arbolRB_1} podemos ver un ejemplo de un {\arn} que respeta la definici\'on que
acabamos de presentar: las etiquetas de los nodos representan la informaci\'on que se puede 
almacenar en ellos, siendo en este caso n\'umeros naturales, las etiquetas $x$ en las hojas 
representan los nodos vacíos y estos son negros, y los subíndices que aparecen en los nodos 
negros representan la altura negra del sub\'arbol. Con este ejemplo se ilustra c\'omo la cantidad de 
nodos negros de la ra\'iz a cualquier hoja es siempre la misma.

Nos interesa estudiar este tipo no trivial de \textit{\'arboles binarios de búsqueda} para poder
demostrar la correcci\'on de la implementaci\'on funcional en \cite{tesisG} usando el asistente de 
pruebas {\coq}. De esta manera mostraremos las ventajas y desventajas del proceso de verificaci\'on de una estructura de datos, el cual 
comienza escribiendo de cero una estructura o traduciendo la misma del lenguaje Haskell a {\coq}, y 
de igual manera las demostraciones que se realizar\'an con el asistente de pruebas. En este trabajo se nos enfocaremos en c\'odigo ya traducido. 

\section{Pruebas unitarias}
Las \textit{pruebas unitarias} \cite{unittest} son bloques de c\'odigo, funciones o m\'etodos, que
invocan a otros bloques para poder verificar ciertas suposiciones sobre el programa a probar. Estas
pruebas en principio deben de ser fáciles de escribir, entender, extender, deben ejecutarse en poco
tiempo y sobre todo ser fidedignas. De nada nos servirían pruebas unitarias que estén mal
escritas o que estas mismas sean demasiado complejas y puedan contener errores.

Este tipo de pruebas son usadas para verificar que cada componente de un programa funcione de
la manera esperada. En el caso de los {\arns} este tipo de pruebas nos ayudan a verificar los
invariables de un determinado \'arbol. La figura \ref{unitTestjava} es una prueba unitaria
escrita en el lenguaje de programaci\'on Java, la cual verifica la altura negra de un {\arn} \cite{CanekPU}.

\begin{listing}[!ht]
\centering
\captionsetup{justification=centering}
\begin{minted}{java}

    /* Valida que los caminos del vértice a sus hojas tengan todos
       el mismo número de vértices negros. */
    private static <T extends Comparable<T>> int
    validaCaminos(ArbolRojinegro<T> arbol,
                  VerticeArbolBinario<T> v) {
        int ni = -1, nd = -1;
        if (v.hayIzquierdo()) {
            VerticeArbolBinario<T> i = v.izquierdo();
            ni = validaCaminos(arbol, i);
        } else {
            ni = 1;
        }
        if (v.hayDerecho()) {
            VerticeArbolBinario<T> d = v.derecho();
            nd = validaCaminos(arbol, d);
        } else {
            nd = 1;
        }
        Assert.assertTrue(ni == nd);
        switch (arbol.getColor(v)) {
        case NEGRO:
            return 1 + ni;
        case ROJO:
            return ni;
        default:
            Assert.fail();
        }
        // Inalcanzable.
        return -1;
    }

\end{minted}
\caption{Prueba unitaria escrita en Java\cite{CanekPU}.}
\label{unitTestjava}
\end{listing}

Sin embargo, como se menciono en la motivaci\'on, el hecho de que las pruebas unitarias puedan verificar las invariantes de un \'arbol
dado no nos asegura que todos los \'arboles creados por nuestras operaciones de inserci\'on y
eliminaci\'on las respeten. La \'unica manera de que esta prueba podr\'ia verificar esto ser\'ia
realizando todas las combinaciones de operaciones y entradas posibles y aplicar la prueba a todos los
resultados de estas entradas. Esto es una prueba exhaustiva y en este caso\footnote{De hecho en la
mayoría.} las posibilidades son infinitas, es decir, no existe modo de realizar todas estas pruebas
en un tiempo razonable. M\'as a\'un, no existe memoria suficiente para almacenar los infinitos valores que se necesitar\'ian para realizar dichas pruebas. 

Finalmente, como las pruebas unitarias no nos permiten cubrir todos los casos necesarios para poder probar en su totalidad un programa, es que se requiere utilizar algún otro tipo de herramienta para realizar una verificaci\'on, en este trabajo usaremos {\coq}.



\section{Traducción de Haskell a {\coq}}
El enfoque que hemos decidido darle a este trabajo consiste en considerar una implementaci\'on de 
la estructura descrita en la secci\'on \ref{motivation} y solamente realizar la verificaci\'on formal de la 
misma. Para este trabajo decidimos usar {\coq}. Este asistente nos gu\'ia a través de 
la prueba recordando cu\'ales son las \textit{metas}\footnote{Una meta consiste de dos cosas: la 
conclusi\'on y un contexto. El segundo contiene las hipotesis, definiciones y variables que podemos 
usar para demostrar la conclusi\'on.\cite{GOALS}.} que debemos demostrar y tambi\'en nos ofrece 
\textit{t\'acticas}\footnote{Una táctica es un comando que especifica c\'omo se va a transformar el 
estado actual de una prueba. Una sucesión de tácticas eventualmente generan una prueba 
completa.\cite{TACTICS}.} para poder resolver dichas metas. Sin embargo, \'esta tambi\'en nos presenta 
nuevos desaf\'ios.

Al comenzar a utilizar el asistente nos encontramos con la problem\'atica de decidir si vamos a 
escribir el programa directamente en el lenguaje de {\coq} o si lo que buscamos es traducir un 
programa ya existente al lenguaje del asistente de pruebas. Lo que se busca es poder verificar un 
programa antes escrito, es decir, para el caso de los {\arns} queremos traducir este código de 
Haskell a {\coq}.

En el articulo `Total Haskell is Reasonable Coq' \cite{thrc} se describen las principales ventajas
y desventajas de traducir de Haskell a {\coq}, los cuales expondremos a continuaci\'on.

\subsection{Ventajas}
\begin{itemize}
    \item Haskell es un excelente lenguaje para escribir programas funcionales puros.
    \item La gran comunidad de programadores que usan y mantienen Haskell.
    \item El compilador GHC de Haskell, el cual es muy usado e incluso a nivel industrial.
    \item El ambiente de {\coq} es ``amigable'' para desarrollar demostraciones formales.
    \item {\coq} permite razonar acerca de programas funcionales totales.
\end{itemize}

\subsection{Desventajas}
\begin{itemize}
    \item Haskell utiliza el razonamiento ecuacional, por lo que en general no se usa un lenguaje 
    formal para realizar las demostraciones.
    \item Los programadores de Haskell razonan acerca de su código informalmente, si se llegan a
    realizar pruebas de \'este, generalmente est\'a hecho a mano ``en papel'', lo cual es tedioso y
    susceptible a errores.
    \item {\coq} no tiene la extensa biblioteca de funciones ni la misma cantidad de programadores
    que lo usen y mantengan como lo tiene Haskell.
    \item El hecho de que los programadores de Haskell s\'olo razonen acerca de su código
    informalmente puede que resulte en que se generen funciones parciales, es decir, que no se
    cubran todas la combinaciones de parámetros posibles para una funci\'on.
    \item La traducci\'on de Haskell a {\coq} s\'olo es posible si todas las funciones a traducir 
    son totales.
\end{itemize}

El art\'iculo propone el uso de una herramienta llamada \textit{hs-to-coq}, la cual actualmente se
encuentra en etapa de desarrollo y est\'a siendo usada para traducir código de Haskell a {\coq}. Por
las razones expuestas al comienzo de esta secci\'on es que decidimos usar esta herramienta de 
traducci\'on y enfocarnos \'unicamente en la verificaci\'on de los {\arns}.

\section{Sobre este trabajo}
El contenido y demostraciones que se describen en este trabajo se encuentran almacenados en:
\url{https://github.com/DavidFHCh/Tesis-FTW}. Aqu\'i presentamos definiciones, lemas y clases sin
incluir las demostraciones en {\coq}, en otras palabras, los \emph{scripts} de prueba. En su lugar se 
describen de manera informal las demostraciones para poder entender en alto nivel la estructura de 
la verificaci\'on formal realizada.

La herramienta \textit{hs-to-coq} fue utilizada para obtener las traducciones de las bibliotecas de Haskell; est\'as fueron
usadas para poder verificar la implementación, comenzando con el trabajo de \cite{tesisG} y 
adecuándolo a la implementaci\'on de \cite{MSetRBT} que se us\'o en el trabajo.

En los siguientes capítulos se describe el procedimiento usado para la verificaci\'on de la
estructura de datos, la cual esta enfocada a las invariantes de los cuatro puntos de la definici\'on
 \ref{defn_arn} dejando de lado la verificaci\'on de ser un \'arbol de b\'usqueda\footnote{El lector
  interesado puede revisar \cite{appel}.}.
