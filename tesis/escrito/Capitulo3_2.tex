\subsection{Verificación de la operación de inserción}

Para poder realizar la verificaci\'on de la operaci\'on de inserci\'on es necesario escribir
enunciados, ya sean lemas, proposiciones, etc. Estos enunciados los escribiremos usando las 
definiciones inductivas presentadas en la secci\'on pasada, es decir, 
\hyperref[inductive_isRedB]{$is\_redblack$}, \hyperref[inductive_isRedB]{$nearly\_redblack$} y 
\hyperref[inductive_isRedB]{$redred\_tree$}.

A continuaci\'on mostraremos los lemas \hyperref[lema_1]{$ins\_rr\_rb$}, 
\hyperref[lema_2]{$ins\_arb$} y una instancia \cite{classes} de la clase 
\hyperref[class_rb]{$redblack$}, $add\_rb$. Estos lemas y la instancia fueron obtenidos de 
\cite{MSetRBT}, la idea principal de estos enunciados es explicarnos que ciertos \'arboles de 
b\'usqueda que respeten las definiciones mas generales, es decir, 
\hyperref[inductive_isRedB]{$nearly\_redblack$} y \hyperref[inductive_isRedB]{$redred\_tree$}, 
también como consecuencia respetar\'an \hyperref[inductive_isRedB]{$is\_redblack$}.

\subsubsection{Primer Lema}

\begin{figure}[!ht]
\centering
\captionsetup{justification=centering}
\begin{minted}{coq}
Lemma ins_rr_rb {a} `{GHC.Base.Ord a} (x:a) (s: RB a) (n : nat) :
is_redblack n s ->
     ifred s (redred_tree n (ins x s)) (is_redblack n (ins x s)).
\end{minted}
\caption{Lema $ins\_rr\_rb$}
\label{lema_1}
\end{figure}

En este primer lema (figura \ref{lema_1}) enunciamos lo siguiente: sea \textit{\textbf{s}} un {\arn} 
bajo la definici\'on de \hyperref[inductive_isRedB]{$is\_redblack$}, entonces si \textit{\textbf{s}} 
es un \'arbol con raíz roja, insertar un elemento \textit{\textbf{x}} en \textit{\textbf{s}} resulta 
en un \'arbol que cumple la definci\'on de \hyperref[inductive_isRedB]{$redred\_tree$}, en otro caso 
cumple con la definici\'on de \hyperref[inductive_isRedB]{$is\_redblack$}.

En otras palabras, lo que este enunciado quiere decirnos es que si tenemos un {\arn} e insertamos
un elemento a ese \'arbol el resultado puede tener ra\'iz roja, e incluso puede tener dos nodos
rojos, uno en la ra\'iz y otro en cualquiera de los dos, o incluso en los dos nodos siguientes.

La demostraci\'on de este lema comienza con una inducci\'on sobre el antecedente del enunciado, lo 
cual resulta en tres casos:

 \begin{minted}{coq}
______________________________________(1/3)
ifred E (redred_tree 0 (ins x E)) (is_redblack 0 (ins x E))
______________________________________(2/3)
ifred (T R l k r) (redred_tree n (ins x (T R l k r)))
                  (is_redblack n (ins x (T R l k r)))
______________________________________(3/3)
ifred (T B l k r) (redred_tree (S n) (ins x (T B l k r)))
                  (is_redblack (S n) (ins x (T B l k r)))
 \end{minted}

La funci\'on $ifred$ que se usa en este lema, es una funci\'on auxiliar que nos ayuda a decidir si
un \'arbol es rojo o no.

En el primero de estos casos notamos que su soluci\'on se da simplificando las funciones y resulta
en uno de los casos de \hyperref[inductive_isRedB]{$is\_redblack$}, especificamente en el caso 
$RB\_R$, ya que el \'arbol vacío no es rojo y la simplificaci\'on de $(ins(x,E))$ resulta en un 
\'arbol rojo con un elemento, esto por definici\'on de \hyperref[func_ins]{$ins$}.

Los dos casos sobrantes estan relacionados con los colores de las raices del \'arbol, en el
segundo el \'arbol es rojo y en el tercero es negro.

Analicemos el segundo caso; como el \'arbol es rojo entramos al primer caso de $if\_red$, es decir,
al caso donde se aplica la definici\'on \hyperref[inductive_isRedB]{$redred\_tree$}, lo cual 
significa que al insertar un elemento al \'arbol rojo, sin tener conocimiento de como son los 
subárboles de este, puede resultar en un \'arbol con uno o dos nodos rojos consecutivos en la ra\'iz 
del mismo, ya que la operaci\'on de balanceo se fija en los nodos hijos y nietos del nodo al que se 
le aplica la operaci\'on, y como los nodos hijos de la raíz no tienen nodo abuelo, el balanceo no se 
efectúa en los nodos de la raíz, dando lugar a arboles con uno o mas nodos rojos consecutivos en la 
raíz\footnote{hasta 3, la raíz y sus hijos.}.

El caso sobrante, es decir, el caso del \'arbol con raíz negra se complica un poco mas que el 
anterior ya que este es el caso en el que la altura negra del \'arbol se ve modificada, en otras 
palabras, puede aumentar en uno. Este caso se verifica con las dos funciones de balanceo, 
\hyperref[func_balanceo]{$lbal$} y \hyperref[func_balanceo]{$rbal$}:

\begin{minted}{coq}
H1_ : is_redblack n l
H1_0 : is_redblack n r
IHis_redblack1 :
    ifred l (redred_tree n (ins x l)) (is_redblack n (ins x l))
IHis_redblack2 :
    ifred r (redred_tree n (ins x r)) (is_redblack n (ins x r))
______________________________________(1/2)
is_redblack (S n) (lbal (ins x l) k r)
______________________________________(2/2)
is_redblack (S n) (rbal l k (ins x r))
\end{minted}

Estos casos son análogos y los dos se resuelven simplificando las funciones de balanceo, en otros 
términos, simplificando las expresiones y aplicando las definiciones inductivas 
\footnote{\hyperref[inductive_isRedB]{$is\_redblack$}, 
\hyperref[inductive_isRedB]{$nearly\_redblack$}}.

\subsubsection{Segundo Lema}

\begin{figure}[!ht]
\centering
\captionsetup{justification=centering}
\begin{minted}{coq}
Lemma ins_arb {a} `{GHC.Base.Ord a} (x:a) (s:RB a) (n:nat) :
is_redblack n s -> nearly_redblack n (ins x s).
\end{minted}
\caption{Lema $ins\_arb$}
\label{lema_2}
\end{figure}

Este segundo lema (figura \ref{lema_2}) enuncia lo que en el cap\'itulo anterior se menciono acerca
de la funci\'on \hyperref[func_ins]{$ins$}: la funci\'on \hyperref[func_ins]{$ins$} no garantiza que 
el \'arbol resultante sea un {\arn}, ya que es posible que se termine la ejecuci\'on de la funci\'on 
con un nodo rojo como raíz. La demostraci\'on comienza introduciendo los antecedentes a las 
hipótesis y aplicando el lema anterior, \hyperref[lema_1]{$ins\_rr\_rb$}, a una de las hip\'otesis:

\begin{minted}{coq}
H1 : ifred s (redred_tree n (ins x s)) (is_redblack n (ins x s))
______________________________________(1/1)
nearly_redblack n (ins x s)

\end{minted}

Aplicamos el lema \hyperref[lema_1]{$ins\_rr\_rb$} a la hip\'otesis porque este nos genera dos 
casos, uno con la funci\'on \hyperref[inductive_isRedB]{$redred\_tree$} y otro con 
\hyperref[inductive_isRedB]{$is\_redblack$}. Lo que hace necesario que se aplique este lema, es que 
con ayuda de las definiciones que introduce son con las que 
\hyperref[inductive_isRedB]{$nearly\_redblack$} fue definido.

Como no se sabe si el \'arbol \textit{\textbf{s}} tiene ra\'iz roja o negra, se tienen que probar 
los dos casos: uno con la hipótesis de que el \'arbol resultante sea 
\hyperref[inductive_isRedB]{$is\_redblack$} y otro con \hyperref[inductive_isRedB]{$redred\_tree$}. 
Estos casos se resuelven sencillamente con la aplicación de alguno de los dos casos de la 
definici\'on inductiva de \hyperref[inductive_isRedB]{$nearly\_redblack$}, respectivamente.

\subsubsection{Instancia de la Funci\'on de Inserci\'on}

Para poder probar que el resultado de la inserci\'on es una instancia de la clase 
\hyperref[class_rb]{$redblack$}, es decir, que se cumplen con las propiedades descritas por la 
clase, vamos a necesitar el lema auxiliar que se encuentra en la figura \ref{lema_3}:
\begin{figure}[!ht]
\centering
\captionsetup{justification=centering}
\begin{minted}{coq}
Lemma makeBlack_rb {a} `{GHC.Base.Ord a} n t :
nearly_redblack n t -> redblack (makeBlack t).
\end{minted}
\caption{Lema $makeBlack\_rb$}
\label{lema_3}
\end{figure}

Este lema auxiliar se resuelve siguiendo la idea de que la definici\'on inductiva 
\hyperref[inductive_isRedB]{$nearly\_redblack$} únicamente viola las invariantes al poder tener dos 
nodos rojos consecutivos en su raíz, entonces, si se ``pinta'' la raíz de negro con la función 
\hyperref[raiz_negra_func]{$makeBlack$}, el \'arbol resultante es un {\arn} le cual respeta la la 
definición de la clase \hyperref[class_rb]{$redblack$}.

La figura \ref{instance_ins} enuncia la instancia de inserci\'on de la clase 
\hyperref[class_rb]{$redblack$}, la cual también nos generar\'a una demostraci\'on:

\begin{figure}[!ht]
\centering
\captionsetup{justification=centering}
\begin{minted}{coq}
Instance add_rb {a} `{GHC.Base.Ord a} (x:a) (s: RB a) :
redblack s -> redblack (insert x s).
\end{minted}
\caption{Instancia de inserci\'on de la clase \hyperref[class_rb]{$redblack$}.}
\label{instance_ins}
\end{figure}

Para poder crear la instancia de la clase \hyperref[class_rb]{$redblack$} es necesario usar la 
definici\'on \hyperref[raiz_negra_func]{$insert$}, la cual es una envoltura para la funci\'on 
\hyperref[func_ins]{$ins$}. Esta funci\'on lo que hace es ``pintar'' la ra\'iz del \'arbol 
resultante de la funci\'on \hyperref[func_ins]{$ins$} de color negro. De esta manera podemos 
asegurar que el \'arbol resultante ya no entra en la definici\'on de 
\hyperref[inductive_isRedB]{$redred\_tree$}.

\begin{minted}{coq}
H1 : is_redblack n s
______________________________________(1/1)
redblack (makeBlack (ins x s))
\end{minted}

En este momento se utiliza el lema auxiliar \hyperref[lema_3]{$makeBlack\_rb$} el cual nos devuelve 
lo siguiente:

\begin{minted}{coq}
H1 : is_redblack n s
______________________________________(1/1)
nearly_redblack n (ins x s)
\end{minted}

Esta ultima meta se resuelve aplicando el segundo lema que enunciamos, 
\hyperref[lema_2]{$ins\_arb$}, lo cual nos deja con una meta idéntica a la hipótesis H1 y con esto 
terminamos la verificaci\'on de la operaci\'on de inserci\'on.

Se puede decir que esta implementaci\'on de la funci\'on de inserci\'on es correcta respecto a las 
invariantes establecidas en la definici\'on inductiva \hyperref[inductive_isRedB]{$is\_redblack$}. 
La operaci\'on  ha sido verificada formalmente, ahora continuaremos con la funci\'on de 
eliminaci\'on.
