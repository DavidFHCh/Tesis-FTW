\chapter*{Conclusiones}
Como se ha ilustrado a lo largo de este trabajo, lo que buscamos es otra manera de demostrar la 
correcci\'on de una estructura de datos, en este caso de {\arns} usando un asistente de pruebas 
como lo es {\coq}, con traducciones de una biblioteca de tipos y funciones originarios de Haskell.

Hemos mencionado en repetidas ocasiones que la opci\'on mas tradicional para realizar una prueba de 
este estilo seria usar lápiz y papel, pero como se ha visto en capítulos anteriores el desarrollo de
la prueba puede llegar a tener demasiados casos. Esto lo convierte en una tarea muy complicada y 
tediosa de escribir, y posteriormente de leer y entender por alguien m\'as. En cambio, un asistente de 
pruebas como {\coq} ayuda en organizar esta tarea, también puede simplificar y logra reducirla a usar álgebra 
ecuacional ya que, como se vio mayoritariamente en este trabajo, lo \'unico que se busca obtener es que las metas que 
queremos probar se igualen con alguna de las hip\'otesis que se tienen. 
Esto también tiene sus detrimentos ya que se deja de razonar de manera formal.

Sin embargo, el uso de una herramienta de esta naturaleza por s\'i sola no simplifica del todo este
tipo de pruebas. Para poder llegar a un escenario donde se pueda desarrollar una 
demostraci\'on primero se debe tener claro qué es lo que se quiere probar y codificarlo en el 
lenguaje que la herramienta comprenda.

En la vida real, esto significaría tener un programa escrito en algún lenguaje de programaci\'on
como Java, Python, Haskell, etc. y traducirlo al lenguaje de la herramienta. Esto requeriría la
implentaci\'on de un traductor o en su defecto traducir los programas a mano, esta segunda opci\'on
siendo una soluci\'on no \'optima ya que es susceptible a errores. En este trabajo se us\'o el
traductor de Haskell a {\coq} llamado \textit{hs-to-coq} \cite{thrc}, que aunque nos dio algunas 
bibliotecas de Haskell traducidas a {\coq}, sigue en estado de desarrollo. Adem\'as, aunque Haskell 
comparte el mismo paradigma que el lenguaje de {\coq}, lograr traducir en un $100\%$ un lenguaje 
resulta muy complicado ya que \'este siempre est\'a evolucionando, en especial si es un lenguaje tan 
ampliamente usado como lo es Haskell.

Otra restricci\'on que se tiene que establecer es que no todos los programas escritos en Haskell 
podrían ser traducidos al lenguaje de {\coq}. Este lenguaje a pesar de que entra en la categoría de 
lenguajes funcionales, s\'olo acepta funciones totales. Esto introduce otras 
problemáticas: la traducción un programa de un paradigma imperativo, l\'ogico, etc. a uno funcional 
y después asegurar que todas las funciones de \'este son totales.

Supongamos que resolvemos todos estos problemas que se han presentado hasta ahora, es decir,
tenemos un programa donde todas sus funciones son totales y se logr\'o traducir correcta y
completamente. 
Siguiendo nuestra metodología, se tienen que generar las definiciones inductivas las cuales ayudar\'an a
guardar invariantes del programa, y con \'estas escribir los lemas que se buscan probar para poder
asegurar que el programa en cuesti\'on ha sido verificado formalmente, lo cual 
podría tomar el mismo tiempo que tom\'o traducir todo el programa al lenguaje de la herramienta.

Actualmente en la industria lo que se hace para minimizar los errores en c\'odigo es hacer que este
pase por una serie de filtros, es decir, que otra persona revise el c\'odigo para identificar defectos y ejecutar pruebas ya existentes. 
Con esto se busca asegurar que el nuevo c\'odigo no introduzca errores a componentes que funcionaban 
correctamente dentro del programa y que se escriban pruebas que confirmen el correcto funcionamiento del c\'odigo a introducir. 
En este momento la idea de poder probar que un programa cualquiera puede ser probado formalmente usando un asistente de 
pruebas es muy atractiva, ya que un \'unico desarrollador podr\'ia desarrollar la prueba y no 
depender de código ajeno que muestre que su programa es correcto. Sin embargo, esta idea resulta muy 
poco factible hoy en día ya que, además de los problemas expuestos con anterioridad (las 
traducciones del c\'odigo implementado), se suma el hecho de que se tendrían que traducir y probar 
todas las bibliotecas ocupadas del lenguaje que se est\'a usando antes de pensar en probar un 
programa usando una herramienta como {\coq}.

Otro acercamiento para poder probar este tipo de programas en la industria ser\'ia desarrollar la 
mayor parte de estos en el asistente {\coq}. Realizando esto con las herramientas que su lenguaje 
nos provee, de esta manera se pueden realizar las demostraciones pertinentes y utilizar la 
funcionalidad que \'este posee para extraer c\'odigo en otros lenguajes, después de haber realizado la 
prueba, como lo son Haskell y O'Caml. Sin embargo, esto s\'olo nos permitiría desarrollar programas 
correctos con las funcionalidades que el lenguaje de {\coq} nos ofrezca.

Retomando el punto anterior, otra soluci\'on ser\'ia desarrollar y probar partes clave de los 
programas a crear, es decir, m\'odulos pequeños como lo ser\'ian las estructuras de datos 
({\arns}, listas doblemente ligadas, colas, pilas, u otros tipos de \'arboles) o partes críticas de algún sistema. 
Una vez implementados estos m\'odulos se podrían usar en cualquier parte de c\'odigo. El inconveniente con 
hacer esto es que dependiendo del lenguaje al que se extraiga el programa probado, puede ser 
contraproducente para el desempeño del mismo. Este degradado en el desempeño se puede dar por 
razones ajenas al c\'odigo y por asuntos relacionados a la implementaci\'on del compilador que 
se usar\'a para generar c\'odigo ejecutable y la optimizaci\'on del mismo. Por ejemplo, el 
lenguaje C es conocido por ser muy usado en programas que requieren ser muy rápidos en sus 
operaciones.

Otro problema es que como no todo el co\'digo estaría probado formalmente, para componentes m\'as 
grandes se necesitaría caer nuevamente en hacer otro tipo de pruebas, como las unitarias y las de integración, y como ya 
hemos mencionado, \'estas no nos aseguran que los programas sean correctos o completos. Por lo tanto, 
nuestros programas s\'olo estarían verificados formalmente \textit{parcialmente}.

Como podemos apreciar, demostrar programas con un asistente de pruebas no es el procedimiento mas 
amigable hoy en día, el tiempo que se debe invertir en aprender la herramienta o el esfuerzo para 
obtener una verificación lo más amplia posible son desventajas claras de estos métodos. S
in embargo, si se sigue con la actual trayectoria en el desarrollo de 
herramientas de traducci\'on como lo es \textit{hs-to-coq}, eventualmente la industria podría 
comenzar a utilizar herramientas de este estilo para mejorar la calidad de sus productos. 
Mientras tanto, se tendrán que seguir desarrollando pruebas unitarias de mejor calidad y lograr generar 
programas que tiendan a la correcci\'on.
