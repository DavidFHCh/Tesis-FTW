 \chapter{Conclusiones}
Como se ha ilustrado a lo largo de este trabajo, lo que buscamos es otra manera de demostrar la correcci\'on de una estructura de datos,
en este caso de un {\arn} usando un asistente de pruebas como lo es {\coq} con una biblioteca de
tipos y funciones que se tradujeron de Haskell.

Hemos mencionado en repetidas ocasiones que la opci\'on mas tradicional para realizar una prueba de este estilo seria
usar lápiz y papel, pero como se ha visto en capítulos anteriores el desarrollo de la prueba puede llegar a generar demasiados
casos, esto lo convierte en una tarea muy complicada y tediosa de escribir, y posteriormente de leer y
entender por alguien mas. En cambio un asistente de pruebas como lo es {\coq} da herramientas para
simplificar esta tarea y logra reducirla a álgebra ecuacional, ya que como se vio en este trabajo,
lo \'unico que se busca obtener es que las metas que queremos probar se igualen con alguna de las
hip\'otesis que se tienen, lo cual también tiene sus detrimentos ya que se deja de razonar de manera formal.

Sin embargo, el uso de una herramienta de esta naturaleza por si sola no simplifica del todo este
tipo de pruebas, ya que para poder llegar a un escenario donde se pueda desarrollar una demostraci\'on primero
tenemos que tener claro que es lo que se quiere probar y codificarlo en el lenguaje que la
herramienta comprenda.

En la vida real, esto significaría tener un programa escrito en algún lenguaje de programaci\'on
como Java, Python, Haskell, etc. y traducirlo al lenguaje de la herramienta. Esto requeriría la
implentaci\'on de un traductor o en su defecto traducir los programas a mano, esta segunda opci\'on
siendo una soluci\'on no \'optima ya que es muy susceptible a errores. En este trabajo se uso el
traductor de Haskell a Coq llamado `hs-to-coq' \cite{thrc}, que aunque nos dio algunas bibliotecas
de Haskell traducidas a Coq, esta sigue en estado de desarrollo y aunque Haskell comparte el mismo
paradigma que el lenguaje de Coq, lograr traducir en un $100\%$ un lenguaje resulta muy complicado
ya que este siempre esta evolucionando, en especial si es un lenguaje tan ampliamente usado como
lo es Haskell.

Otra restricci\'on que se tiene que establecer es que no todos los programas escritos en Haskell podrían ser traducidos
al lenguaje de Coq, este lenguaje a pesar de que entra en la categoría de lenguajes funcionales, este solo acepta funciones totales. Entonces esto introduce otras problemáticas,
la traducción un programa de un paradigma imperativo, l\'ogico, etc. a uno funcional y después
asegurar que todas las funciones de este son totales.

Supongamos que resolvemos todos estos problemas que se han presentado hasta ahora, es decir,
tenemos un programa donde todas sus funciones son totales y se logro traducir correcta y
completamente. Ahora se tienen que generar las definiciones inductivas, las cuales te ayudaran a
guardar invariantes de tu programa, y con estas escribir los lemas que se buscan probar para poder
decir que tu programa ha sido verificado formalmente, lo cual podría tomar el mismo tiempo que tomo
traducir todo el programa al lenguaje de la herramienta.

Actualmente en la industria lo que se hace para minimizar los errores en c\'odigo, es hacer que este pase por una serie de filtros, es decir, que otra persona revise tu c\'odigo para ayudarte a encontrar defectos, también en ejecutar pruebas ya existentes para asegurar que el nuevo c\'odigo no introduzca errores a componentes que funcionaban correctamente dentro del programa
y que se escriban pruebas que confirmen el correcto funcionamiento del c\'odigo a introducir.
En este momento la idea de poder probar
que un programa cualquiera puede ser probado formalmente usando un asistente de pruebas es muy
atractiva, ya que un \'unico desarrollador podr\'ia desarrollar la prueba y no depender de código ajeno que muestre que su programa es correcto. Sin embargo, esta idea resulta muy poco factible hoy en día, ya que además de los problemas expuestos con
anterioridad (las traducciones del c\'odigo implementado) se le suma el hecho de que se tendrían que traducir y probar todas las
bibliotecas ocupadas del lenguaje que se esta usando, esto antes de pensar en probar tu programa.

Otro acercamiento para poder probar este tipo de programas en la industria seria desarrollar la
mayor parte de estos en el asistente {\coq}, realizando esto con las herramientas que su lenguaje nos provee, de esta manera se pueden realizar
las demostraciones pertinentes y utilizar la funcionalidad que este posee para extraer c\'odigo en
otros lenguajes, después de haber realizado la prueba, como lo son Haskell y O'Caml. Sin embargo, esto solo nos permitiría desarrollar
programas correctos con las funcionalidades que el lenguaje de Coq nos ofrezca.

Retomando el punto anterior, otra soluci\'on seria desarrollar y probar partes clave de los programas a crear, es decir, 
m\'odulos pequeños como lo serian las estructuras de datos a usar, como lo podrian ser los {\arns}, listas doblemente 
ligadas, colas, pilas u otros tipos de \'arboles. Una vez implementados estos m\'odulos se podrían usar en cualquier 
parte de c\'odigo, el problema con hacer esto es que dependiendo del lenguaje al que se extraiga el programa 
probado, puede ser contraproducente para el desempeño del mismo. Este degradado en el desempeño se puede dar por razones
ajenas al c\'odigo y mas por asuntos relacionados a la implementaci\'on del compilador que se usar\'a para generar 
c\'odigo ejecutable y que tan optimizado es el mismo. Por ejemplo, el lenguaje C es conocido por ser muy 
usado en programas que requieren ser muy rápidos en sus operaciones.

Otro problema es que como no todo el co\'digo estaría probado formalmente, para componentes mas grandes se necesitaría caer 
nuevamente en hacer otro tipo de pruebas, como las unitarias, y como ya hemos mencionado, estas no nos aseguran que los 
programas sean correctos o completos y por lo tanto, nuestros programas solo estarían parcialmente verificados formalmente.

Como podemos apreciar el demostrar programas con un asistente de pruebas, no es el procedimiento mas amigable hoy en día, 
sin embargo, si se sigue con la actual trayectoria en el desarrollo de herramientas de traducci\'on como lo es 'hs-to-coq', eventualmente la industria podría comenzar a utilizar herramientas de este estilo para mejorar la calidad de sus productos. Mientras tanto, se tendrán que seguir desarrollando pruebas unitarias de mejor calidad y lograr generar programas que tiendan a la correcci\'on y completud.
