%\documentclass[letterpaper,8pt,leqno,spanish]{book}
%\usepackage[top=1in, left=0.9in, right=1.25in, bottom=1in]{geometry}
\documentclass[8pt,leqno,pdflatex,spanish]{book}
%\usepackage[es-noquoting]{babel}
\usepackage{bachelorstitlepageUNAM}
\usepackage[utf8]{inputenc}
\usepackage{hyperref}

\usepackage[T1]{fontenc}
\usepackage[utf8]{inputenc}
\usepackage[spanish,es-nodecimaldot,es-tabla]{babel}
%\usepackage{graphicx}
%\usepackage{tikz}
%\usepackage{tocloft}
%\graphicspath{{./figs/}}
%\usepackage{setspace}
\usepackage{minted}
\usepackage{amsmath,amsfonts,amssymb,amsthm,epsfig,epstopdf,titling,array}
\usepackage{url}
\usepackage{caption}
\usepackage{breakcites}

\interfootnotelinepenalty=10000

\newcommand{\coq}{\textit{Coq}}
\newcommand{\Arns}{\'Arboles rojinegros}
\newcommand{\Arn}{Árbol rojinegro}
\newcommand{\arn}{árbol rojinegro}
\newcommand{\arns}{\'arboles rojinegros}
%\usepackage[round]{natbib}

%para pintar arboles bonitos...
\usepackage{tikz}
\usetikzlibrary{arrows}
\usetikzlibrary{positioning}
%\usetikzlibrary{arrows}


\tikzset{
treenode/.style = {align=center, inner sep=0pt, text centered,
font=\sffamily},
arn_n/.style = {treenode, circle, white, font=\sffamily\bfseries, draw=black,
fill=black, text width=1.5em},% arbre rouge noir, noeud noir
arn_r/.style = {treenode, circle, white, draw=red, fill= red,
text width=1.5em, very thick},% arbre rouge noir, noeud rouge
arn_x/.style = {treenode, rectangle, white, draw=black, fill= black,
minimum width=0.5em, minimum height=0.5em},% arbre rouge noir, nil
arn_xx/.style = {treenode, rectangle, white, draw=black, fill= black,
minimum width=0.7em, minimum height=0.7em},
arn_bb/.style = {treenode, circle, white, font=\sffamily\bfseries, draw=gray,
fill=gray, text width=1.5em},
arn_nb/.style = {treenode, circle, black, font=\sffamily\bfseries, draw=black,
fill=white, text width=1.5em},
arn_w/.style = {treenode, circle, white, font=\sffamily\bfseries, draw=brown,
fill=brown, text width=1.5em},
arn_xb/.style = {treenode, rectangle, white, draw=gray, fill= gray,
minimum width=0.5em, minimum height=0.5em}
}

%\renewcommand\cftsecpresnum{\S}
%\renewcommand\cftsubsecpresnum{\S}

\theoremstyle{plain}
\newtheorem{thm}{Teorema}[section]
\newtheorem{lem}[thm]{Lemma}
\newtheorem{prop}[thm]{Proposici\'on}
\newtheorem*{cor}{Corollary}

\theoremstyle{definition}
\newtheorem{defn}{Definci\'on}[section]
\newtheorem{exmp}{Ejemplo}[section]

\theoremstyle{remark}
\newtheorem*{rem}{Remark}
\newtheorem*{note}{Note}


\begin{document}
%------------------------------

\begin{titlepage}
\thispagestyle{empty}
\begin{minipage}[c][0.17\textheight][c]{0.25\textwidth}
\begin{center}
\includegraphics[height=3.5cm]{Escudo-UNAM.pdf}
\end{center}
\end{minipage}
\begin{minipage}[c][0.195\textheight][t]{0.75\textwidth}
\begin{center}
\vspace{0.3cm}
\textsc{\large Universidad Nacional Aut\'onoma de M\'exico}\\[0.5cm]
\vspace{0.3cm}
\hrule height2.5pt
\vspace{.2cm}
\hrule height1pt
\vspace{.8cm}
\textsc{Facultad de Ciencias}\\[0.5cm] %
\end{center}
\end{minipage}

\begin{minipage}[c][0.81\textheight][t]{0.25\textwidth}
\vspace*{5mm}
\begin{center}
\hskip2.0mm
\vrule width1pt height13cm
\vspace{5mm}
\hskip2pt
\vrule width2.5pt height13cm
\hskip2pt
\vrule width1pt height13cm \\
\vspace{5mm}
\includegraphics[height=3.5cm]{Escudo-FCIENCIAS.pdf}
\end{center}
\end{minipage}
\begin{minipage}[c][0.81\textheight][t]{0.75\textwidth}
\begin{center}
\vspace{1cm}

{\large\scshape Verificación formal de \'arboles rojinegros}\\[.2in]

\vspace{2cm}

\textsc{\LARGE T\hspace{1.5cm}E\hspace{1.5cm}S\hspace{1.5cm}I\hspace{1.5cm}S}\\[0.5cm]
\textsc{\large que para obtener el t\'itulo de:}\\[0.5cm]
\textsc{\large Licenciado en Ciencias de la Computación}\\[0.5cm]
\textsc{\large presenta:}\\[0.5cm]
\textsc{\large {David Felipe Hern\'andez Chiapa}}\\[2cm]

\vspace{0.5cm}

{\large\scshape Tutora:\\[0.3cm] {Dra. Lourdes del Carmen Gonz\'alez 
Huesca}}\\[.2in]

\vspace{0.5cm}

\large{Ciudad Universitaria, Ciudad de México,}{ }{2021}
\end{center}
\end{minipage}
\end{titlepage}



%---------------------------------
\frontmatter
%\maketitle
\chapter*{}
\begin{flushright}%
\emph{Dedicatoria ...}
\thispagestyle{empty}
\end{flushright}

\chapter{Agradecimientos}
Tesis realizada bajo el proyecto PAPIME 102117 ``T\'opicos en Ciencia de la Computaci\'on Te\'orica''.


\tableofcontents
\listoffigures
\renewcommand\listingscaption{C\'odigo}
\renewcommand\listoflistingscaption{\'Indice de c\'odigos}
\listoflistings


\mainmatter

\chapter{Introducción}

\section{Motivación}
Hoy en día en el desarrollo de software existe un conjunto de normas a las cuales se les denomina
\textit{buenas pr\'acticas de programaci\'on}, las cuales van desde tener una correcta indentaci\'on,
elegir si usaremos espacios o tabuladores para realizar este acomodo, la documentaci\'on del
c\'odigo, respetar las convenciones del lenguaje que estemos usado y probar nuestro programa, en
espec\'ifico, realizar pruebas unitarias.

Las \textit{pruebas unitarias} nos ayudan a saber si nuestro código esta teniendo el comportamiento que
buscamos, pero esto sol\'o nos sirve hasta cierto punto; por ejemplo, si tenemos una funci\'on que
recibe un par números naturales, para poder estar totalmente seguros de que la funci\'on es correcta
se tendrían que probar todos los casos, es decir, todas las combinaciones de números naturales que existan,
sin embargo, estas combinaciones son infinitas y se necesitaría la misma cantidad de memoria y de
tiempo para poder escribir una prueba unitaria exhaustiva, como la que se esta sugiriendo.
Teóricamente esto es posible, pero en la pr\'actica simplemente no contamos con recursos los recursos suficientes.

Siendo as\'i, escribir una prueba unitaria exhaustiva no es factible, en tal caso ¿que podríamos
hacer?; escribir una prueba unitaria que itere sobre un conjunto representativo de los datos que la
funci\'on puede recibir como par\'ametros. Sin embargo, ¿si la misma prueba unitaria es errónea?, no
hay una respuesta clara para esto y la misma industria hoy en día utiliza métodos, como el expuesto
anteriormente, para probar su c\'odigo pero ciertamente esto no nos dice si el programa es correcto o completo.

La única manera de que podamos probar que una función o programa es correcto y completo es mediante
una demostraci\'on matemática formal, el problema con este método es que es muy complejo y
complicado. A lo largo del tiempo se ha buscado la manera de hacer este proceso mas amigable al
programador, un ejemplo de esto son los lenguajes de programación funcionales.Este paradigma lleva%aqui va la correcion
a los programas a un contexto donde la notaci\'on es muy parecida a lo que se usa en las
matemáticas tradicionales, es decir, funciones que van de un conjunto de datos a otro. Esto
facilita la demostraci\'on formal de los programas escritos en estos tipos de lenguajes, sin embargo,
estas demostraciones son realizadas a mano y la \'unica manera de que estemos
seguros de que nuestra prueba es correcta es que otra persona la lea y la entienda. Como el proceso
de creaci\'on y revisi\'on de la prueba esta hecho por humanos, estos son susceptibles a errores.

En los últimos años se han estado desarrollando programas que ayudan con este tipo de problemas,
programas que nos ayuden para hacer demostraciones formales de programas escritos en lenguajes
funcionales, este tipo de programas se conoce como asistentes de prueba y en este trabajo usaremos
el asistente llamado `Coq'. Estos asistentes nos ayudan en la parte de la revisi\'on de la prueba,
al llevar el control de que casos hemos demostrado y cuales nos hacen falta.

Sin embargo, el uso de este asistente genera otro problema, no podemos probar cualquier programa
escrito en un lenguaje funcional, tenemos que primero traducir este programa al lenguaje de Coq
para poder comenzar con las demostraciones. Aquí tenemos dos opciones, traducir a mano o utilizar
una herramienta que nos ayude a traducir. En este trabajo usaremos la segunda opci\'on, el nombre
de la herramienta es `hs-to-coq' \cite{thrc}, lo que buscamos es lograr probar formalmente la
completud y correcci\'on de una estructura de datos como lo son los {\arns}.

En un trabajo anterior \cite{tesisG}, se realizaron diversas implementaciones de {\arns}, usando el
lenguaje de programaci\'on Haskell, se desarrollaron constructores inteligentes y una
implementaci\'on mas compleja usando tipos anidados. Siendo este trabajo la principal motivaci\'on
de elegir esta estructura de datos no trivial para realizar la demostraci\'on formal de su
corrección y completud.

\section{Arboles Roji-negros}
Los {\arns} son una estructura de datos donde las operaciones de inserci\'on, eliminaci\'on y
búsqueda se efectúan en tiempo logarítmico, es decir, la complejidad de esas operaciones es
$O(log(n))$, estos son una subclase de los arboles binarios de búsqueda, en los cuales la
complejidad de dichas operaciones crece hasta $O(n)$. Esta mejora en la complejidad se obtiene
gracias a la introducción de colores a los nodos del \'arbol y a invariantes relacionados con estos
colores, las cuales describiremos en la siguiente definici\'on.

\subsection{Definici\'on de {\arns}}
Un {\arn} se define de la siguiente manera:
Un \'arbol binario de búsqueda es un {\arn} si satisface lo siguiente:
\begin{enumerate}
    \item Todos sus nodos son rojos o negros.
    \item El \'arbol vació es negro.
    \item La raíz es negra \footnote{Decimos que un \'arbol es negro o rojo si el nodo de la raiz
    es de ese color.}.
    \item Las siguientes invariantes se tienen que cumplir:
    \begin{itemize}
        \item Un nodo rojo debe tener hijos negros.
        \item Todos los caminos de la raíz a las hojas deben tener la misma cantidad de nodos
        negros.
        \item Todas las hojas del \'arbol son vacías y de color negro \footnote{Esta invariante
        depende de la idea de que los nodos vacíos son de color negro, mas adelante se ver\'a que
        esta condici\'on no es del todo necesaria.}.
    \end{itemize}
\end{enumerate}

\begin{figure}
\centering
\captionsetup{justification=centering}
 \begin{tikzpicture}[-,level/.style={sibling distance = 4cm/#1,level distance = 1cm}]
%\draw[style=help lines] (-5,-7) grid (5,0);
\node [arn_n] at (-2,0) {6}
        child{ node [arn_r] {2}
            child{ node [arn_n] {1}
                child{ node [arn_x] {}} %for a named pointer
                child{ node [arn_x] {}}
            }
            child{ node [arn_n] {4}
                child{ node [arn_r] {3}
                    child{ node [arn_x] {}}
                    child{ node [arn_x] {}}
                }
                child{ node [arn_r] {5}
                    child{ node [arn_x] {}}
                    child{ node [arn_x] {}}
                }
            }
        }
        child{ node [arn_n] {8}
            child{ node [arn_x] {}}
            child{ node [arn_r] {9}
                child{ node [arn_x] {}}
                child{ node [arn_x] {}}
            }
        }
;

\end{tikzpicture}
\caption {\Arn con nodos vac\'ios.}
\label{arbolRB_1}
\end{figure}

En la figura \ref{arbolRB_1} podemos ver un ejemplo de un \arn que respeta la definici\'on que
acabamos de presentar. Los cuadros de las hojas representan los nodos vacíos y podemos ver que
estos son negros.

Nos interesa estudiar este tipo no trivial de arboles binarios de búsqueda para poder demostrar la
correcci\'on y completud de estos usando el asistente de pruebas {\coq} y así poder mostrar las
ventajas, y problemáticas de este proceso, desde la escritura/traducción de la estructura del
lenguaje Haskell a Coq, hasta las demostraciones que se realizaran con el asistente.

\section{Traducción de Haskell a {\coq} (hs-to-coq)}
Como mencionamos en la motivación de este trabajo, existen programas que nos ayudan a verificar
formalmente otros programas, en particular en este trabajo se decidió usar {\coq}, sin embargo el
usar esta herramienta trae sus propias problemáticas.

El primer problema que se encuentra al empezar a usar la herramienta es si se va a escribir el
programa directamente a {\coq} o si se va a traducir un programa ya existente al lenguaje del
asistente de pruebas, como lo que se busca es poder verificar un programa ya existente, es decir,
los {\arns}, queremos poder traducir código de Haskell a {\coq}.

En el articulo `Total Haskell is Reasonable Coq' \cite{thrc} se describen las principales ventajas
y desventajas de traducir de Haskell a Coq, los cuales describiré a continuaci\'on:

\subsection{Ventajas}
\begin{itemize}
    \item Haskell es un excelente lenguaje para escribir programas funcionales puros.
    \item La gran comunidad de programadores que usa y mantienen el lenguaje, al igual que el
    compilador GHC de Haskell, el cual tiene usos incluso industriales.
    \item El ambiente de Coq para desarrollar demostraciones formales.
    \item Coq permite razonar acerca de programas funcionales totales.
\end{itemize}

\subsection{Desventajas}
\begin{itemize}
    \item Los programadores de Haskell razonan acerca de su código informalmente, si se llegan a
    realizar pruebas de este, generalmente esta hecho a mano `en papel', lo cual es tedioso y
    susceptible a errores.
    \item Coq no tiene la extensa biblioteca de funciones ni la misma cantidad de programadores
    que lo usen y mantengan como lo tiene Haskell.
    \item El hecho de que los programadores de Haskell solo razonen acerca de su código
    informalmente puede que resulte en que se generen funciones parciales, es decir que no se
    cubran todas la combinaciones de parámetros posibles para una funci\'on.
    \item La traducci\'on de Haskell a Coq solo es posible si se tienen funciones totales.
\end{itemize}

Este articulo propone el uso de una herramienta llamada `hs-to-coq', la cual sigue en desarrollo y
es usada para traducir código de Haskell a Coq. En este trabajo se opto por usar esta herramienta
para la traducci\'on, esto porque la traducci\'on manual resulta ser muy tediosa y susceptible a
errores, esto desviaría el enfoque de este trabajo que es la verificaci\'on de la estructura, no la
traducci\'on de esta.

\section {Sobre este trabajo}
El contenido, demostraciones de este trabajo se encuentran almacenados en
\url{https://github.com/DavidFHCh/Tesis-FTW} Aqu\'i se presentan definiciones, lemas y clases sin
incluir las demostraciones en {\coq}, es decir, los scripts de prueba. En su lugar se describen de
forma informal las demostraciones para poder entender en alto nivel la estructura de la
verificaci\'on formal realizada.

La herramienta `hs-to-coq' fue usada para obtener las traducciones de las bibliotecas de Haskell
usadas para poder verificar la implementación de {\arns} que se uso en el trabajo.

En los siguientes capítulos se describe el procedimiento usado para la verificaci\'on de la
estructura de datos, al igual que las implementaciones y las pruebas realizadas para poder obtener
el resultado que se buscaba.

\chapter{Implementación de arboles roji-negros en {\coq}}

\section{Traducción de implementaciones}
Se tuvieron un par de aproximaciones para la implementación de {\arns}: la primera fue obtener las
implementaciones de estos \cite{tesisG} en Haskell, estas fueron utilizadas como entrada para la
utilidad \textit{hs-to-coq}, es decir, una traducci\'on directa. La segunda aproximación y la que
se uso para este trabajo, fue obtener de \cite{MSetRBT} la implementaci\'on de los {\arns} que se
usan en Coq, los cuales son una versi\'on de los {\arns} de Okasaki; en este caso se usaron
las bibliotecas traducidas de Haskell a Coq, las cuales contienen los tipos y comparaciones del
primer lenguaje. Esta traducción se obtuvo con la ayuda del traductor \textit{hs-to-coq} y estas
sustituyeron a los tipos y operaciones de Coq. A continuación profundizaremos de estos 
dos acercamientos.

\subsection{Traducción directa de implementaciones de Haskell a Coq}
De un trabajo anterior \cite{tesisG} se obtuvieron diversas implementaciones de {\arns}; estas
variaban en su mayor parte en las operaciones de borrado, es por ello que dicha operación es
significativamente mas compleja que su contraparte, i.e. la operación de insersi\'on. Estas 
variantes son: la implementación de Okasaki\footnote{siendo esta la m\'as simple},
los constructores inteligentes \footnote{implementaci\'on anterior con optimizaciones} y los 
tipos anidados \footnote{una implementaci\'on totalmente diferente a las anteriores y mas elegante}.

Por la compleja naturaleza de estas implementaciones\footnote{incluso Okasaki} la traducción
manual del código de Haskell resulto ser muy problemática, esto porque las implementaciones en
Haskell se aprovechan del hecho de que en este lenguaje se pueden declarar funciones parciales, lo
cual representa un reto al momento de intentar traducir a Coq, ya que este lenguaje únicamente acepta
funciones totales. Se buscaron soluciones para totalizar estas funciones, sin embargo, estas solo
traerían problemas al intentar realizar las demostraciones, ya que al totalizar se incluirian casos
inalcanzables en la ejecuci\'on, pero tendrian que ser demostrados como tales.

 A pasear de ello, se realizo trabajo para intentar totalizar las funciones de Haskell y asi poder usar
 la utilidad \textbf{hs-to-coq} y de esta manera facilitar la traducci\'on, pero por las mismas
 razones antes descritas \footnote{las funciones no eran totales o estas eran demasiado complejas
 que no se sabia que casos hacian falta.}, la herramienta caía en alguna de estas dos situaciones:

\begin{itemize}
    \item El tiempo de ejecuci\'on de la herramienta era muy alto y eventualmente los recursos de
    la maquina virtual, donde esta herramienta se ejecuto, se quedaba sin recursos\footnote{en
    especial memoria}. Esto probablemente se deba a la falta de totalidad en alguna función.
    \item La herramienta generaba c\'odigo en Coq pero con elementos de Haskell cuyas bibliotecas
    todavía no habían sido traducidas del todo. Esto porque las implementaciones en Haskell podian
    llegar a ser muy complejas y utilizar modulos de GHC, a los cuales todav\'ia no se les hab\'ia
    traducido con la herramienta.
\end{itemize}{}

Por estas razones se busco otro acercamiento para poder verificar esta estructura, entonces,
sabemos que el equipo de desarrollo de la herramienta hs-to-coq ha traducido exitosamente una
fracci\'on de las bibliotecas de Haskell a Coq, por esta raz\'on, se opto por el uso de la
implementación de {\arns} de las bibliotecas de {\coq}, \cite{MSetRBT}, pero usando los tipos y
operaciones obtenidos de las traducciones con la herramienta.

\section{Inserción de elementos en un {\arn}}

La inserci\'on de elementos a un {\arn} es la operaci\'on mas sencilla de las dos que se
verificar\'an en este trabajo. La idea principal detrás de este algoritmo es que \'unicamente se agreguen
hojas al \'arbol binario y se efectúen ``giros''\footnote{funciones de balanceo.} para mantener los
invariantes de la estructura (ver figura \ref{arbolRB_2} y \ref{arbolRB_3}).
\begin{figure}
\centering
\captionsetup{justification=centering}
 \begin{tikzpicture}[-,level/.style={sibling distance = 4cm/#1,level distance = 1cm}]
%\draw[style=help lines] (-5,-7) grid (5,0);
\node [arn_n] at (-2,0) {6}
        child{ node [arn_r] {2}
            child{ node [arn_n] {1}
                child{ node [arn_x] {}} %for a named pointer
                child{ node [arn_x] {}}
            }
            child{ node [arn_n] {4}
                child{ node [arn_r] {3}
                    child{ node [arn_x] {}}
                    child{ node [arn_x] {}}
                }
                child{ node [arn_r] {5}
                    child{ node [arn_x] {}}
                    child{ node [arn_x] {}}
                }
            }
        }
        child{ node [arn_n] {8}
            child{ node [arn_x] {}}
            child{ node [arn_r] {9}
                child{ node [arn_x] {}}
                child{ node [arn_x] {}}
            }
        }
;

{
}
;
\end{tikzpicture}
\caption{{\Arn} antes de insertar nodo 7.}
\label{arbolRB_2}
\end{figure}

\begin{figure}
\centering
\captionsetup{justification=centering}
 \begin{tikzpicture}[-,level/.style={sibling distance = 4cm/#1,level distance = 1cm}]
%\draw[style=help lines] (-5,-7) grid (5,0);
\node [arn_n] at (-2,0) {6}
        child{ node [arn_r] {2}
            child{ node [arn_n] {1}
                child{ node [arn_x] {}} %for a named pointer
                child{ node [arn_x] {}}
            }
            child{ node [arn_n] {4}
                child{ node [arn_r] {3}
                    child{ node [arn_x] {}}
                    child{ node [arn_x] {}}
                }
                child{ node [arn_r] {5}
                    child{ node [arn_x] {}}
                    child{ node [arn_x] {}}
                }
            }
        }
        child{ node [arn_n] {8}
            child{ node [arn_r] {7}
                child{ node [arn_x] {}}
                child{ node [arn_x] {}}
            }
            child{ node [arn_r] {9}
                child{ node [arn_x] {}}
                child{ node [arn_x] {}}
            }
        }
;

{
}
;
\end{tikzpicture}
\caption{{\Arn} después de insertar nodo 7.}
\label{arbolRB_3}
\end{figure}
\subsection{Operaciones de Balanceo}
Los giros antes mencionados están definidos en las operaciones de balanceo, se tienen dos, una para
los subárboles izquierdos y otra para los derechos. Estas funciones (ver figura \ref{func_balanceo})
se encargan de solucionar los casos en los que inmediatamente después de agregar una hoja alguno de 
los invariantes sean violados, por ejemplo, dos nodos rojos que resultan contiguos en algún lugar 
de la estructura del \'arbol.

El balanceo elimina el doble nodo rojo al crear \'unicamente un nodo rojo con dos hijos negros, de igual
manera esto nos asegura que el árbol crece de forma controlada en n\'umero de nodos negros
\footnote{este n\'umero de nodos negros se conoce como altura negra}, esto se debe a que en ningún
momento se están agregando dos nodos negros contiguos\footnote{Nodos padre e hijo negros después de
balancear.}; cabe mencionar que esta es la única operación en donde se agregan nodos negros, con la
excepción de $makeBlack$, la cual describiremos m\'as adelante.

\begin{figure}
\centering
\captionsetup{justification=centering}
\begin{minted}{coq}
Definition lbal {a} `{GHC.Base.Ord a} (l:RB a) (k:a) (r:RB a) :=
 match l with
 | T R (T R a x b) y c => T R (T B a x b) y (T B c k r)
 | T R a x (T R b y c) => T R (T B a x b) y (T B c k r)
 | _ => T B l k r
 end.

 Definition rbal {a} `{GHC.Base.Ord a} (l:RB a) (k:a) (r:RB a) :=
 match r with
 | T R (T R b y c) z d => T R (T B l k b) y (T B c z d)
 | T R b y (T R c z d) => T R (T B l k b) y (T B c z d)
 | _ => T B l k r
 end.
\end{minted}
\caption{Funciones de Balanceo.}
\label{func_balanceo}
\end{figure}

En puntos posteriores se explicar\'an los casos de uso de esta función, se desarrollar\'a el porqu\'e los
\'unicos casos a los que se les da un trato especial es a los de nodos rojos contiguos y en el
resto s\'olo se regresa un \'arbol con ra\'iz negra sin hacer mayor acomodo.

\subsection {Funci\'on de inserci\'on}
Esta funci\'on es donde se presenta por primera vez el uso de las bibliotecas traducidas de
Haskell, podemos apreciar como los tipos \footnote{El tipo que se usa en los \arns es representado
con la letra \textbf{\textit{a}}.} de los elementos que se est\'an agregando al \'arbol son tipos ordenados de la
biblioteca $Base$ del compilador de GHC y por esa misma raz\'on estamos usando las comparaciones de
esa biblioteca.
\begin{figure}
\centering
\captionsetup{justification=centering}
\begin{minted}{coq}
Fixpoint ins {a} `{GHC.Base.Ord a} (x:a) (s:RB a) :=
 match s with
 | E => T R E x E
 | T c l y r =>
    if x GHC.Base.< y : bool then
      match c with
       | R => T R (ins x l) y r
       | B => lbal (ins x l) y r
      end
    else
    if x GHC.Base.> y : bool then
      match c with
       | R => T R l y (ins x r)
       | B => rbal l y (ins x r)
      end
    else s
 end.
\end{minted}
\caption{Funci\'on ins.}
\label{func_ins}
\end{figure}

Analizando m\'as detenidamente la funci\'on (figura \ref{func_ins}) se puede observar que las
operaciones de balanceo solo se efectúan cuando el nodo por el que se esta pasando es negro, esto
sucede por la raz\'on de que los nodos de este color son los que se toman en cuenta para decidir si
un \'arbol cumple con el balanceo adecuado. Al aplicar el balanceo en estos nodos, podemos garantizar
que no quedar\'an con nodos negros extras alguno de los hijos de este nodo, es decir, que ninguno de
los caminos de la ra\'iz a las hojas tenga mas nodos negros que
los demas. Esto se puede apreciar si nos regresamos a las definiciones de las operaciones de
balanceo, tomemos $rbal$ (figura \ref{func_balanceo})\footnote{Con $lbal$ la idea es an\'aloga}, 
tenemos dos casos:

\begin{itemize}
    \item Sean \textbf{\textit{x}}, \textbf{\textit{y}} y \textbf{\textit{z}} nodos del \'arbol y sea \textbf{\textit{t}} un subárbol, \textbf{\textit{x}} es el nodo al que se le
    aplica la operaci\'on de balanceo y este es de color negro, \textbf{\textit{t}} es el subárbol izquierdo, \textbf{\textit{y}}
    es el nodo derecho de \textbf{\textit{x}} y \textbf{\textit{z}} es hijo de \textbf{\textit{y}} \footnote{Es irrelevante si es derecho o
    izquierdo, el resultado es el mismo.}. Suponiendo que \textbf{\textit{y}} y \textbf{\textit{z}} son rojos\footnote{se viola una
    invariante, dos nodos rojos contiguos}, se cae en cualquiera de los dos casos de $rbal$ que no
    sean el caso general. En este momento es donde se efectúa el \textit{balanceo} del árbol y
    resulta lo siguiente: \textbf{\textit{x}} se vuelve el hijo izquierdo de \textbf{\textit{y}} y \textbf{\textit{z}} se pinta de negro
    \footnote{El hijo se vuelve padre y el padre se vuelve hijo.}, todas las dem\'as estructuras del
    \'arbol permanecen igual.

    En el momento en que \textbf{\textit{x}} se convierte en hijo izquierdo de \textbf{\textit{y}} el \'arbol se desbalancea, es
    por esto que se pinta de negro a \textbf{\textit{z}}, así los dos nodos negros son hijos de \textbf{\textit{y}} y la invariante
    se conserva.
    \item En cualquier otro caso el \'arbol no sufre modificaci\'on alguna.
\end{itemize}

Este balanceo es necesario en esta funci\'on, ya que todos los elementos nuevos que se agregan al \'arbol 
son hojas rojas, esto puede traer consigo violaciones a los invariantes, en especial al de que existan dos nodos rojos 
contiguos y esta opearci\'on ayuda a mitigar este problema.

A pesar de que las operaciones de balanceo cuidan la mayoria invariantes en el cuerpo del \'arbol,
la función $ins$ no necesariamente cumple con uno de los invariantes, espec\'ificamente
en el que la raíz del árbol es negra, es por ello que se introducen las definiciones de 
la figura \ref{raiz_negra_func}.

\begin{figure}
\centering
\captionsetup{justification=centering}
\begin{minted}{coq}
Definition makeBlack {a} `{GHC.Base.Ord a} (t:RB a) :=
 match t with
 | E => E
 | T _ a x b => T B a x b
 end.

Definition insert {a} `{GHC.Base.Ord a} (x:a) (s:RB a) :=
                                          makeBlack (ins x s).
\end{minted}
\caption{Definiciones para pintar ra\'iz de negro.}
\label{raiz_negra_func}
\end{figure}

La definici\'on $makeBlack$ únicamente colorea un nodo de color negro y la definición
$insert$ es una envoltura de $ins$, con la cual nos aseguramos de que la ra\'iz de los \'arboles
siempre sea de color negro, esto se logra con ayuda de $makeBlack$.

Estas funciones y definiciones son suficientes para poder construir {\arns} que respeten las
invariantes que planteamos en la definici\'on 1.2.1.

\section{Eliminación de elementos en un {\arn}}

Como se menciono en la secci\'on anterior, la operaci\'on de eliminaci\'on es significativamente m\'as
compleja que su contra parte, esto se debe al hecho de que pueden ser eliminados cualesquiera nodos
 en un {\arn}, mientras que en la inserci\'on s\'olo se agregan hojas de color rojo, es decir,
la altura \'unicamente se modifica en la inserción cuando se aplica el balanceo.

La acci\'on de eliminar nodos de cualquier parte de un {\arn} presenta una problemática muy grande para
el balanceo del mismo, esto se suscita al eliminar un nodo del \'arbol, los dos subárboles de este
tienen que ser concatenados de alguna forma y los invariantes de los mismos tienen que ser
respetados.

\subsection{Funci\'on de eliminaci\'on}

Para poder comprender la l\'ogica de las funciones que conforman a la operaci\'on de eliminaci\'on
es necesario comenzar por la funci'on que retira el nodo del \'arbol (ver la figura \ref{func_del}).
La idea central de esta operaci\'on es bastante simple: como los {\arns} son \'arboles de búsqueda,
lo primero que hacemos es buscar el nodo a eliminar, si se encuentra se elimina y se concatenan los
subárboles restantes de esta operaci\'on (ver figuras \ref{arbolRB_4}, \ref{arbolRB_5} y
\ref{arbolRB_6}). A continuaci\'on se describen m\'as a fondo los casos de la misma:

\begin{itemize}
    \item Si se recibe un \'arbol vacío como argumento de la funci\'on, se regresa este mismo; pues
    eliminar un elemento del \'arbol vacío termina siendo vacio. También este caso sirve para
    cuando un elemento no es encontrado en el \'arbol, es el caso base de la recursi\'on de
    búsqueda del nodo a eliminar.
    \item En otro caso, se realiza recursivamente la búsqueda del elemento a eliminar. Si el nodo
    actual no contiene el elemento que buscamos, se compara si es menor o mayor para seguir
    buscando en el \'arbol izquierdo o derecho respectivamente. Si el siguiente nodo es negro y se
    encuentra en el sub\'arbol izquierdo, se realiza la operaci\'on $lbalS$ \footnote{Funci\'on de
    balanceo extendida para subarboles izquierdos.}, por otro lado, si el nodo se encuentra en el
    sub\'arbol derecho se aplica $rbalS$ \footnote{Funci\'on de balanceo extendida para subarboles
    derechos.}. Si el elemento en el que estamos no es ni mayor ni menor al que buscamos, en ese
    caso eliminamos el elemento y concatenamos los subárboles restantes usando la función $append$
    \footnote{Funci\'on donde se juntan lo arboles restantes de esta operaci\'on}.
\end{itemize}

\begin{figure}
\centering
\captionsetup{justification=centering}
\begin{minted}{coq}
Fixpoint del {a} `{GHC.Base.Ord a} (x:a) (t:RB a) :=
 match t with
 | E => E
 | T _ a y b =>
    if x GHC.Base.< y : bool then
      match a with
       | T B _ _ _ => lbalS (del x a) y b
       | _ => T R (del x a) y b
      end
    else
    if x GHC.Base.> y : bool then
      match b with
       | T B _ _ _ => rbalS a y (del x b)
       | _ => T R a y (del x b)
      end
    else append a b
 end.

Definition remove x t := makeBlack (del x t).
\end{minted}
\caption{Función de eliminación}
\label{func_del}
\end{figure}

\begin{figure}
\centering
\captionsetup{justification=centering}
 \begin{tikzpicture}[-,level/.style={sibling distance = 4cm/#1,level distance = 1cm}]
%\draw[style=help lines] (-5,-7) grid (5,0);
\node [arn_n] at (-2,0) {6}
        child{ node [arn_r] {2}
            child{ node [arn_n] {1}
                child{ node [arn_x] {}} %for a named pointer
                child{ node [arn_x] {}}
            }
            child{ node [arn_n] {4}
                child{ node [arn_r] {3}
                    child{ node [arn_x] {}}
                    child{ node [arn_x] {}}
                }
                child{ node [arn_r] {5}
                    child{ node [arn_x] {}}
                    child{ node [arn_x] {}}
                }
            }
        }
        child{ node [arn_n] {8}
            child{ node [arn_x] {}}
            child{ node [arn_r] {9}
                child{ node [arn_x] {}}
                child{ node [arn_x] {}}
            }
        }
;

{
}
;
\end{tikzpicture}
\caption{{\Arn antes de eliminar nodo 6.}}
\label{arbolRB_4}
\end{figure}

\begin{figure}
\centering
\captionsetup{justification=centering}
 \begin{tikzpicture}[-,level/.style={sibling distance = 4cm/#1,level distance = 1cm}]
%\draw[style=help lines] (-5,-7) grid (5,0);
%\node [arn_n] at (-2,0) {6}

        \node [arn_r] {2}
            child{ node [arn_n] {1}
                child{ node [arn_x] {}} %for a named pointer
                child{ node [arn_x] {}}
            }
            child{ node [arn_n] {4}
                child{ node [arn_r] {3}
                    child{ node [arn_x] {}}
                    child{ node [arn_x] {}}
                }
                child{ node [arn_r] {5}
                    child{ node [arn_x] {}}
                    child{ node [arn_x] {}}
                }
            }
;
\end{tikzpicture}
\begin{tikzpicture}[-,level/.style={sibling distance = 4cm/#1,level distance = 1cm}]

            \node [arn_n] {8}
            child{ node [arn_x] {}}
            child{ node [arn_r] {9}
                child{ node [arn_x] {}}
                child{ node [arn_x] {}}
            }

;
\end{tikzpicture}

\caption{{\Arn} roto, después de eliminar nodo 6.}
\label{arbolRB_5}
\end{figure}

\begin{figure}
\centering
\captionsetup{justification=centering}
 \begin{tikzpicture}[-,level/.style={sibling distance = 4cm/#1,level distance = 1cm}]
%\draw[style=help lines] (-5,-7) grid (5,0);
\node [arn_n] at (-2,0) {5}
        child{ node [arn_r] {2}
            child{ node [arn_n] {1}
                child{ node [arn_x] {}} %for a named pointer
                child{ node [arn_x] {}}
            }
            child{ node [arn_n] {4}
                child{ node [arn_r] {3}
                    child{ node [arn_x] {}}
                    child{ node [arn_x] {}}
                }
                child{ node [arn_x] {}}
            }
        }
        child{ node [arn_n] {8}
            child{ node [arn_x] {}}
            child{ node [arn_r] {9}
                child{ node [arn_x] {}}
                child{ node [arn_x] {}}
            }
        }
;

{
}
;
\end{tikzpicture}
\caption{{\Arn} después de aplicar función append.}
\label{arbolRB_6}
\end{figure}

Podemos ver que las funciones de balanceo $lbalS$ y $rbalS$ se aplican cuando el nodo en el que
estamos parados, llamémoslo \textbf{\textit{n}}, es negro; esto evita que después de eliminar un nodo y aplicar la
funci\'on $append$ se acabe con dos nodos rojos seguidos, es decir, que el hijo y alguno de los
nietos del nodo \textbf{\textit{n}} sean rojos.

\subsection{Funci\'on de concatenaci\'on}

La funci\'on de concatenación (figura \ref{func_app}) es usada cuando se encuentra el elemento que
se busca eliminar de un {\arn}, esto es porque la acci\'on de retirar un nodo del \'arbol resulta
en dos \'arboles que tienen que ser concatenados, los cuales deben de respetar los invariantes de
los {\arns}. Esta funci\'on recibe como parámetros los dos \'arboles\footnote{Estos arboles pueden
no cumplir las invariantes de ls {\arns}.} que estamos buscando juntar. Esta operación se describe
con mayor detalle en seguida.

\begin{figure}
\centering
\captionsetup{justification=centering}
\begin{minted}{coq}
Fixpoint append {a} `{GHC.Base.Ord a} (l:RB a) : RB a -> RB a :=
 match l with
 | E => fun r => r
 | T lc ll lx lr =>
   fix append_l (r:RB a) : RB a :=
   match r with
   | E => l
   | T rc rl rx rr =>
     match lc, rc with
     | R, R =>
       let lrl := append lr rl in
       match lrl with
       | T R lr' x rl' => T R (T R ll lx lr') x (T R rl' rx rr)
       | _ => T R ll lx (T R lrl rx rr)
       end
     | B, B =>
       let lrl := append lr rl in
       match lrl with
       | T R lr' x rl' => T R (T B ll lx lr') x (T B rl' rx rr)
       | _ => lbalS ll lx (T B lrl rx rr)
       end
     | B, R => T R (append_l rl) rx rr
     | R, B => T R ll lx (append lr r)
     end
   end
 end.
\end{minted}
\caption{Funci\'on de concatenaci\'on, append}
\label{func_app}
\end{figure}


Sean \textbf{\textit{a}} y \textbf{\textit{b}} los dos subárboles a los que se les aplicar\'a la funci\'on append, es decir,
\textit{append a b}, tenemos los siguientes casos:

\begin{itemize}
    \item Si \textbf{\textit{a}} es el \'arbol vacío, entonces se regresa \textbf{\textit{b}}.
    \item Si \textbf{\textit{b}} es el \'arbol vacío, entonces regresamos \textbf{\textit{a}}.
    \item Si \textbf{\textit{a}} y \textbf{\textit{b}} son \'arboles con raíces rojas, entonces se aplica $append$ al subárbol
    derecho de \textbf{\textit{a}}, sea este \textbf{\textit{ar}}, junto con el subárbol izquierdo de \textbf{\textit{b}}, sea \textbf{\textit{bl}}, es decir,
    \textit{append ar bl}. Tenemos subcasos:
    \begin{itemize}
      \item Si el resultado de esta operación es un árbol con raíz roja, sea \textbf{\textit{arbl}}, los \'arboles
      \textbf{\textit{a}} y \textbf{\textit{b}} se pintan de rojo y se concatenan con la raíz de \textbf{\textit{arbl}}, igual de color rojo; \textbf{\textit{ar}}
      se reemplaza por el subárbol izquierdo de \textbf{\textit{arbl}} y \textbf{\textit{bl}} se reemplaza por el subárbol derecho
      de \textbf{\textit{arbl}}.
      \item En otro caso, si el \'arbol resultante de \textit{append ar bl} no es rojo, tomamos \textbf{\textit{a}}
      y \textbf{\textit{b}}, los pintamos de rojo, el subárbol derecho de \textbf{\textit{a}} se reemplaza por \textbf{\textit{b}} y el subárbol
      izquierdo de \textbf{\textit{b}} se reemplaza por el resultado de \textit{append ar bl}.
    \end{itemize}
    \item Si \textbf{\textit{a}} y \textbf{\textit{b}} son arboles con raíces negras, entonces se aplica $append$ al subárbol
    derecho de \textbf{\textit{a}}, sea \textbf{\textit{ar}}, con el subárbol izquierdo de \textbf{\textit{b}}, sea \textbf{\textit{bl}}, es decir, \textit{append
    ar bl}. Tenemos casos:
    \begin{itemize}
      \item Si el resultado de esta operación es un árbol con raíz roja, sea \textbf{\textit{arbl}}, los \'arboles
      \textbf{\textit{a}} y \textbf{\textit{b}} se pintan de negro y se concatenan con la raíz de \textbf{\textit{arbl}}, esta de color rojo; \textbf{\textit{ar}}
      se reemplaza por el subárbol izquierdo de \textbf{\textit{arbl}} y \textbf{\textit{bl}} se reemplaza por el subárbol derecho
      de \textbf{\textit{arbl}}.
      \item En otro caso, si el \'arbol resultante de \textit{append ar bl} no es rojo, tomamos \textbf{\textit{a}}
      y \textbf{\textit{b}}, el subárbol derecho de \textbf{\textit{a}} se reemplaza por \textbf{\textit{b}} y el subárbol izquierdo de \textbf{\textit{b}} se
      reemplaza por el resultado de \textit{append ar bl} y a este resultado le aplicamos una
      función de balanceo, $lbalS$.
    \end{itemize}
    \item Si \textbf{\textit{a}} es un \'arbol de color negro y \textbf{\textit{b}} de color rojo, entonces se toma \textbf{\textit{b}}, se pinta
    de rojo pero en lugar de su subárbol izquierdo, sea \textbf{\textit{bl}}, se aplica una llamada recursiva a
    \textbf{\textit{bl}} con la funci\'on embebida en $append$, llamada $append\_l$, es decir: \textit{append\_l
    bl}, esta llamada tambi\'en carga al \'arbol \textbf{\textit{a}} gracias al currying\cite{Currying}.
    \item Si \textbf{\textit{a}} es un \'arbol de color rojo y \textbf{\textit{b}} de color negro, entonces se toma \textbf{\textit{a}}, se pinta
    de rojo pero en lugar de su subárbol derecho, sea \textbf{\textit{ar}}, se hace una llamada recursiva con
    $append(ar,b)$.
\end{itemize}

Debemos mencionar que el \'arbol resultante de aplicar esta funci\'on no necesariamente cumple los
invariantes de un {\arn}, estas invariantes se logran conservar ya que en la funci\'on $del$ se
realizan llamadas a las funciones extendidas de balanceo, las cuales desarrollaremos en la
siguiente sección.

\subsection{Extensi\'on de funciones de balanceo}

En la secci\'on 2.2 de este trabajo se trato la inserci\'on de elementos a un {\arn}, en donde se
describen un par de funciones llamadas `funciones de balanceo', tratadas en las subsecci\'on
2.2.1, estas funciones a su vez toman los nombres $rbal$ y $lbal$ (figura \ref{func_balanceo}).
Estas funcionesresultan insuficientes para balancear un \'arbol al momento de eliminar un nodo y
concatenar los dos \'arboles restantes con la función $append$, es por eso que se implementan las
extensiones de estas funciones, llamadas $lbalS$ y $rbalS$ (figuras \ref{lbalS} y \ref{rbalS}
respectivamente) las cuales a su vez llaman a las funciones $rbal'$\footnote{La funci\'on $rbal'$
es una variación de la función $rbal$, solo cambia el orden de la caza de patrones.} (figura
\ref{rbal_2}) y $lbal$. Estas extensiones agregan mas casos de manejo de subárboles negros, esto
porque existen casos en los que se puede llegar a eliminar un nodo negro intermedio y se tiene que
asegurar que las invariantes no se violen después de concatenar los subárboles resultantes de
aplicar la función $del$.

\begin{figure}
\centering
\captionsetup{justification=centering}
\begin{minted}{coq}

Definition lbalS {a} `{GHC.Base.Ord a} (l:RB a) (k:a) (r:RB a) :=
 match l with
 | T R a x b => T R (T B a x b) k r
 | _ =>
   match r with
   | T B a y b => rbal' l k (T R a y b)
   | T R (T B a y b) z c => T R (T B l k a) y (rbal' b z (makeRed c))
   | _ => T R l k r
   end
 end.

\end{minted}
\caption{Funci\'on de balanceo de lado izquierdo extendida.}
\label{lbalS}
\end{figure}

Las funciones $rbalS$ y $lbalS$ son usadas en la funci\'on $del$ (figura \ref{func_del}) cuando el
caso en el que se cae es un nodo de color negro y al aplicar la funci\'on en estos nodos podemos
asegurar que los dos subárboles de este nodo no se van a desequilibrar, es decir, que un subárbol
tenga mas nodos negros que el otro. Después de aplicar la función de balanceo se aplica otra llamada
recursiva a $del$.


\begin{figure}
\centering
\captionsetup{justification=centering}
\begin{minted}{coq}

Definition rbalS {a} `{GHC.Base.Ord a} (l:RB a) (k:a) (r:RB a) :=
 match r with
 | T R b y c => T R l k (T B b y c)
 | _ =>
   match l with
   | T B a x b => lbal (T R a x b) k r
   | T R a x (T B b y c) => T R (lbal (makeRed a) x b) y (T B c k r)
   | _ => T R l k r
   end
 end.

\end{minted}
\caption{Funciones de balanceo de lado derecho extendida.}
\label{rbalS}
\end{figure}

Existe otra función donde se utiliza una de estas operaciones de balanceo, específicamente $lbalS$,
esta funci\'on es $append$, en el caso de que los arboles que se le esten pasando como parámetros
sean negros, esto es por la misma razón por la cual se aplican las funciones de balanceo en $del$
sobre los nodos de color negro: para que sus subárboles no se desbalanceen.

\begin{figure}
\centering
\captionsetup{justification=centering}
\begin{minted}{coq}
Definition rbal' {a} `{GHC.Base.Ord a} (l:RB a) (k:a) (r:RB a) :=
 match r with
 | T R b y (T R c z d) => T R (T B l k b) y (T B c z d)
 | T R (T R b y c) z d => T R (T B l k b) y (T B c z d)
 | _ => T B l k r
 end.

\end{minted}
\caption{Funci\'on de balanceo de lado derecho alternativa.}
\label{rbal_2}
\end{figure}

Estas definiciones y funciones son suficientes para poder eliminar nodos de un {\arn} y que el
resultado no viole los invariantes de estos. Al menos eso es lo que nos gustaria poder afirmar en
esta etapa del trabajo, sin embargo, esta sentencia tiene que ser demostrada, es decir, tenemos que
probar que nuestros {\arns} cumplen con la definición.
\chapter{Implementación de arboles roji-negros en {\coq}}

\section{Traducción de implementaciones}
Se tuvieron un par de aproximaciones para la implementación de {\arns}: la primera fue obtener las
implementaciones de estos \cite{tesisG} en Haskell, estas fueron utilizadas como entrada para la
utilidad \textit{hs-to-coq}, es decir, una traducci\'on directa. La segunda aproximación y la que
se uso para este trabajo, fue obtener de \cite{MSetRBT} la implementaci\'on de los {\arns} que se
usan en Coq, los cuales son una versi\'on de los {\arns} de Okasaki; en este caso se usaron
las bibliotecas traducidas de Haskell a Coq, las cuales contienen los tipos y comparaciones del
primer lenguaje. Esta traducción se obtuvo con la ayuda del traductor \textit{hs-to-coq} y estas
sustituyeron a los tipos y operaciones de Coq. A continuación profundizaremos de estos 
dos acercamientos.

\subsection{Traducción directa de implementaciones de Haskell a Coq}
De un trabajo anterior \cite{tesisG} se obtuvieron diversas implementaciones de {\arns}; estas
variaban en su mayor parte en las operaciones de borrado, es por ello que dicha operación es
significativamente mas compleja que su contraparte, i.e. la operación de insersi\'on. Estas 
variantes son: la implementación de Okasaki\footnote{siendo esta la m\'as simple},
los constructores inteligentes \footnote{implementaci\'on anterior con optimizaciones} y los 
tipos anidados \footnote{una implementaci\'on totalmente diferente a las anteriores y mas elegante}.

Por la compleja naturaleza de estas implementaciones\footnote{incluso Okasaki} la traducción
manual del código de Haskell resulto ser muy problemática, esto porque las implementaciones en
Haskell se aprovechan del hecho de que en este lenguaje se pueden declarar funciones parciales, lo
cual representa un reto al momento de intentar traducir a Coq, ya que este lenguaje únicamente acepta
funciones totales. Se buscaron soluciones para totalizar estas funciones, sin embargo, estas solo
traerían problemas al intentar realizar las demostraciones, ya que al totalizar se incluirian casos
inalcanzables en la ejecuci\'on, pero tendrian que ser demostrados como tales.

 A pasear de ello, se realizo trabajo para intentar totalizar las funciones de Haskell y asi poder usar
 la utilidad \textbf{hs-to-coq} y de esta manera facilitar la traducci\'on, pero por las mismas
 razones antes descritas \footnote{las funciones no eran totales o estas eran demasiado complejas
 que no se sabia que casos hacian falta.}, la herramienta caía en alguna de estas dos situaciones:

\begin{itemize}
    \item El tiempo de ejecuci\'on de la herramienta era muy alto y eventualmente los recursos de
    la maquina virtual, donde esta herramienta se ejecuto, se quedaba sin recursos\footnote{en
    especial memoria}. Esto probablemente se deba a la falta de totalidad en alguna función.
    \item La herramienta generaba c\'odigo en Coq pero con elementos de Haskell cuyas bibliotecas
    todavía no habían sido traducidas del todo. Esto porque las implementaciones en Haskell podian
    llegar a ser muy complejas y utilizar modulos de GHC, a los cuales todav\'ia no se les hab\'ia
    traducido con la herramienta.
\end{itemize}{}

Por estas razones se busco otro acercamiento para poder verificar esta estructura, entonces,
sabemos que el equipo de desarrollo de la herramienta hs-to-coq ha traducido exitosamente una
fracci\'on de las bibliotecas de Haskell a Coq, por esta raz\'on, se opto por el uso de la
implementación de {\arns} de las bibliotecas de {\coq}, \cite{MSetRBT}, pero usando los tipos y
operaciones obtenidos de las traducciones con la herramienta.

\section{Inserción de elementos en un {\arn}}

La inserci\'on de elementos a un {\arn} es la operaci\'on mas sencilla de las dos que se
verificar\'an en este trabajo. La idea principal detrás de este algoritmo es que \'unicamente se agreguen
hojas al \'arbol binario y se efectúen ``giros''\footnote{funciones de balanceo.} para mantener los
invariantes de la estructura (ver figura \ref{arbolRB_2} y \ref{arbolRB_3}).
\begin{figure}
\centering
\captionsetup{justification=centering}
 \begin{tikzpicture}[-,level/.style={sibling distance = 4cm/#1,level distance = 1cm}]
%\draw[style=help lines] (-5,-7) grid (5,0);
\node [arn_n] at (-2,0) {6}
        child{ node [arn_r] {2}
            child{ node [arn_n] {1}
                child{ node [arn_x] {}} %for a named pointer
                child{ node [arn_x] {}}
            }
            child{ node [arn_n] {4}
                child{ node [arn_r] {3}
                    child{ node [arn_x] {}}
                    child{ node [arn_x] {}}
                }
                child{ node [arn_r] {5}
                    child{ node [arn_x] {}}
                    child{ node [arn_x] {}}
                }
            }
        }
        child{ node [arn_n] {8}
            child{ node [arn_x] {}}
            child{ node [arn_r] {9}
                child{ node [arn_x] {}}
                child{ node [arn_x] {}}
            }
        }
;

{
}
;
\end{tikzpicture}
\caption{{\Arn} antes de insertar nodo 7.}
\label{arbolRB_2}
\end{figure}

\begin{figure}
\centering
\captionsetup{justification=centering}
 \begin{tikzpicture}[-,level/.style={sibling distance = 4cm/#1,level distance = 1cm}]
%\draw[style=help lines] (-5,-7) grid (5,0);
\node [arn_n] at (-2,0) {6}
        child{ node [arn_r] {2}
            child{ node [arn_n] {1}
                child{ node [arn_x] {}} %for a named pointer
                child{ node [arn_x] {}}
            }
            child{ node [arn_n] {4}
                child{ node [arn_r] {3}
                    child{ node [arn_x] {}}
                    child{ node [arn_x] {}}
                }
                child{ node [arn_r] {5}
                    child{ node [arn_x] {}}
                    child{ node [arn_x] {}}
                }
            }
        }
        child{ node [arn_n] {8}
            child{ node [arn_r] {7}
                child{ node [arn_x] {}}
                child{ node [arn_x] {}}
            }
            child{ node [arn_r] {9}
                child{ node [arn_x] {}}
                child{ node [arn_x] {}}
            }
        }
;

{
}
;
\end{tikzpicture}
\caption{{\Arn} después de insertar nodo 7.}
\label{arbolRB_3}
\end{figure}
\subsection{Operaciones de Balanceo}
Los giros antes mencionados están definidos en las operaciones de balanceo, se tienen dos, una para
los subárboles izquierdos y otra para los derechos. Estas funciones (ver figura \ref{func_balanceo})
se encargan de solucionar los casos en los que inmediatamente después de agregar una hoja alguno de 
los invariantes sean violados, por ejemplo, dos nodos rojos que resultan contiguos en algún lugar 
de la estructura del \'arbol.

El balanceo elimina el doble nodo rojo al crear \'unicamente un nodo rojo con dos hijos negros, de igual
manera esto nos asegura que el árbol crece de forma controlada en n\'umero de nodos negros
\footnote{este n\'umero de nodos negros se conoce como altura negra}, esto se debe a que en ningún
momento se están agregando dos nodos negros contiguos\footnote{Nodos padre e hijo negros después de
balancear.}; cabe mencionar que esta es la única operación en donde se agregan nodos negros, con la
excepción de $makeBlack$, la cual describiremos m\'as adelante.

\begin{figure}
\centering
\captionsetup{justification=centering}
\begin{minted}{coq}
Definition lbal {a} `{GHC.Base.Ord a} (l:RB a) (k:a) (r:RB a) :=
 match l with
 | T R (T R a x b) y c => T R (T B a x b) y (T B c k r)
 | T R a x (T R b y c) => T R (T B a x b) y (T B c k r)
 | _ => T B l k r
 end.

 Definition rbal {a} `{GHC.Base.Ord a} (l:RB a) (k:a) (r:RB a) :=
 match r with
 | T R (T R b y c) z d => T R (T B l k b) y (T B c z d)
 | T R b y (T R c z d) => T R (T B l k b) y (T B c z d)
 | _ => T B l k r
 end.
\end{minted}
\caption{Funciones de Balanceo.}
\label{func_balanceo}
\end{figure}

En puntos posteriores se explicar\'an los casos de uso de esta función, se desarrollar\'a el porqu\'e los
\'unicos casos a los que se les da un trato especial es a los de nodos rojos contiguos y en el
resto s\'olo se regresa un \'arbol con ra\'iz negra sin hacer mayor acomodo.

\subsection {Funci\'on de inserci\'on}
Esta funci\'on es donde se presenta por primera vez el uso de las bibliotecas traducidas de
Haskell, podemos apreciar como los tipos \footnote{El tipo que se usa en los \arns es representado
con la letra \textbf{\textit{a}}.} de los elementos que se est\'an agregando al \'arbol son tipos ordenados de la
biblioteca $Base$ del compilador de GHC y por esa misma raz\'on estamos usando las comparaciones de
esa biblioteca.
\begin{figure}
\centering
\captionsetup{justification=centering}
\begin{minted}{coq}
Fixpoint ins {a} `{GHC.Base.Ord a} (x:a) (s:RB a) :=
 match s with
 | E => T R E x E
 | T c l y r =>
    if x GHC.Base.< y : bool then
      match c with
       | R => T R (ins x l) y r
       | B => lbal (ins x l) y r
      end
    else
    if x GHC.Base.> y : bool then
      match c with
       | R => T R l y (ins x r)
       | B => rbal l y (ins x r)
      end
    else s
 end.
\end{minted}
\caption{Funci\'on ins.}
\label{func_ins}
\end{figure}

Analizando m\'as detenidamente la funci\'on (figura \ref{func_ins}) se puede observar que las
operaciones de balanceo solo se efectúan cuando el nodo por el que se esta pasando es negro, esto
sucede por la raz\'on de que los nodos de este color son los que se toman en cuenta para decidir si
un \'arbol cumple con el balanceo adecuado. Al aplicar el balanceo en estos nodos, podemos garantizar
que no quedar\'an con nodos negros extras alguno de los hijos de este nodo, es decir, que ninguno de
los caminos de la ra\'iz a las hojas tenga mas nodos negros que
los demas. Esto se puede apreciar si nos regresamos a las definiciones de las operaciones de
balanceo, tomemos $rbal$ (figura \ref{func_balanceo})\footnote{Con $lbal$ la idea es an\'aloga}, 
tenemos dos casos:

\begin{itemize}
    \item Sean \textbf{\textit{x}}, \textbf{\textit{y}} y \textbf{\textit{z}} nodos del \'arbol y sea \textbf{\textit{t}} un subárbol, \textbf{\textit{x}} es el nodo al que se le
    aplica la operaci\'on de balanceo y este es de color negro, \textbf{\textit{t}} es el subárbol izquierdo, \textbf{\textit{y}}
    es el nodo derecho de \textbf{\textit{x}} y \textbf{\textit{z}} es hijo de \textbf{\textit{y}} \footnote{Es irrelevante si es derecho o
    izquierdo, el resultado es el mismo.}. Suponiendo que \textbf{\textit{y}} y \textbf{\textit{z}} son rojos\footnote{se viola una
    invariante, dos nodos rojos contiguos}, se cae en cualquiera de los dos casos de $rbal$ que no
    sean el caso general. En este momento es donde se efectúa el \textit{balanceo} del árbol y
    resulta lo siguiente: \textbf{\textit{x}} se vuelve el hijo izquierdo de \textbf{\textit{y}} y \textbf{\textit{z}} se pinta de negro
    \footnote{El hijo se vuelve padre y el padre se vuelve hijo.}, todas las dem\'as estructuras del
    \'arbol permanecen igual.

    En el momento en que \textbf{\textit{x}} se convierte en hijo izquierdo de \textbf{\textit{y}} el \'arbol se desbalancea, es
    por esto que se pinta de negro a \textbf{\textit{z}}, así los dos nodos negros son hijos de \textbf{\textit{y}} y la invariante
    se conserva.
    \item En cualquier otro caso el \'arbol no sufre modificaci\'on alguna.
\end{itemize}

Este balanceo es necesario en esta funci\'on, ya que todos los elementos nuevos que se agregan al \'arbol 
son hojas rojas, esto puede traer consigo violaciones a los invariantes, en especial al de que existan dos nodos rojos 
contiguos y esta opearci\'on ayuda a mitigar este problema.

A pesar de que las operaciones de balanceo cuidan la mayoria invariantes en el cuerpo del \'arbol,
la función $ins$ no necesariamente cumple con uno de los invariantes, espec\'ificamente
en el que la raíz del árbol es negra, es por ello que se introducen las definiciones de 
la figura \ref{raiz_negra_func}.

\begin{figure}
\centering
\captionsetup{justification=centering}
\begin{minted}{coq}
Definition makeBlack {a} `{GHC.Base.Ord a} (t:RB a) :=
 match t with
 | E => E
 | T _ a x b => T B a x b
 end.

Definition insert {a} `{GHC.Base.Ord a} (x:a) (s:RB a) :=
                                          makeBlack (ins x s).
\end{minted}
\caption{Definiciones para pintar ra\'iz de negro.}
\label{raiz_negra_func}
\end{figure}

La definici\'on $makeBlack$ únicamente colorea un nodo de color negro y la definición
$insert$ es una envoltura de $ins$, con la cual nos aseguramos de que la ra\'iz de los \'arboles
siempre sea de color negro, esto se logra con ayuda de $makeBlack$.

Estas funciones y definiciones son suficientes para poder construir {\arns} que respeten las
invariantes que planteamos en la definici\'on 1.2.1.

\section{Eliminación de elementos en un {\arn}}

Como se menciono en la secci\'on anterior, la operaci\'on de eliminaci\'on es significativamente m\'as
compleja que su contra parte, esto se debe al hecho de que pueden ser eliminados cualesquiera nodos
 en un {\arn}, mientras que en la inserci\'on s\'olo se agregan hojas de color rojo, es decir,
la altura \'unicamente se modifica en la inserción cuando se aplica el balanceo.

La acci\'on de eliminar nodos de cualquier parte de un {\arn} presenta una problemática muy grande para
el balanceo del mismo, esto se suscita al eliminar un nodo del \'arbol, los dos subárboles de este
tienen que ser concatenados de alguna forma y los invariantes de los mismos tienen que ser
respetados.

\subsection{Funci\'on de eliminaci\'on}

Para poder comprender la l\'ogica de las funciones que conforman a la operaci\'on de eliminaci\'on
es necesario comenzar por la funci'on que retira el nodo del \'arbol (ver la figura \ref{func_del}).
La idea central de esta operaci\'on es bastante simple: como los {\arns} son \'arboles de búsqueda,
lo primero que hacemos es buscar el nodo a eliminar, si se encuentra se elimina y se concatenan los
subárboles restantes de esta operaci\'on (ver figuras \ref{arbolRB_4}, \ref{arbolRB_5} y
\ref{arbolRB_6}). A continuaci\'on se describen m\'as a fondo los casos de la misma:

\begin{itemize}
    \item Si se recibe un \'arbol vacío como argumento de la funci\'on, se regresa este mismo; pues
    eliminar un elemento del \'arbol vacío termina siendo vacio. También este caso sirve para
    cuando un elemento no es encontrado en el \'arbol, es el caso base de la recursi\'on de
    búsqueda del nodo a eliminar.
    \item En otro caso, se realiza recursivamente la búsqueda del elemento a eliminar. Si el nodo
    actual no contiene el elemento que buscamos, se compara si es menor o mayor para seguir
    buscando en el \'arbol izquierdo o derecho respectivamente. Si el siguiente nodo es negro y se
    encuentra en el sub\'arbol izquierdo, se realiza la operaci\'on $lbalS$ \footnote{Funci\'on de
    balanceo extendida para subarboles izquierdos.}, por otro lado, si el nodo se encuentra en el
    sub\'arbol derecho se aplica $rbalS$ \footnote{Funci\'on de balanceo extendida para subarboles
    derechos.}. Si el elemento en el que estamos no es ni mayor ni menor al que buscamos, en ese
    caso eliminamos el elemento y concatenamos los subárboles restantes usando la función $append$
    \footnote{Funci\'on donde se juntan lo arboles restantes de esta operaci\'on}.
\end{itemize}

\begin{figure}
\centering
\captionsetup{justification=centering}
\begin{minted}{coq}
Fixpoint del {a} `{GHC.Base.Ord a} (x:a) (t:RB a) :=
 match t with
 | E => E
 | T _ a y b =>
    if x GHC.Base.< y : bool then
      match a with
       | T B _ _ _ => lbalS (del x a) y b
       | _ => T R (del x a) y b
      end
    else
    if x GHC.Base.> y : bool then
      match b with
       | T B _ _ _ => rbalS a y (del x b)
       | _ => T R a y (del x b)
      end
    else append a b
 end.

Definition remove x t := makeBlack (del x t).
\end{minted}
\caption{Función de eliminación}
\label{func_del}
\end{figure}

\begin{figure}
\centering
\captionsetup{justification=centering}
 \begin{tikzpicture}[-,level/.style={sibling distance = 4cm/#1,level distance = 1cm}]
%\draw[style=help lines] (-5,-7) grid (5,0);
\node [arn_n] at (-2,0) {6}
        child{ node [arn_r] {2}
            child{ node [arn_n] {1}
                child{ node [arn_x] {}} %for a named pointer
                child{ node [arn_x] {}}
            }
            child{ node [arn_n] {4}
                child{ node [arn_r] {3}
                    child{ node [arn_x] {}}
                    child{ node [arn_x] {}}
                }
                child{ node [arn_r] {5}
                    child{ node [arn_x] {}}
                    child{ node [arn_x] {}}
                }
            }
        }
        child{ node [arn_n] {8}
            child{ node [arn_x] {}}
            child{ node [arn_r] {9}
                child{ node [arn_x] {}}
                child{ node [arn_x] {}}
            }
        }
;

{
}
;
\end{tikzpicture}
\caption{{\Arn antes de eliminar nodo 6.}}
\label{arbolRB_4}
\end{figure}

\begin{figure}
\centering
\captionsetup{justification=centering}
 \begin{tikzpicture}[-,level/.style={sibling distance = 4cm/#1,level distance = 1cm}]
%\draw[style=help lines] (-5,-7) grid (5,0);
%\node [arn_n] at (-2,0) {6}

        \node [arn_r] {2}
            child{ node [arn_n] {1}
                child{ node [arn_x] {}} %for a named pointer
                child{ node [arn_x] {}}
            }
            child{ node [arn_n] {4}
                child{ node [arn_r] {3}
                    child{ node [arn_x] {}}
                    child{ node [arn_x] {}}
                }
                child{ node [arn_r] {5}
                    child{ node [arn_x] {}}
                    child{ node [arn_x] {}}
                }
            }
;
\end{tikzpicture}
\begin{tikzpicture}[-,level/.style={sibling distance = 4cm/#1,level distance = 1cm}]

            \node [arn_n] {8}
            child{ node [arn_x] {}}
            child{ node [arn_r] {9}
                child{ node [arn_x] {}}
                child{ node [arn_x] {}}
            }

;
\end{tikzpicture}

\caption{{\Arn} roto, después de eliminar nodo 6.}
\label{arbolRB_5}
\end{figure}

\begin{figure}
\centering
\captionsetup{justification=centering}
 \begin{tikzpicture}[-,level/.style={sibling distance = 4cm/#1,level distance = 1cm}]
%\draw[style=help lines] (-5,-7) grid (5,0);
\node [arn_n] at (-2,0) {5}
        child{ node [arn_r] {2}
            child{ node [arn_n] {1}
                child{ node [arn_x] {}} %for a named pointer
                child{ node [arn_x] {}}
            }
            child{ node [arn_n] {4}
                child{ node [arn_r] {3}
                    child{ node [arn_x] {}}
                    child{ node [arn_x] {}}
                }
                child{ node [arn_x] {}}
            }
        }
        child{ node [arn_n] {8}
            child{ node [arn_x] {}}
            child{ node [arn_r] {9}
                child{ node [arn_x] {}}
                child{ node [arn_x] {}}
            }
        }
;

{
}
;
\end{tikzpicture}
\caption{{\Arn} después de aplicar función append.}
\label{arbolRB_6}
\end{figure}

Podemos ver que las funciones de balanceo $lbalS$ y $rbalS$ se aplican cuando el nodo en el que
estamos parados, llamémoslo \textbf{\textit{n}}, es negro; esto evita que después de eliminar un nodo y aplicar la
funci\'on $append$ se acabe con dos nodos rojos seguidos, es decir, que el hijo y alguno de los
nietos del nodo \textbf{\textit{n}} sean rojos.

\subsection{Funci\'on de concatenaci\'on}

La funci\'on de concatenación (figura \ref{func_app}) es usada cuando se encuentra el elemento que
se busca eliminar de un {\arn}, esto es porque la acci\'on de retirar un nodo del \'arbol resulta
en dos \'arboles que tienen que ser concatenados, los cuales deben de respetar los invariantes de
los {\arns}. Esta funci\'on recibe como parámetros los dos \'arboles\footnote{Estos arboles pueden
no cumplir las invariantes de ls {\arns}.} que estamos buscando juntar. Esta operación se describe
con mayor detalle en seguida.

\begin{figure}
\centering
\captionsetup{justification=centering}
\begin{minted}{coq}
Fixpoint append {a} `{GHC.Base.Ord a} (l:RB a) : RB a -> RB a :=
 match l with
 | E => fun r => r
 | T lc ll lx lr =>
   fix append_l (r:RB a) : RB a :=
   match r with
   | E => l
   | T rc rl rx rr =>
     match lc, rc with
     | R, R =>
       let lrl := append lr rl in
       match lrl with
       | T R lr' x rl' => T R (T R ll lx lr') x (T R rl' rx rr)
       | _ => T R ll lx (T R lrl rx rr)
       end
     | B, B =>
       let lrl := append lr rl in
       match lrl with
       | T R lr' x rl' => T R (T B ll lx lr') x (T B rl' rx rr)
       | _ => lbalS ll lx (T B lrl rx rr)
       end
     | B, R => T R (append_l rl) rx rr
     | R, B => T R ll lx (append lr r)
     end
   end
 end.
\end{minted}
\caption{Funci\'on de concatenaci\'on, append}
\label{func_app}
\end{figure}


Sean \textbf{\textit{a}} y \textbf{\textit{b}} los dos subárboles a los que se les aplicar\'a la funci\'on append, es decir,
\textit{append a b}, tenemos los siguientes casos:

\begin{itemize}
    \item Si \textbf{\textit{a}} es el \'arbol vacío, entonces se regresa \textbf{\textit{b}}.
    \item Si \textbf{\textit{b}} es el \'arbol vacío, entonces regresamos \textbf{\textit{a}}.
    \item Si \textbf{\textit{a}} y \textbf{\textit{b}} son \'arboles con raíces rojas, entonces se aplica $append$ al subárbol
    derecho de \textbf{\textit{a}}, sea este \textbf{\textit{ar}}, junto con el subárbol izquierdo de \textbf{\textit{b}}, sea \textbf{\textit{bl}}, es decir,
    \textit{append ar bl}. Tenemos subcasos:
    \begin{itemize}
      \item Si el resultado de esta operación es un árbol con raíz roja, sea \textbf{\textit{arbl}}, los \'arboles
      \textbf{\textit{a}} y \textbf{\textit{b}} se pintan de rojo y se concatenan con la raíz de \textbf{\textit{arbl}}, igual de color rojo; \textbf{\textit{ar}}
      se reemplaza por el subárbol izquierdo de \textbf{\textit{arbl}} y \textbf{\textit{bl}} se reemplaza por el subárbol derecho
      de \textbf{\textit{arbl}}.
      \item En otro caso, si el \'arbol resultante de \textit{append ar bl} no es rojo, tomamos \textbf{\textit{a}}
      y \textbf{\textit{b}}, los pintamos de rojo, el subárbol derecho de \textbf{\textit{a}} se reemplaza por \textbf{\textit{b}} y el subárbol
      izquierdo de \textbf{\textit{b}} se reemplaza por el resultado de \textit{append ar bl}.
    \end{itemize}
    \item Si \textbf{\textit{a}} y \textbf{\textit{b}} son arboles con raíces negras, entonces se aplica $append$ al subárbol
    derecho de \textbf{\textit{a}}, sea \textbf{\textit{ar}}, con el subárbol izquierdo de \textbf{\textit{b}}, sea \textbf{\textit{bl}}, es decir, \textit{append
    ar bl}. Tenemos casos:
    \begin{itemize}
      \item Si el resultado de esta operación es un árbol con raíz roja, sea \textbf{\textit{arbl}}, los \'arboles
      \textbf{\textit{a}} y \textbf{\textit{b}} se pintan de negro y se concatenan con la raíz de \textbf{\textit{arbl}}, esta de color rojo; \textbf{\textit{ar}}
      se reemplaza por el subárbol izquierdo de \textbf{\textit{arbl}} y \textbf{\textit{bl}} se reemplaza por el subárbol derecho
      de \textbf{\textit{arbl}}.
      \item En otro caso, si el \'arbol resultante de \textit{append ar bl} no es rojo, tomamos \textbf{\textit{a}}
      y \textbf{\textit{b}}, el subárbol derecho de \textbf{\textit{a}} se reemplaza por \textbf{\textit{b}} y el subárbol izquierdo de \textbf{\textit{b}} se
      reemplaza por el resultado de \textit{append ar bl} y a este resultado le aplicamos una
      función de balanceo, $lbalS$.
    \end{itemize}
    \item Si \textbf{\textit{a}} es un \'arbol de color negro y \textbf{\textit{b}} de color rojo, entonces se toma \textbf{\textit{b}}, se pinta
    de rojo pero en lugar de su subárbol izquierdo, sea \textbf{\textit{bl}}, se aplica una llamada recursiva a
    \textbf{\textit{bl}} con la funci\'on embebida en $append$, llamada $append\_l$, es decir: \textit{append\_l
    bl}, esta llamada tambi\'en carga al \'arbol \textbf{\textit{a}} gracias al currying\cite{Currying}.
    \item Si \textbf{\textit{a}} es un \'arbol de color rojo y \textbf{\textit{b}} de color negro, entonces se toma \textbf{\textit{a}}, se pinta
    de rojo pero en lugar de su subárbol derecho, sea \textbf{\textit{ar}}, se hace una llamada recursiva con
    $append(ar,b)$.
\end{itemize}

Debemos mencionar que el \'arbol resultante de aplicar esta funci\'on no necesariamente cumple los
invariantes de un {\arn}, estas invariantes se logran conservar ya que en la funci\'on $del$ se
realizan llamadas a las funciones extendidas de balanceo, las cuales desarrollaremos en la
siguiente sección.

\subsection{Extensi\'on de funciones de balanceo}

En la secci\'on 2.2 de este trabajo se trato la inserci\'on de elementos a un {\arn}, en donde se
describen un par de funciones llamadas `funciones de balanceo', tratadas en las subsecci\'on
2.2.1, estas funciones a su vez toman los nombres $rbal$ y $lbal$ (figura \ref{func_balanceo}).
Estas funcionesresultan insuficientes para balancear un \'arbol al momento de eliminar un nodo y
concatenar los dos \'arboles restantes con la función $append$, es por eso que se implementan las
extensiones de estas funciones, llamadas $lbalS$ y $rbalS$ (figuras \ref{lbalS} y \ref{rbalS}
respectivamente) las cuales a su vez llaman a las funciones $rbal'$\footnote{La funci\'on $rbal'$
es una variación de la función $rbal$, solo cambia el orden de la caza de patrones.} (figura
\ref{rbal_2}) y $lbal$. Estas extensiones agregan mas casos de manejo de subárboles negros, esto
porque existen casos en los que se puede llegar a eliminar un nodo negro intermedio y se tiene que
asegurar que las invariantes no se violen después de concatenar los subárboles resultantes de
aplicar la función $del$.

\begin{figure}
\centering
\captionsetup{justification=centering}
\begin{minted}{coq}

Definition lbalS {a} `{GHC.Base.Ord a} (l:RB a) (k:a) (r:RB a) :=
 match l with
 | T R a x b => T R (T B a x b) k r
 | _ =>
   match r with
   | T B a y b => rbal' l k (T R a y b)
   | T R (T B a y b) z c => T R (T B l k a) y (rbal' b z (makeRed c))
   | _ => T R l k r
   end
 end.

\end{minted}
\caption{Funci\'on de balanceo de lado izquierdo extendida.}
\label{lbalS}
\end{figure}

Las funciones $rbalS$ y $lbalS$ son usadas en la funci\'on $del$ (figura \ref{func_del}) cuando el
caso en el que se cae es un nodo de color negro y al aplicar la funci\'on en estos nodos podemos
asegurar que los dos subárboles de este nodo no se van a desequilibrar, es decir, que un subárbol
tenga mas nodos negros que el otro. Después de aplicar la función de balanceo se aplica otra llamada
recursiva a $del$.


\begin{figure}
\centering
\captionsetup{justification=centering}
\begin{minted}{coq}

Definition rbalS {a} `{GHC.Base.Ord a} (l:RB a) (k:a) (r:RB a) :=
 match r with
 | T R b y c => T R l k (T B b y c)
 | _ =>
   match l with
   | T B a x b => lbal (T R a x b) k r
   | T R a x (T B b y c) => T R (lbal (makeRed a) x b) y (T B c k r)
   | _ => T R l k r
   end
 end.

\end{minted}
\caption{Funciones de balanceo de lado derecho extendida.}
\label{rbalS}
\end{figure}

Existe otra función donde se utiliza una de estas operaciones de balanceo, específicamente $lbalS$,
esta funci\'on es $append$, en el caso de que los arboles que se le esten pasando como parámetros
sean negros, esto es por la misma razón por la cual se aplican las funciones de balanceo en $del$
sobre los nodos de color negro: para que sus subárboles no se desbalanceen.

\begin{figure}
\centering
\captionsetup{justification=centering}
\begin{minted}{coq}
Definition rbal' {a} `{GHC.Base.Ord a} (l:RB a) (k:a) (r:RB a) :=
 match r with
 | T R b y (T R c z d) => T R (T B l k b) y (T B c z d)
 | T R (T R b y c) z d => T R (T B l k b) y (T B c z d)
 | _ => T B l k r
 end.

\end{minted}
\caption{Funci\'on de balanceo de lado derecho alternativa.}
\label{rbal_2}
\end{figure}

Estas definiciones y funciones son suficientes para poder eliminar nodos de un {\arn} y que el
resultado no viole los invariantes de estos. Al menos eso es lo que nos gustaria poder afirmar en
esta etapa del trabajo, sin embargo, esta sentencia tiene que ser demostrada, es decir, tenemos que
probar que nuestros {\arns} cumplen con la definición.

\chapter{Verificación de {\arns}}

En el primer cap\'itulo de este trabajo se menciono que los {\arns} son una estructura de datos
que mejora el tiempo de acceso, de inserción y eliminación de elementos con respecto a otras
estructuras de datos como: las listas simples, las listas doblemente ligadas y \'arboles de
búsqueda. En el segundo cap\'itulo se muestran las implementaciones de los algoritmos de esta
estructura de datos y podemos notar como estas implementaciones no son triviales, es decir, son
rebuscadas, enredosas y complicadas de programar en un lenguaje que utilice el paradigma de programaci\'on funcional como
lo hace {\coq}, e incluso en un lenguaje con un paradigma imperativo como \textit{Java} o \textit{C}.
Por esta razón es que nos preocupa que las implementaciones que realicemos sean correctas, en otras palabras, se desea verificar que las implementaciones de las operaciones
descritas en el capitulo anterior respeten en todos los casos los invariantes de los {\arns}.

\section{Pruebas unitarias}
Las \textit{pruebas unitarias} \cite{unittest} son bloques de c\'odigo, funciones o m\'etodos, que
invocan a otros bloques para poder verificar ciertas suposiciones sobre el programa a probar. Estas
pruebas en principio deben de ser fáciles de escribir, entender, extender, que se ejecuten en poco
tiempo y sobre todo que sean fidedignas. De nada nos servirían pruebas unitarias que estén mal
escritas o que estas mismas sean demasiado complejas y puedan contener errores.

Este tipo de pruebas son usadas para verificar que cada componente de un programa funcione de
manera correcta, en el caso de los {\arns} este tipo de pruebas nos ayudan a verificar los
invariables de un determinado \'arbol. La figura \ref{unitTestjava} es una prueba unitaria
escrita en el lenguaje de programaci\'on Java, la cual verifica la altura negra de un {\arn}.

\begin{figure}[!ht]
\centering
\captionsetup{justification=centering}
\begin{minted}{java}

    /* Valida que los caminos del vértice a sus hojas tengan todos
       el mismo número de vértices negros. */
    private static <T extends Comparable<T>> int
    validaCaminos(ArbolRojinegro<T> arbol,
                  VerticeArbolBinario<T> v) {
        int ni = -1, nd = -1;
        if (v.hayIzquierdo()) {
            VerticeArbolBinario<T> i = v.izquierdo();
            ni = validaCaminos(arbol, i);
        } else {
            ni = 1;
        }
        if (v.hayDerecho()) {
            VerticeArbolBinario<T> d = v.derecho();
            nd = validaCaminos(arbol, d);
        } else {
            nd = 1;
        }
        Assert.assertTrue(ni == nd);
        switch (arbol.getColor(v)) {
        case NEGRO:
            return 1 + ni;
        case ROJO:
            return ni;
        default:
            Assert.fail();
        }
        // Inalcanzable.
        return -1;
    }

\end{minted}
\caption{Prueba unitaria escrita en Java.\cite{CanekPU}}
\label{unitTestjava}
\end{figure}

Sin embargo, el hecho de que las pruebas unitarias puedan verificar los invariantes de un \'arbol
dado, no nos asegura que todos los \'arboles creados por nuestras operaciones de inserci\'on y
eliminaci\'on los respeten. La \'unica manera de que esta prueba podr\'ia verificar esto ser\'ia
realizando todas las combinaciones de operaciones y entradas posibles y aplicar la prueba a todos los
resultados de estas entradas. Esto es una prueba exhaustiva y en este caso \footnote{De hecho en la
mayoria.} las posibildades son infinitas, es decir, no existe modo de realizar todas estas pruebas
en un tiempo razonable, lo cual contradice totalmente la definici\'on de prueba unitaria.


\section{Verificaci\'on Formal en {\coq}}

Es claro que las pruebas unitarias no nos son suficientes para poder verificar formalmente un
programa, es por esto que se requieren realizar demostraciones matemáticas para poder obtener los
resultados que buscamos, pero de igual manera no queremos escribir a mano estas demostraciones ya
que al igual que las pruebas unitarias estas son susceptibles al error humano, por esta raz\'on se
usar\'a {\coq} para realizar estas pruebas formales.

La verificaci\'on formal de un programa usando {\coq} esta compuesto de las siguientes etapas:
\begin{itemize}
    \item Capturar los invariantes de los {\arns} usando definiciones inductivas en {\coq}, de
    esta manera podemos saber si las operaciones que se implementaron las respetan.
    \item Enunciar lemas, corolarios y teoremas que describan los comportamientos de las
    operaciones que queremos verificar y escribirlos en {\coq}, usando las definiciones inductivas
    descritas en el punto anterior.
    \item Por \', demostrar todos los enunciados del punto anterior utilizando las t\'acticas
    que el asistente de pruebas nos provee.
\end{itemize}{}

\subsection{Capturando invariantes de los {\Arns}}
Una de las etapas m\'as importantes al realizar la verificaci\'on formal en {\coq} de cualquier estructura de
datos, inclusive de cualquier programa, es capturar sus invariantes de manera correcta, es decir,
poder escribir una o varias definiciones inductivas que describan a la estructura de datos y sus
invariantes. Después, con estas mismas es que se enuncian los lemas, clases y posteriormente
instancias de las clases.

A continuaci\'on se describen dos conjuntos de definiciones inductivas, muy similares entre ellas,
los cuales nos ayudaran a verificar formalmente los {\arns}. La primera es un primer intento que es
insuficiente ya que los tipos inductivos y los principios de demostraci\'on no son los \'optimos.
El segundo intento es un conjunto de definiciones inductivas que tienen mas detalle para describir
los invariantes. Estas definiciones est\'an relacionadas con las propiedades de las operaciones de
inserci\'on y eliminaci\'on.

\subsubsection{Primer Conjunto de Definiciones Inductivas}
Los dos conjuntos de definiciones inductivas comparten la misma idea: una definici\'on que describe
estrictamente lo que es un {\arn} y otra definici\'on m\'as laxa de la misma.

\begin{figure}[!ht]
\centering
\captionsetup{justification=centering}
\begin{minted}{coq}
Inductive isRB : Tree -> color -> nat -> Prop :=
 | IsRB_leaf: forall c, isRB E c 0
 | IsRB_r: forall tl k kv tr n,
          isRB tl Red n ->
          isRB tr Red n ->
          isRB (T Red tl k kv tr) Black n
 | IsRB_b: forall c tl k kv tr n,
          isRB tl Black n ->
          isRB tr Black n ->
          isRB (T Black tl k kv tr) c (S n).
\end{minted}
\caption{Funci\'on inductiva isRB.}
\label{inductive_isRB}
\end{figure}

La primera definici\'on llamada $isRB$ (figura \ref{inductive_isRB}) tiene tres casos, los
cuales describiremos a continuaci\'on:
\begin{itemize}
        \item \textbf{IsRB\_Leaf}: el árbol vacío con altura negra 0 es roji-negro. En este caso
        se tiene un s\'olo nodo, es decir, una hoja\footnote{Recordemos que las hojas son vacias y de
        color negro.}.
        \item \textbf{IsRB\_r}: Para cualesquiera \'arboles \textbf{\textit{tl}} y \textbf{\textit{tr}} que cumplan con la
        definici\'on $isRB$ con color rojo y altura \textbf{\textit{n}}, se cumple que un \'arbol de color rojo con
        \textbf{\textit{tl}} y \textbf{\textit{tr}} como subarboles, sea \textbf{\textit{t}}, cumple con $isRB$ con color negro y altura \textbf{\textit{n}}. El
        color se refiere al color del padre, en este caso, en la llamada de $isRB$ a \textbf{\textit{tl}} y \textbf{\textit{tr}} se
        le pasa el color rojo porque \textbf{\textit{t}} es rojo. La altura \textbf{\textit{n}} se refiere a que hay \textbf{\textit{n}} nodos
        negros en cualquier camino del nodo actual a alguna hoja, aqui no crece \textbf{\textit{n}} porque tanto
        \textbf{\textit{tl}} y \textbf{\textit{tr}} son rojos.
        \item \textbf{IsRB\_b}: Para cualesquiera \'arboles \textbf{\textit{tl}} y \textbf{\textit{tr}} que cumplan con la
        definici\'on $isRB$ con color negro y altura \textbf{\textit{n}}, se cumple que un \'arbol de color negro
        con \textbf{\textit{tl}} y \textbf{\textit{tr}} como subarboles, sea \textbf{\textit{t}}, cumple con $isRB$ con cualquier color y altura
        $S(n)$. En este caso, en la llamada de $isRB$ a \textbf{\textit{tl}} y \textbf{\textit{tr}} se le pasa el color negro
        porque \textbf{\textit{t}} es negro y aqui se le suma uno a \textbf{\textit{n}} porque tanto \textbf{\textit{tl}} y \textbf{\textit{tr}} son negros.
\end{itemize}

Con estos tres casos podemos asegurar que los invariantes se respetan, pero esta funci\'on
inductiva es demasiado restrictiva y esto dificulta poder demostrar las propiedades de los {\arns},
por esto pasamos a la segunda definici\'on inductiva, $nearRB$, esta permite mas flexibilidad en el
\'arbol, se muestra y describe en la figura \ref{inductive_nearRB}.
%punto
\begin{figure}[!ht]
\centering
\captionsetup{justification=centering}
\begin{minted}{coq}
Inductive nearRB : Tree -> nat -> Prop :=
| nrRB_r: forall tl k kv tr n,
         isRB tl Black n ->
         isRB tr Black n ->
         nearRB (T Red tl k kv tr) n
| nrRB_b: forall tl k kv tr n,
         isRB tl Black n ->
         isRB tr Black n ->
         nearRB (T Black tl k kv tr) (S n).
\end{minted}
\caption{Funci\'on inductiva nearRB.}
\label{inductive_nearRB}
\end{figure}

Podemos apreciar que solo se tienen dos casos y no se tiene un argumento para un color, sin
embargo, a diferencia de $isRB$ esta no se llama recursivamente, en lugar de eso se llama a $isRb$
inmediatamente, además podemos ver que ambas definiciones comparten el contador de nodos negros.
Con estas modificaciones se permite una cosa, que en la ra\'iz del \'arbol puedan haber a lo m\'as
dos nodos rojos contiguos.

\paragraph{Intento de Verificaci\'on}
Utilizando las funciones inductivas descritas en esta secci\'on se realiz\'o un intento fallido de
verificac\'on de la operaci\'on de inserci\'on, como se muestra en \cite{appel}, sin embargo, al
estar desarrollando la demostraci\'on se encontró un problema, la falta de un conjunto de
hipótesis para poder probar una meta. Esto se debe probablemente a una mala elección de estilo de
demostraci\'on, implementaci\'on o de las definiciones inductivas mostradas anteriormente. Se noto
que el hecho de que toda la informaci\'on referente a los invariantes estuviera codificada en las
dos funciones inductivas, sin uso de ``funciones auxiliares'' complica la verificaci\'on. Se
llego a esta conclusi\'on ya que el caso ``sencillo'' de la verificaci\'on de {\arns} es la inserci
\'on y con este conjunto de funciones inductivas las demostraciones se volvían muy largas y
complicadas de seguir.

\subsubsection{Segundo Conjunto de Definiciones Inductivas}

Con el conocimiento que se obtuvo del conjunto de definiciones anterior, nos realizamos la
siguiente pregunta: ¿c\'omo capturar las invariantes de los {\arns}, y al mismo tiempo facilitar la
verificaci\'on de estos?

Utilizamos una definición inductiva, llamada $is\_redblack$ para poder capturar los invariantes,
la cual lleva como parámetros un contador y un \'arbol. El contador lleva el control de la
cantidad de nodos negros, es decir, la altura negra del nodo, mientras que el \'arbol es aquel que
estamos buscando verificar que cumpla con las invariantes de un {\arn}. Se presenta esta definici\'
on en la figura \ref{inductive_isRedB}.

\begin{figure}[!ht]
\centering
\captionsetup{justification=centering}
\begin{minted}{coq}
Inductive is_redblack {a} `{GHC.Base.Ord a} : nat -> RB a -> Prop :=
 | RB_Leaf : is_redblack 0 E
 | RB_R n l k r : notred l -> notred r ->
                  is_redblack n l -> is_redblack n r ->
                  is_redblack n (T R l k r)
 | RB_B n l k r : is_redblack n l -> is_redblack n r ->
                  is_redblack (S n) (T B l k r).
\end{minted}
\caption{Funci\'on inductiva $is\_redblack$.}
\label{inductive_isRedB}
\end{figure}

Podemos notar ciertas similitudes con la definición inductiva de la secci\'on pasada, sin
embargo, el principal cambio que presenta esta definición, es el hecho de que se dejan de controlar
los colores de los subárboles en los parámetros de la definici\'on y se crea la funci\'on $notred$,
la cual, como su nombre dice, verifica que el \'arbol que se este pasando no tenga raíz roja. La
definici\'on $is\_redblack$ tiene tres casos,$RB\_Leaf$, $RB\_R$ y $RB\_B$. Desarrollando la idea de cada caso:

\begin{itemize}
        \item \textbf{RB\_Leaf}: el árbol vació es roji-negro. Este caso nos dice que el \'arbol
        vacío es un {\arn}.
        \item \textbf{RB\_R}: un árbol rojo donde lleves contados \textbf{\textit{n}} nodos negros, donde sus
        hijos sean{\arns} y no sean rojos. Este caso nos dice explícitamente que los subárboles
        del árbol que esta recibiendo la función no pueden ser rojos, esto porque el árbol que se
        esta analizando tiene raíz roja. Como no se esta analizando algún nodo negro, la altura
        negra se mantiene en \textbf{\textit{n}}.
        \item \textbf{RB\_B}: un árbol negro donde lleves contados $n+1$ nodos negros, incluido
        el actual, y sus hijos sean {\arns}. En este ultimo caso se tiene la libertad de que los
        subárboles sean del color que sea, pero la altura del consecuente es $S(n)$ porque el nodo
        que se esta analizando es de color negro, los antecedentes al no tomar en cuenta a su nodo
        padre tienen altura \textbf{\textit{n}}.
\end{itemize}

Esta definici\'on captura los invariantes que estamos buscando, sin embargo, no es suficiente para
poder probar la correcci\'on de los {\arns}, la definici\'on es demasiado restrictiva y costaría mucho
trabajo proceder con las demostraciones solamente con ella. Por esta razón se agregan dos
definiciones inductivas auxiliares; $redred\_tree$ y $nearly\_redblack$ (figura \ref{inductive_aux}).

\begin{figure}[!ht]
\centering
\captionsetup{justification=centering}
\begin{minted}{coq}
Inductive redred_tree {a}
                  `{GHC.Base.Ord a} (n:nat) : RB a -> Prop :=
 | RR_Rd l k r : is_redblack n l -> is_redblack n r
                                 -> redred_tree n (T R l k r).

Inductive nearly_redblack {a}
                  `{GHC.Base.Ord a} (n:nat)(t:RB a) : Prop :=
 | ARB_RB : is_redblack n t -> nearly_redblack n t
 | ARB_RR : redred_tree n t -> nearly_redblack n t.
\end{minted}
\caption{Funciones inductivas $redred\_tree$ y $nearly\_redblack$.}
\label{inductive_aux}
\end{figure}

Podemos notar que estas definiciones son versiones menos restrictivas de $is\_redblack$. La
definici\'on $nearly\_redblack$ permite que existan dos nodos rojos en la ra\'iz del \'arbol,
aprovech\'andose de $redred\_tree$, pues esta definici\'on es exactamente el caso $RB\_R$ de
$is\_redblack$ pero sin las restricciones de que los sub\'arboles sean rojos, lo cual nos permite
que hayan dos nodos rojos exactamente en la ra\'iz. Entonces un $nearly\_redblack$ es un {\arn} con
la excepci\'on de que la ra\'iz puede ser roja.

Finalmente, lo que se busca demostrar es que los {\arns} con las operaciones de inserci\'on y
eliminaci\'on est\'en dentro de la clase de \'arboles $redblack$ (figura \ref{class_rb}).

\begin{figure}[!ht]
\centering
\captionsetup{justification=centering}
\begin{minted}{coq}
Class redblack {a} `{GHC.Base.Ord a} (t:RB a) :=
                            RedBlack : exists d, is_redblack d t.
\end{minted}
\caption{Clase de \'arboles $redblack$.}
\label{class_rb}
\end{figure}

Lo que estamos describiendo con el enunciado de la figura \ref{class_rb} es que un {\arn} es aquel que tiene una altura
negra $d$ y cumple con las invariantes establecidas por la definici\'on $is\_redblack$.

\paragraph{Segundo Intento de Verificaci\'on}
En contraste con el conjunto de definiciones de la secci\'on pasada, la definici\'on de
$nearly\_redblack$ se reescribe, dejando de codificar las invariantes en las llamadas recursivas
de la definición, creando funciones auxiliares para capturar los invariantes de manera mas
sencilla, como $redred\_tree$ y $notred$. Además se crea la clase de {\arns}, lo cual afina mas la
definici\'on de los mismos. Tomando en cuenta todas estas modificaciones a las definiciones fue que
se eligió este conjunto para verificar formalmente la estructura de datos\footnote{La elecci\'on
de este conjunto fue correcta ya que facilito la demostraci\'on de las propiedades y probo ser
suficiente para verificar la estructura.}.
Estas definiciones inductivas fueron obtenidas de \cite{MSetRBT}.

\subsection{Verificación de la operación de inserción}

Para poder realizar la verificaci\'on de la operaci\'on de inserci\'on es necesario escribir
enunciados, ya sean lemas, proposiciones, etc. Estos enunciados los escribiremos usando las definiciones inductivas presentadas en
la secci\'on pasada, es decir, $is\_redblack$, $nearly\_redblack$ y $redred\_tree$.

A continuaci\'on mostraremos los lemas $ins\_rr\_rb$, $ins\_arb$ y una instancia \cite{classes}
de la clase $redblack$, $add\_rb$. Estos lemas y la instancia fueron obtenidos de \cite{MSetRBT},
la idea principal de estos enunciados es explicarnos que ciertos \'arboles de b\'usqueda que respeten las definiciones
mas generales, es decir, $nearly\_redblack$ y $redred\_tree$, también como consecuencia respetar\'an
$is\_redblack$.

\subsubsection{Primer Lema}

\begin{figure}[!ht]
\centering
\captionsetup{justification=centering}
\begin{minted}{coq}
Lemma ins_rr_rb {a} `{GHC.Base.Ord a} (x:a) (s: RB a) (n : nat) :
is_redblack n s ->
     ifred s (redred_tree n (ins x s)) (is_redblack n (ins x s)).
\end{minted}
\caption{Lema $ins\_rr\_rb$}
\label{lema_1}
\end{figure}

En este primer lema (figura \ref{lema_1}) enunciamos lo siguiente: sea \textit{\textbf{s}} un {\arn} bajo la
definici\'on de $is\_redblack$, entonces si \textit{\textbf{s}} es un \'arbol con raíz roja, insertar un elemento
\textit{\textbf{x}} en \textit{\textbf{s}} resulta en un \'arbol que cumple la definci\'on de $redred\_tree$, en otro caso cumple
con la definici\'on de $is\_redblack$.

En otras palabras, lo que este enunciado quiere decirnos es que si tenemos un {\arn} e insertamos
un elemento a ese \'arbol el resultado puede tener ra\'iz roja, e incluso puede tener dos nodos
rojos, uno en la ra\'iz y otro en cualquiera de los dos, o incluso en los dos nodos siguientes.

La demostraci\'on de este lema comienza con una inducci\'on sobre el antecedente del enunciado, lo cual
resulta en tres casos:

 \begin{minted}{coq}
______________________________________(1/3)
ifred E (redred_tree 0 (ins x E)) (is_redblack 0 (ins x E))
______________________________________(2/3)
ifred (T R l k r) (redred_tree n (ins x (T R l k r)))
                  (is_redblack n (ins x (T R l k r)))
______________________________________(3/3)
ifred (T B l k r) (redred_tree (S n) (ins x (T B l k r)))
                  (is_redblack (S n) (ins x (T B l k r)))
 \end{minted}

La funci\'on $ifred$ que se usa en este lema, es una funci\'on auxiliar que nos ayuda a decidir si
un \'arbol es rojo o no.

En el primero de estos casos notamos que su soluci\'on se da simplificando las funciones y resulta
en uno de los casos de $is\_redblack$, especificamente en el caso $RB\_R$, ya que el \'arbol vacío
no es rojo y la simplificaci\'on de $(ins(x,E))$ resulta en un \'arbol rojo con un elemento, esto
por definici\'on de $ins$.

Los dos casos sobrantes estan relacionados con los colores de las raices del \'arbol, en el
segundo el \'arbol es rojo y en el tercero es negro.

Analicemos el segundo caso; como el \'arbol es rojo entramos al primer caso de $if\_red$, es decir,
al caso donde se aplica la definici\'on $redred\_tree$, lo cual significa que al insertar un elemento al
\'arbol rojo, sin tener conocimiento de como son los subárboles de este, puede resultar en un
\'arbol con uno o dos nodos rojos consecutivos en la ra\'iz del mismo, ya que la operaci\'on de
balanceo se fija en los nodos hijos y nietos del nodo al que se le aplica la operaci\'on, y como
los nodos hijos de la raíz no tienen nodo abuelo, el balanceo no se efectúa en los nodos
de la raíz, dando lugar a arboles con uno o mas nodos rojos consecutivos en la raíz\footnote{hasta 3, la raíz y
sus hijos.}.

El caso sobrante, es decir, el caso del \'arbol con raíz negra se
complica un poco mas que el anterior ya que este es el caso en el que la altura negra del \'arbol
se ve modificada, en otras palabras, puede aumentar en uno. Este caso se verifica con las dos funciones de balanceo, $lbal$ y $rbal$:

\begin{minted}{coq}
H1_ : is_redblack n l
H1_0 : is_redblack n r
IHis_redblack1 :
    ifred l (redred_tree n (ins x l)) (is_redblack n (ins x l))
IHis_redblack2 :
    ifred r (redred_tree n (ins x r)) (is_redblack n (ins x r))
______________________________________(1/2)
is_redblack (S n) (lbal (ins x l) k r)
______________________________________(2/2)
is_redblack (S n) (rbal l k (ins x r))
\end{minted}

Estos casos son análogos y los dos se resuelven simplificando las funciones de balanceo, en otros términos,
simplificando las expresiones y aplicando las definiciones inductivas\footnote{$is\_redblack$,
$nearly\_redblack$}.

\subsubsection{Segundo Lema}

\begin{figure}[!ht]
\centering
\captionsetup{justification=centering}
\begin{minted}{coq}
Lemma ins_arb {a} `{GHC.Base.Ord a} (x:a) (s:RB a) (n:nat) :
is_redblack n s -> nearly_redblack n (ins x s).
\end{minted}
\caption{Lema $ins\_arb$}
\label{lema_2}
\end{figure}

Este segundo lema (figura \ref{lema_2}) enuncia lo que en el cap\'itulo anterior se menciono acerca
de la funci\'on $ins$: la funci\'on $ins$ no garantiza que el \'arbol resultante sea un {\arn}, ya
que es posible que se termine la ejecuci\'on de la funci\'on con un nodo rojo como raíz. La
demostraci\'on comienza introduciendo los antecedentes a las hipótesis y aplicando el lema
anterior, $ins\_rr\_rb$, a una de las hip\'otesis:

\begin{minted}{coq}
H1 : ifred s (redred_tree n (ins x s)) (is_redblack n (ins x s))
______________________________________(1/1)
nearly_redblack n (ins x s)

\end{minted}

Aplicamos el lema $ins\_rr\_rb$ a la hip\'otesis porque este nos genera dos casos, uno con la funci\'on $redred\_tree$ y otro con $is\_redblack$. Lo que hace necesario
que se aplique este lema, es que con ayuda de las definiciones que introduce son con las que $nearly\_redblack$ fue definido.

Como no se sabe si el \'arbol \textit{\textbf{s}} tiene ra\'iz roja o negra, se tienen que probar los dos casos:
uno con la hipótesis de que el \'arbol resultante sea $is\_redblack$ y otro con $redred\_tree$.
Estos casos se resuelven sencillamente con la aplicación de alguno de los dos casos de la definici\'on
inductiva de $nearly\_redblack$, respectivamente.

\subsubsection{Instancia de la Funci\'on de Inserci\'on}

Para poder probar que el resultado de la inserci\'on es una instancia de la clase $redblack$,es decir, que se cumplen con las propiedades descritas por la clase, vamos a necesitar el
lema auxiliar que se encuentra en la figura \ref{lema_3}:
\begin{figure}[!ht]
\centering
\captionsetup{justification=centering}
\begin{minted}{coq}
Lemma makeBlack_rb {a} `{GHC.Base.Ord a} n t :
nearly_redblack n t -> redblack (makeBlack t).
\end{minted}
\caption{Lema $makeBlack\_rb$}
\label{lema_3}
\end{figure}

Este lema auxiliar se resuelve siguiendo la idea de que la definici\'on inductiva $nearly\_redblack$ únicamente viola las invariantes al poder tener dos 
nodos rojos consecutivos en su raíz, entonces, si se ``pinta'' la raíz de negro con la función $makeBlack$, el \'arbol resultante es un {\arn} le cual respeta la la definición de la clase $redblack$.

La figura \ref{instance_ins} enuncia la instancia de inserci\'on de la clase $redblack$, la cual también nos generar\'a una demostraci\'on:

\begin{figure}[!ht]
\centering
\captionsetup{justification=centering}
\begin{minted}{coq}
Instance add_rb {a} `{GHC.Base.Ord a} (x:a) (s: RB a) :
redblack s -> redblack (insert x s).
\end{minted}
\caption{Instancia de inserci\'on de la clase $redBlack$.}
\label{instance_ins}
\end{figure}

Para poder crear la instancia de la clase $redblack$ es necesario usar la definici\'on $insert$,
la cual es una envoltura para la funci\'on $ins$. Esta funci\'on lo que hace es ``pintar'' la
ra\'iz del \'arbol resultante de la funci\'on $ins$ de color negro. De esta manera podemos
asegurar que el \'arbol resultante ya no entra en la definici\'on de $redred\_tree$.

\begin{minted}{coq}
H1 : is_redblack n s
______________________________________(1/1)
redblack (makeBlack (ins x s))
\end{minted}

En este momento se utiliza el lema auxiliar $makeBlack\_rb$ el cual nos devuelve lo siguiente:

\begin{minted}{coq}
H1 : is_redblack n s
______________________________________(1/1)
nearly_redblack n (ins x s)
\end{minted}

Esta ultima meta se resuelve aplicando el segundo lema que enunciamos, $ins\_arb$, lo cual nos deja
con una meta idéntica a la hipótesis H1 y con esto terminamos la verificaci\'on de la operaci\'on
de inserci\'on.

Se puede decir que esta implementaci\'on de la funci\'on de inserci\'on es correcta
respecto a las invariantes establecidas en la definici\'on inductiva $is\_redblack$. La operaci\'on ha
sido verificada formalmente, ahora continuaremos con la funci\'on de eliminaci\'on.

\subsection{Verificación de la operación de eliminación}
Al igual que en la funci\'on de inserci\'on se enuncian lemas para ayudarnos a llegar al resultado
de verificar la operación de eliminación. Estos lemas giran en torno a las funciones auxiliares
que se usaron para poder demostrar la operación, como \hyperref[func_app]{$append$} y 
\hyperref[func_del]{$del$}.

A continuación se describe el razonamiento usado para poder verificar dichas funciones.

\subsubsection{Primer lema}


La funci\'on m\'as importante para la operaci\'on de eliminaci\'on es \hyperref[func_app]{$append$}, 
la cual concatena dos subárboles. Estos dos subárboles son el resultado de buscar, encontrar y 
eliminar un nodo. En este primer lema se enuncia lo antes descrito: que para cualesquiera dos 
\'arboles si estos cumplen con la definici\'on inductiva de 
\hyperref[inductive_isRedB]{$is\_redblack$}, ambos con altura $n$, el resultado de concatenar es 
casi un {\arn}, en otras palabras, la concatenaci\'on cumple con la definición de
\hyperref[inductive_isRedB]{$nearly\_redblack$}. Pero si los \'arboles que se van a concatenar 
además de cumplir con \hyperref[inductive_isRedB]{$is\_redblack$}, también cumplen con $notred$, es 
decir, las ra\'ices de dichos \'arboles no son rojas, el resultado de concatenar respeta también la 
definici\'on \hyperref[inductive_isRedB]{$is\_redblack$}. La demostracio\'on de este lema en {\coq} 
se describe en seguida:

\begin{minted}{coq}
______________________________________(1/1)
Forall (r : RB a) (n : nat),
  is_redblack n l
  -> is_redblack n r
    -> nearly_redblack n (append l r)
      /\ (notred l -> notred r -> is_redblack n (append l r))
\end{minted}

\begin{figure}[!ht]
  \centering
  \captionsetup{justification=centering}
  \begin{minted}{coq}
  Lemma append_arb_rb {a} `{GHC.Base.Ord a} (n:nat) (l r: RB a) :
  is_redblack n l -> is_redblack n r ->
  (nearly_redblack n (append l r)) /\
  (notred l -> notred r -> is_redblack n (append l r)).
  \end{minted}
  \caption{Lema $append\_arb\_rb$.}
  \label{lema_4}
  \end{figure}

En este primera etapa de la demostraci\'on podemos ver lo que se describió en el párrafo anterior.
Se decidió proseguir con esta demostraci\'on usando inducci\'on, primero sobre el árbol $l$ y
posteriormente sobre $r$. Los casos base de estas inducciones consisten en simplificación de las
expresiones y fácilmente se llega a una hipótesis o a una contradicci\'on. Estos casos no se
trataran m\'as a fondo en este trabajo, pasaremos directamente a los casos m\'as interesantes.



\begin{figure}[!ht]
\centering
\captionsetup{justification=centering}
\begin{minted}{coq}
______________________________________(1/4)
forall n : nat,
is_redblack n (T R ll lx lr)
-> is_redblack n (T R rl rx rr)
-> nearly_redblack n (append (T R ll lx lr) (T R rl rx rr))
/\ (notred (T R ll lx lr)
-> notred (T R rl rx rr)
-> is_redblack n (append (T R ll lx lr) (T R rl rx rr)))
______________________________________(2/4)
forall n : nat,
is_redblack n (T R ll lx lr)
-> is_redblack n (T B rl rx rr)
-> nearly_redblack n (append (T R ll lx lr) (T B rl rx rr))
/\ (notred (T R ll lx lr)
-> notred (T B rl rx rr)
-> is_redblack n (append (T R ll lx lr) (T B rl rx rr)))
______________________________________(3/4)
forall n : nat,
is_redblack n (T B ll lx lr)
-> is_redblack n (T R rl rx rr)
-> nearly_redblack n (append (T B ll lx lr) (T R rl rx rr))
/\ (notred (T B ll lx lr)
-> notred (T R rl rx rr)
-> is_redblack n (append (T B ll lx lr) (T R rl rx rr)))
______________________________________(4/4)
forall n : nat,
is_redblack n (T B ll lx lr)
-> is_redblack n (T B rl rx rr)
-> nearly_redblack n (append (T B ll lx lr) (T B rl rx rr))
/\ (notred (T B ll lx lr)
-> notred (T B rl rx rr)
-> is_redblack n (append (T B ll lx lr) (T B rl rx rr)))
\end{minted}
\caption{Casos del lema $append\_arb\_rb$.}
\label{casos_append}
\end{figure}

Esta doble inducci\'on nos deja con los siguientes cuatro casos, expuestos en la figura \ref{casos_append}:
\begin{itemize}
    \item Los \'arboles a concatenar son rojos.
    \item El \'arbol que se concatenar\'a a la izquierda es rojo y el derecho 
es negro.
    \item El \'arbol que se concatenar\'a a la izquierda es negro y el derecho 
es rojo.
    \item Los \'arboles a concatenar son negros.
\end{itemize}

En estos cuatro casos se cuida el hecho de no desbalancear el \'arbol, en otras palabras, 
que la altura negra sea la misma al terminar de la concatenaci\'on. Por eso es que se tiene cuidado
especial en los casos donde se manejan nodos negros, ya que \'estos son los únicos nodos
considerados para el balanceo.

En las siguientes subsecciones explicamos mas a fondo los pasos usados para probar estos casos.


\paragraph{Concatenaci\'on de dos \'arboles rojos.}

Este primer caso es la concatenaci\'on de dos \'arboles con raíces rojas, en el siguiente
fragmento de la salida del asistente de pruebas se observa c\'omo la meta es una conjunci\'on.

\begin{minted}{coq}
IHlr : forall (r : RB a) (n : nat),
    is_redblack n lr
     -> is_redblack n r
       -> nearly_redblack n (append lr r)
         /\ (notred lr -> notred r -> is_redblack n (append lr r))
IHrl : forall n : nat,
     is_redblack n (T R ll lx lr)
     -> is_redblack n rl
       -> nearly_redblack n (append (T R ll lx lr) rl)
         /\ (notred (T R ll lx lr)
            -> notred rl -> is_redblack n (append (T R ll lx lr) rl))
______________________________________(1/1)
forall n : nat,
  is_redblack n (T R ll lx lr)
  -> is_redblack n (T R rl rx rr)
   -> nearly_redblack n (append (T R ll lx lr) (T R rl rx rr))
     /\ (notred (T R ll lx lr)
       -> notred (T R rl rx rr)
        -> is_redblack n (append (T R ll lx lr) (T R rl rx rr)))
\end{minted}

Podemos observar que la segunda parte de la conjunci\'on es una contradicci\'on, ya que al
introducir los antecedentes de la meta tendríamos lo siguiente:

\begin{minted}{coq}
H21 : notred (T R ll lx lr)
H22 : notred (T R rl rx rr)
______________________________________(1/1)
is_redblack n (append (T R ll lx lr) (T R rl rx rr))
\end{minted}

Evidentemente las dos funciones $notred$ de las hip\'otesis $H21$ y $H22$ se eval\'uan a falso y por
esto es una contradicci\'on.

Nos queda por demostrar la primera parte de la conjunci\'on. La meta de este caso, como se ve en
seguida, es que al ser concatenados un par de \'arboles rojos el árbol resultante cumple con la
definici\'on de ser de \hyperref[inductive_isRedB]{$nearly\_redblack$}, es decir, que la raíz del 
\'arbol puede tener dos nodos rojos consecutivos.

\begin{minted}{coq}
______________________________________(1/2)
nearly_redblack n (append (T R ll lx lr) (T R rl rx rr))
\end{minted}

El siguiente paso ser\'ia simplificar esta expresión, la cual cae en el caso de dos nodos rojos de la 
función \hyperref[func_app]{$append$} y nos resulta en la siguiente meta:

\begin{minted}{coq}
______________________________________(1/1)
redred_tree n
  match append lr rl with
  | T R lr' x rl' => T R (T R ll lx lr') x (T R rl' rx rr)
  | _ => T R ll lx (T R (append lr rl) rx rr)
  end
\end{minted}

Podemos ver que la caza de patrones depende de la evaluaci\'on de la expresi\'on $append(lr,rl)$,
digamos $rbt$, esto nos dar\'ia dos casos:

\begin{itemize}
    \item El primer caso, como se ve en seguida, se puede resolver usando las definiciones
    inductivas de \hyperref[inductive_isRedB]{$redred\_tree$} e 
    \hyperref[inductive_isRedB]{$is\_redblack$}, y las metas resultantes son resultados directos de 
    aplicar las hipótesis que se muestran.
    \begin{minted}{coq}
        H8 : notred lr
        H9 : is_redblack n ll
        H14 : notred rl
        H16 : notred rr
        H18 : is_redblack n rr
        H19 : nearly_redblack n E
        H20 : notred lr -> notred rl -> is_redblack n E
        H21 : redred_tree n E
        ______________________________________(1/2)
        redred_tree n (T R ll lx (T R E rx rr))
    \end{minted}
    \item El segundo caso es un poco m\'as complejo, pues se tienen que ver los casos en que $rbt$,
    resulta en un \'arbol con raíz roja y negra:
    \begin{itemize}
        \item En caso de que el \'arbol sea rojo, se aplica de igual manera las definiciones
        inductivas mencionadas en el caso anterior y las metas resultantes son implicaciones
        directas de las hipótesis que se muestran.
    \begin{minted}{coq}
    H6 : notred ll
    H9 : is_redblack n ll
    H14 : notred rl
    H16 : notred rr
    H20 : notred lr ->
          notred rl -> is_redblack n (T R r1_1 a0 r1_2)
    ______________________________________(1/1)
    redred_tree n (T R (T R ll lx r1_1) a0 (T R r1_2 rx rr))
    \end{minted}
        \item El caso donde el \'arbol es negro, al igual que en el caso pasado se hacen uso de
        las definicionces inductivas ya mencionadas y se siguen directamente de las siguientes
        hipótesis.
    \begin{minted}{coq}
    H6 : notred ll
    H8 : notred lr
    H9 : is_redblack n ll
    H10 : is_redblack n lr
    H14 : notred rl
    H16 : notred rr
    H17 : is_redblack n rl
    H18 : is_redblack n rr
    H19 : nearly_redblack n (T B r1_1 a0 r1_2)
    H20 : notred lr ->
          notred rl -> is_redblack n (T B r1_1 a0 r1_2)
    ______________________________________(1/1)
    redred_tree n (T R ll lx (T R (T B r1_1 a0 r1_2) rx rr))
    \end{minted}
    \end{itemize}
\end{itemize}

Con este procedimiento queda demostrado este caso de concatenar/unir dos \'aboles rojos con la 
funci\'on \hyperref[func_app]{$append$}. Se puede apreciar como los pasos de la demostraci\'on 
tienden a repetirse, esto puede significar que existan una serie de comandos y/o t\'acticas del 
asistente de pruebas que nos ayuden a acortar esta prueba. Sin embargo, en \'este trabajo se esta 
tomando el camino m\'as extenso para mostrar la simplificaci\'on de la l\'inea de pensamiento al 
demostrar estructuras complejas.

\paragraph{Concatenaci\'on de un \'arbol rojo y uno negro.}

Ahora es turno de analizar la demonstraci\'on del caso donde se concatena un \'arbol rojo y uno 
negro, es decir, $append(r,b)$, donde \textbf{\textit{r}} es un \'arbol rojo y \textbf{\textit{b}} es uno negro.

\begin{minted}{coq}
IHlr : forall (r : RB a) (n : nat),
     is_redblack n lr
     -> is_redblack n r
       -> nearly_redblack n (append lr r)
         /\ (notred lr -> notred r -> is_redblack n (append lr r))
______________________________________(1/1)
forall n : nat,
  is_redblack n (T R ll lx lr)
  -> is_redblack n (T B rl rx rr)
   -> nearly_redblack n (append (T R ll lx lr) (T B rl rx rr))
     /\ (notred (T R ll lx lr)
       -> notred (T B rl rx rr)
        -> is_redblack n (append (T R ll lx lr) (T B rl rx rr)))
\end{minted}

En este segundo caso la conjunci\'on tambi\'en contiene una contradicci\'on en la mitad derecha de
\'esta, ya que se tiene \textit{notred (T R ll lx lr)}, entonces al igual que el caso pasado s\'olo
resolveremos la primera mitad de la conjunci\'on. Para esta demostración tenemos que guiar al 
asistente de pruebas un poco m\'as de lo normal, pues le tenemos que decirle que $r$ y el árbol 
\textit{(T B rl rx rr)} son el mismo.

\begin{minted}{coq}
r := T B rl rx rr : RB a
IHlr : forall n : nat,
     is_redblack n lr
     -> is_redblack n r
       -> nearly_redblack n (append lr r)
         /\ (notred lr -> notred r -> is_redblack n (append lr r))
n : nat
H1 : is_redblack n (T R ll lx lr)
H2 : is_redblack n r
______________________________________(1/1)
nearly_redblack n (T R ll lx (append lr r))
\end{minted}

Podemos observar que se realiz\'o una simplificaci\'on de la meta, donde se desarroll\'o lo mas
posible la funci\'on append y se introdujeron los antecedentes a las hip\'otesis. Para poder
demostrar esta nueva meta tenemos que destruir la hipótesis `IHlr', lo cual nos introduciría los
dos antecedentes de la misma como metas. Se destruye esta hipótesis para poder obtener su 
consecuente como hip\'otesis.

\begin{minted}{coq}
r := T B rl rx rr : RB a
IHlr : forall n : nat,
     is_redblack n lr
     -> is_redblack n r
       -> nearly_redblack n (append lr r)
         /\ (notred lr -> notred r -> is_redblack n (append lr r))
IHrl : forall n : nat,
     is_redblack n (T R ll lx lr)
     -> is_redblack n rl
       -> nearly_redblack n (append (T R ll lx lr) rl)
         /\ (notred (T R ll lx lr)
            -> notred rl -> is_redblack n (append (T R ll lx lr) rl))
n : nat
H1 : is_redblack n (T R ll lx lr)
H2 : is_redblack n r
______________________________________(1/3)
is_redblack n lr
______________________________________(2/3)
is_redblack n r
______________________________________(3/3)
nearly_redblack n (T R ll lx (append lr r))
\end{minted}

Para poder demostrar los dos primeros casos basta con aplicar la definici\'on inductiva
\hyperref[inductive_isRedB]{$is\_redblack$}, lo cual nos introduce las hipótesis necesarias para 
poder cumplir las metas.
Para el \'ultimo caso nos basta de igual manera con aplicar la misma definici\'on inductiva a H1 y
a la meta aplicar las definiciones de \hyperref[inductive_isRedB]{$nearly\_redblack$} y 
\hyperref[inductive_isRedB]{$redred\_tree$} y esto nos da metas, que gracias a las nuevas 
hip\'otesis integradas por H1, se pueden probar sin mayor problema.

Con esto demostrado este caso est\'a completo.

\paragraph{Concatenaci\'on de un \'arbol negro y uno rojo.}

En este caso se invierten los colores con respecto al caso anterior; el \'arbol derecho es rojo y
el izquierdo es negro. Este caso, al igual que el pasado, requiere de una pequeña ayuda al
asistente de pruebas, la cual explicaremos m\'as adelante.

\begin{minted}{coq}
IHlr : forall (r : RB a) (n : nat),
     is_redblack n lr
     -> is_redblack n r
       -> nearly_redblack n (append lr r)
         /\ (notred lr -> notred r -> is_redblack n (append lr r))
IHrl : forall n : nat,
     is_redblack n (T B ll lx lr)
     -> is_redblack n rl
       -> nearly_redblack n (append (T B ll lx lr) rl)
         /\ (notred (T B ll lx lr)
            -> notred rl -> is_redblack n (append (T B ll lx lr) rl))
______________________________________(1/1)
forall n : nat,
  is_redblack n (T B ll lx lr)
  -> is_redblack n (T R rl rx rr)
    -> nearly_redblack n (append (T B ll lx lr) (T R rl rx rr))
      /\ (notred (T B ll lx lr)
         -> notred (T R rl rx rr)
           -> is_redblack n (append (T B ll lx lr) (T R rl rx rr)))
\end{minted}

En este caso al igual que los dos pasados, como uno\footnote{O los dos.} de los arboles es de
color rojo, la segunda parte de la conjunci\'on vuelve a ser una contradicci\'on, por la expresión
$notred$.

Entonces s\'olo nos quedamos con la primera mitad de la conjunci\'on:

\begin{minted}{coq}
l := T B ll lx lr : RB a
IHrl : forall n : nat,
     is_redblack n l
     -> is_redblack n rl
       -> nearly_redblack n (append l rl)
         /\ (notred l -> notred rl -> is_redblack n (append l rl))
n : nat
H1 : is_redblack n l
H2 : is_redblack n (T R rl rx rr)
______________________________________(1/2)
nearly_redblack n (T R (append l rl) rx rr)
\end{minted}

Podemos apreciar que este caso es el caso espejo del caso pasado, entonces el procedimiento a usar
para demostrar esta meta es el mismo, lo \'unico que cambia es cu\'ales hip\'otesis se usan para
lograr esto. En el caso pasado se destruy\'o la hip\'otesis IHlr, en este caso se usa su contraparte
IHrl y el resto de la demostraci\'on se sigue directamente de las nuevas metas introducidas y del
uso de las definiciones inductivas mencionadas en el caso pasado.

\paragraph{Concatenaci\'on de dos \'arboles negros.}

Este \'ultimo caso es el \'unico que no incluye una contradicci\'on ya que \'esta se
daba al tener un \'arbol rojo como uno de los dos \'arboles que se pasan a la funci\'on 
\hyperref[func_app]{$append$}, pero en este caso los dos \'arboles a concatenar son negros, entonces 
la conjunci\'on completa ser\'a probada.

\begin{minted}{coq}
IHlr : forall (r : RB a) (n : nat),
     is_redblack n lr
     -> is_redblack n r
       -> nearly_redblack n (append lr r)
         /\ (notred lr -> notred r -> is_redblack n (append lr r))
IHrl : forall n : nat,
     is_redblack n (T B ll lx lr)
     -> is_redblack n rl
       -> nearly_redblack n (append (T B ll lx lr) rl)
         /\ (notred (T B ll lx lr)
            -> notred rl -> is_redblack n (append (T B ll lx lr) rl))
______________________________________(1/1)
forall n : nat,
  is_redblack n (T B ll lx lr)
  -> is_redblack n (T B rl rx rr)
    -> nearly_redblack n (append (T B ll lx lr) (T B rl rx rr))
      /\ (notred (T B ll lx lr)
         -> notred (T B rl rx rr)
           -> is_redblack n (append (T B ll lx lr) (T B rl rx rr)))
\end{minted}

Esta demostraci\'on se inicia con una inducci\'on sobre la altura negra, es decir, una inducción 
sobre $n$. Esto porque hacer la inducción sobre esta propiedad nos garantiza que el resultado de la 
concatenación no estará desbalanceado.

Esta inducci\'on nos da el caso base con $n=0$
, seguido de la separaci\'on de la conjunci\'on y esto nos da 2 casos base, como se muestra en
seguida:

\begin{minted}{coq}
IHlr : forall n : nat,
     is_redblack n lr
     -> is_redblack n rl
       -> nearly_redblack n (append lr rl)
         /\ (notred lr -> notred rl -> is_redblack n (append lr rl))
IHrl : forall n : nat,
     is_redblack n (T B ll lx lr)
     -> is_redblack n rl
       -> nearly_redblack n (append (T B ll lx lr) rl)
         /\ (notred (T B ll lx lr)
            -> notred rl -> is_redblack n (append (T B ll lx lr) rl))
H1 : is_redblack 0 (T B ll lx lr)
H2 : is_redblack 0 (T B rl rx rr)
______________________________________(1/2)
nearly_redblack 0 (append (T B ll lx lr) (T B rl rx rr))
______________________________________(2/2)
notred (T B ll lx lr)
-> notred (T B rl rx rr)
  -> is_redblack 0 (append (T B ll lx lr) (T B rl rx rr))
\end{minted}

Estos casos base se resuelven aplicando las definiciones inductivas correspondientes tanto a las
metas como a las hip\'otesis H1 y H2, esto nos da las hipótesis necesarias para probar estos dos
casos.

Nos queda por probar el paso inductivo, en seguida podemos ver que la hipótesis de inducci\'on
`IH' es parte de IHlr, la cual se obtuvo de destruir esa hipótesis, en el siguiente paso se explica 
por qu\'e se decidió destruir esta hip\'otesis y no su contraparte IHrl.

\begin{minted}{coq}
IHlr : forall n : nat,
     is_redblack n lr
     -> is_redblack n rl
       -> nearly_redblack n (append lr rl)
         /\ (notred lr -> notred rl -> is_redblack n (append lr rl))
IHrl : forall n : nat,
     is_redblack n (T B ll lx lr)
     -> is_redblack n rl
       -> nearly_redblack n (append (T B ll lx lr) rl)
         /\ (notred (T B ll lx lr)
            -> notred rl -> is_redblack n (append (T B ll lx lr) rl))
n : nat
H1 : is_redblack (S n) (T B ll lx lr)
H2 : is_redblack (S n) (T B rl rx rr)
IH : nearly_redblack n (append lr rl)
______________________________________(1/1)
nearly_redblack (S n) (append (T B ll lx lr) (T B rl rx rr))
/\ (notred (T B ll lx lr)
   -> notred (T B rl rx rr)
     -> is_redblack (S n) (append (T B ll lx lr) (T B rl rx rr)))
\end{minted}

Proseguimos con la separaci\'on de la conjunci\'on, lo cual nos da dos casos que trabajaremos por
separado:

\paragraph{Primera mitad de conjunci\'on}

\begin{minted}{coq}
______________________________________(1/1)
nearly_redblack (S n) (append (T B ll lx lr) (T B rl rx rr))
\end{minted}

Después de simplificar la meta de este caso, nos queda una meta que depende del resultado de una
llamada recursiva a \hyperref[func_app]{$append$} de los subárboles $lr$ y $rl$, lo cual nos genera 
otros dos casos:

\begin{minted}{coq}
______________________________________(1/2)
is_redblack (S n) (lbalS ll lx (T B E rx rr))
______________________________________(2/2)
is_redblack (S n)
  match c with
  | R => T R (T B ll lx r1) a0 (T B r2 rx rr)
  | B => lbalS ll lx (T B (T c r1 a0 r2) rx rr)
  end
\end{minted}

Como podemos ver en ambas metas, tenemos una funci\'on nueva, \hyperref[lbalS]{$lbalS$}. Esta es una 
funci\'on de balanceo, la cual extiende a las funciones que ya se habían usado con anterioridad en la 
funci\'on de inserci\'on, como lo son \hyperref[rbal_2]{$rbal'$}, \hyperref[func_balanceo]{$rbal$} y 
\hyperref[func_balanceo]{$lbal$}.

Para poder resolver esta parte de la demostraci\'on nos apoyaremos de otro lema (figura
\ref{lema_5}), el cual ilustra una propiedad de la operaci\'on \hyperref[lbalS]{$lbalS$}.

\begin{figure}[!ht]
\centering
\captionsetup{justification=centering}
\begin{minted}{coq}
Lemma lbalS_rb {a} `{GHC.Base.Ord a}
(n : nat) (l : RB a) (x : a ) (r : RB a) :
nearly_redblack n l -> is_redblack (S n) r ->
              notred r -> is_redblack (S n) (lbalS l x r).
\end{minted}
\caption{Lema $lbalS\_rb$.}
\label{lema_5}
\end{figure}


Lo que el lema, arriba escrito en sintaxis de {\coq}, quiere decir es que si tenemos un par de
\'arboles, sean \textbf{\textit{l}} y \textbf{\textit{r}}, un n\'umero natural \textbf{\textit{n}} y
un elemento \textbf{\textit{x}}, si el \'arbol \textbf{\textit{l}} cumple con la
definici\'on inductiva \hyperref[inductive_isRedB]{$nearly\_redblack$} y \textbf{\textit{r}} no es de color rojo y 
cumple con la definici\'on inductiva \hyperref[inductive_isRedB]{$is\_redblack$}, entonces balancear 
estos dos arboles con \hyperref[lbalS]{$lbalS$} resulta en un {\arn} que cumple con la definici\'on 
$is\_redblack$.

La demostraci\'on de este lema se convierte en un análisis de casos en el cual solamente es
necesario simplificar, aplicar las definiciones inductivas y las metas que se generan son
consecuencias directas de las hipótesis generadas, la inducci\'on no es necesaria.

Regresando a las dos metas generadas por destruir la funci\'on \emph{append}, si nos fijamos en la primera,
podemos ver que se puede aplicar directamente el lema $lbalS_rb$, lo cual nos genera 3 nuevas 
metas:

\begin{minted}{coq}
H1 : is_redblack (S n) (T B ll lx lr)
H2 : is_redblack (S n) (T B rl rx rr)
IH : nearly_redblack n E
______________________________________(1/3)
nearly_redblack n ll
______________________________________(2/3)
is_redblack (S n) (T B E rx rr)
______________________________________(3/3)
notred (T B E rx rr)
\end{minted}

Estas metas de nuevo caen en el caso de simplificar y aplicar las respectivas definiciones
inductivas para obtener las metas deseadas, de esta manera el primer caso queda resuelto.

Ahora nos vamos al segundo caso generado al destruir la funci\'on \hyperref[func_app]{$append$} el 
cual nos dice que tenemos que hacer un análisis de casos sobre el color del nodo:

\begin{minted}{coq}
H1 : is_redblack (S n) (T B ll lx lr)
H2 : is_redblack (S n) (T B rl rx rr)
IH : nearly_redblack n (T R r1 a0 r2)
______________________________________(1/2)
is_redblack (S n) (T R (T B ll lx r1) a0 (T B r2 rx rr))
______________________________________(2/2)
is_redblack (S n) (lbalS ll lx (T B (T B r1 a0 r2) rx rr))
\end{minted}

Ese análisis de casos nos da dos metas nuevas, una por color. El primer caso solamente requiere
simplificaci\'on y aplicaci\'on de definiciones inductivas para obtener las metas deseadas. El
segundo caso sigue los mismos pasos con la única diferencia de volver a aplicar el lema 
\hyperref[lbalS]{$lbalS$}.

De esta manera queda demostrada la primera mitad de la conjunci\'on.

\paragraph{Segunda mitad de conjunci\'on}

\begin{minted}{coq}
______________________________________(1/1)
notred (T B ll lx lr)
-> notred (T B rl rx rr)
  -> is_redblack (S n) (append (T B ll lx lr) (T B rl rx rr))
\end{minted}

Esta segunda mitad sigue exactamente el mismo procedimiento antes descrito con la \'unica
diferencia de que se agregan hip\'otesis nuevas:

\begin{minted}{coq}
H1 : is_redblack (S n) (T B ll lx lr)
H2 : is_redblack (S n) (T B rl rx rr)
IH : nearly_redblack n (append lr rl)
H3 : notred (T B ll lx lr)
H4 : notred (T B rl rx rr)
______________________________________(1/1)
is_redblack (S n)
  match append lr rl with
  | T R lr' x rl' => T R (T B ll lx lr') x (T B rl' rx rr)
  | _ => lbalS ll lx (T B (append lr rl) rx rr)
  end
\end{minted}

Al hacer el análisis de casos destruyendo la funci\'on \hyperref[func_app]{$append$} con los 
par\'ametros $lr$ y $rl$, obtenemos exactamente las mismas metas que en la primera parte de la 
conjunci\'on y al tener mas hipótesis la demostraci\'on se acorta por un par de pasos pero el 
procedimiento es el mismo.

De esta manera queda demostrado el primer lema. A pesar de ser larga y
tediosa, se puede observar el poder del asistente de pruebas, ya que las demostraciones se reducen
a álgebra ecuacional, es decir, tratar de igualar la meta con lo que se tiene como hip\'otesis.
Esto se seguirá viendo en las siguientes pruebas.

\begin{figure}[!ht]
  \centering
  \captionsetup{justification=centering}
  \begin{minted}{coq}
  Lemma del_arb {a} `{GHC.Base.Ord a} (s:RB a) (x:a) (n:nat) :
        is_redblack (S n) s ->
        isblack s -> nearly_redblack n (del x s)
   with del_rb  {a} `{GHC.Base.Ord a} (s:RB a) (x:a) (n:nat) :
        is_redblack n s ->
        notblack s -> is_redblack n (del x s).
  \end{minted}
  \caption{Lema $del\_arb$}
  \label{lema_6}
  \end{figure}

\subsubsection{Segundo lema}
Este siguiente lema utiliza una palabra especial en el lenguaje de {\coq}, \emph{\textbf{with}}. Esta palabra
es un truco para demostrar dos lemas simultáneamente, el cual es el caso como se ve en la figura
\ref{lema_6}.

Como podemos ver, el lema en s\'i define dos lemas. Esto se hace de esta manera porque la
demostraci\'on de uno de estos lemas depende del otro. De esta manera podemos definir ambos lemas y
s\'olo usar una sola prueba.

\begin{minted}{coq}
  del_arb : forall (a : Type) (H : Base.Eq_ a) (H0 : Base.Ord a)
  (s : RB a) (x : a) (n : nat),
  is_redblack (S n) s ->
  isblack s -> nearly_redblack n (del x s)
  del_rb : forall (a : Type) (H : Base.Eq_ a) (H0 : Base.Ord a)
  (s : RB a) (x : a) (n : nat),
  is_redblack n s ->
  notblack s -> is_redblack n (del x s)
  ______________________________________(1/2)
  is_redblack (S n) s -> isblack s -> nearly_redblack n (del x s)
  ______________________________________(2/2)
  is_redblack n s -> notblack s -> is_redblack n (del x s)
\end{minted}

Lo que el asistente de pruebas est\'a haciendo es que nos est\'a integrando al ambiente de hip\'otesis
los dos lemas. De esta manera podemos realizar suposiciones con \'estos y as\'i ayudarnos a demostrar
los lemas. Realizaremos las pruebas de estos lemas por separado.

\paragraph{Prueba de \hyperref[lema_6]{$del\_arb$}}

\begin{minted}{coq}
______________________________________(1/1)
is_redblack (S n) s -> isblack s -> nearly_redblack n (del x s)
\end{minted}

Este lema enuncia lo siguiente: sea un \'arbol $s$, un elemento $x$ y un n\'umero natural $n$, si
$s$ cumple con la definicion inductiva \hyperref[inductive_isRedB]{$is\_redblack$} y $s$ es negro, 
entonces $s$ cumple con la definición de \hyperref[inductive_isRedB]{$nearly\_redblack$} después de 
eliminar el elemento $x$. En otras palabras, si tenemos un \'arbol con la raíz de color negro, el 
resultado de eliminar un elemento ser\'a un \'arbol casi rojinegro.

La prueba empieza con una inducci\'on sobre $s$ lo cual nos da las dos metas siguientes:
\begin{minted}{coq}
______________________________________(1/2)
forall n : nat, is_redblack (S n) E ->
                isblack E -> nearly_redblack n (del x E)
______________________________________(2/2)
forall n : nat,
  is_redblack (S n) (T c s1 a0 s2) ->
  isblack (T c s1 a0 s2) ->
  nearly_redblack n (del x (T c s1 a0 s2))
\end{minted}

A primera vista podemos apreciar que la primera meta contiene un antecedente falso ya que el \'arbol
vacío $E$ no puede ser negro, entonces esta meta es una contradicci\'on. La segunda meta podemos
ver que si analizamos los dos casos del color del \'arbol, el caso rojo es igualmente una
contradicci\'on por el mismo antecedente $isblack$. Esto s\'olo nos deja con el caso negro de la
segunda meta.

\begin{minted}{coq}
IHs1 : forall n : nat, is_redblack (S n) s1 ->
                       isblack s1 -> nearly_redblack n (del x s1)
IHs2 : forall n : nat, is_redblack (S n) s2 ->
                       isblack s2 -> nearly_redblack n (del x s2)
______________________________________(1/1)
forall n : nat,
is_redblack (S n) (T B s1 a0 s2)
-> isblack (T B s1 a0 s2) ->
nearly_redblack n (del x (T B s1 a0 s2))
\end{minted}

Después de introducir los antecedentes y simplificar la meta, se hace un análisis de casos sobre la
operaci\'on \hyperref[func_del]{$del$}, primero se ve el caso si el nodo a eliminar est\'a en el 
subárbol derecho y después en el izquierdo.

\begin{minted}{coq}
IHs1 : forall n : nat, is_redblack (S n) s1 ->
                       isblack s1 -> nearly_redblack n (del x s1)
IHs2 : forall n : nat, is_redblack (S n) s2 ->
                       isblack s2 -> nearly_redblack n (del x s2)
H1 : is_redblack (S n) (T B s1 a0 s2)
H2 : isblack (T B s1 a0 s2)
H6 : is_redblack n s1
H8 : is_redblack n s2
______________________________________(1/2)
nearly_redblack n
match s1 with
| T B _ _ _ => lbalS (del x s1) a0 s2
| _ => T R (del x s1) a0 s2
end
______________________________________(2/2)
nearly_redblack n
(if _GHC.Base.>_ x a0
 then
  match s2 with
  | T B _ _ _ => rbalS s1 a0 (del x s2)
  | _ => T R s1 a0 (del x s2)
  end
 else append s1 s2)
\end{minted}

Seguimos con el análisis de los dos casos:
\begin{itemize}
  \item En el primer caso seguimos con la destrucci\'on del \'arbol $s1$ para tener un análisis de
  casos, si simplificamos con las definiciones inductivas estas metas, eventualmente encontramos
  metas de las siguientes formas:
\begin{minted}{coq}
is_redblack n (del x E)

is_redblack n (del x (T R s1_1 a1 s1_2))
\end{minted}
  Estos casos son particulares del lema \hyperref[lema_6]{$del\_rb$}, para solucionar esto 
  usamos una táctica de {\coq} llamada $assert$, la cual nos deja agregar hip\'otesis, las cuales 
  después tendremos que demostrar, en este caso \'esta quedar'a demostrada al terminar de demostrar 
  todo este lema. Entonces como en este caso estamos destruyendo $s1$, la hip\'otesis a agregar 
  ser\'ia:
\begin{minted}{coq}
IHl' : forall n : nat, is_redblack n s1 ->
                  notblack s1 -> is_redblack n (del x s1)
\end{minted}
  Al agregarla al inicio de la prueba podemos solamente aplicarla cuando lleguemos a los casos
  arriba mencionados.
  \item El segundo caso se divide en dos, la primera parte es el caso espejo al pasado, se realiza 
  lo mismo pero para el \'arbol $s2$, lo cual nos da la siguiente hipótesis a agregar:
\begin{minted}{coq}
IHr' : forall n : nat, is_redblack n s2 ->
                  notblack s2 -> is_redblack n (del x s2)
\end{minted}
  Se aplica de la misma manera y llegamos a la segunda parte donde nos resulta la siguiente meta:
\begin{minted}{coq}
______________________________________(1/1)
nearly_redblack n (append s1 s2)
\end{minted}
  La cual es un caso particular del lema antes demostrado \hyperref[lema_4]{$append\_arb\_rb$}.
\end{itemize}

\paragraph{Prueba de \hyperref[lema_6]{$del\_rb$}}

\begin{minted}{coq}
______________________________________(1/1)
is_redblack n s -> notblack s -> is_redblack n (del x s)
\end{minted}

Este segundo lema enuncia lo siguiente: sea un \'arbol $s$, un elemento $x$ y un n\'umero natural
$n$, si $s$ cumple con la definicion inductiva \hyperref[inductive_isRedB]{$is\_redblack$} y $s$ no 
es negro, entonces $s$ cumple con la definición de \hyperref[inductive_isRedB]{$is\_redblack$} 
después de eliminar el elemento $x$. En otras palabras, si tenemos un \'arbol con la raíz de color 
rojo, el resultado de eliminar un elemento ser\'a un \'arbol casi rojinegro.

El enunciado con respecto al anterior busca que el resultado sea mas especifico, pues la propiedad
de ser \hyperref[inductive_isRedB]{$is\_redblack$} es la que buscamos que las operaciones cumplan. 
Sin embargo, la demostraci\'on en términos de {\coq} no es muy distinta; iniciamos con inducci\'on 
sobre s.

\begin{minted}{coq}
______________________________________(1/2)
forall n : nat, is_redblack n E ->
           notblack E ->
           is_redblack n (del x E)
______________________________________(2/2)
forall n : nat, is_redblack n (T c s1 a0 s2) ->
           notblack (T c s1 a0 s2) ->
           is_redblack n (del x (T c s1 a0 s2))
\end{minted}

La primera meta se soluciona simplificando hasta obtener la meta deseada, mientras que la segunda
meta se le hace un análisis sobre el color $c$, el cual arroja dos casos, rojo y negro. El caso de
que $c$ sea negro es una contradicci\'on porque no cumple con la meta de ser $notblack$, solamente
nos enfocaremos en el caso en que $c$ es rojo.

\begin{minted}{coq}
IHs1 : forall n : nat,
         is_redblack n s1 ->
         notblack s1 -> is_redblack n (del x s1)
IHs2 : forall n : nat,
         is_redblack n s2 ->
         notblack s2 -> is_redblack n (del x s2)
H1 : is_redblack n (T R s1 a0 s2)
H2 : notblack (T R s1 a0 s2)
______________________________________(1/1)
is_redblack n (del x (T R s1 a0 s2))
\end{minted}

Al igual que en \hyperref[lema_6]{$del\_arb$} se hacen los casos de si el elemento a eliminar est\'a 
en el subárbol derecho o izquierdo. Otra similitud que esta prueba tiene con respecto con la pasada 
es que también tenemos que agregar una hipótesis extra, en este caso de 
\hyperref[lema_6]{$del\_arb$}. De aquí en adelante la prueba es muy similar a la anterior, 
simplificar, aplicar definiciones inductivas hasta llegar a contradicciones o a las metas deseadas.

En este lema se usan lemas auxiliares muy similares a $lbalS\_rb$\footnote{Descrito en la prueba de
la  funci\'on \hyperref[func_app]{$append$}.}, como su espejo $rbalS\_rb$ o sus contrapartes 
$rbalS\_arb$ y $lbalS\_arb$. Estos son lemas de balanceo sencillos de demostrar pero muy largos, 
tediosos y repetitivos, por lo tanto no se incluirán en este trabajo\footnote{Se pueden consultar 
en: \url{https://github.com/DavidFHCh/Tesis-FTW}.}.

Hasta este momento s\'olo se han demostrado partes de la operación total de 
eliminación, como unir
dos subárboles despu\'es de eliminar su ra\'iz, qu\'e sucede si eliminamos de 
un árbol con raíz roja o
de uno con raíz negra. En seguida uniremos todos estos lemas en uno.

\subsubsection{Instancia de la funci\'on de eliminaci\'on}

\begin{figure}[!ht]
\centering
\captionsetup{justification=centering}
\begin{minted}{coq}
Instance remove_rb s x : redblack s -> redblack (remove x s).
\end{minted}
\caption{Instancia de eliminaci\'on de la clase \hyperref[class_rb]{$redblack$}.}
\label{instance_del}
\end{figure}


Al igual que en la funci\'on de inserción, terminamos la operación de eliminaci\'on generando una
instancia de la clase \hyperref[class_rb]{$redblack$} (figura \ref{instance_del}). Al igual que en la 
operación opuesta, requerimos de un lema auxiliar con respecto a la clase 
\hyperref[class_rb]{$redblack$} (figura \ref{lema_7}).

\begin{figure}[!ht]
\centering
\captionsetup{justification=centering}
\begin{minted}{coq}
Lemma makeBlack_rb {a} `{GHC.Base.Ord a} n t :
nearly_redblack n t -> redblack (makeBlack t).
\end{minted}
\caption{Lema $makeBlack\_rb$.}
\label{lema_7}
\end{figure}


Lo que este enunciado describe es la propiedad de que si un \'arbol $t$ cumple con la definici\'on
inductiva de ser \hyperref[inductive_isRedB]{$nearly\_redblack$}, pintar su raíz de color negro lo 
convierte en una instancia de la clase \hyperref[class_rb]{$redblack$}. La 
demostraci\'on de este 
lema es muy simple gracias al asistente de pruebas, ya que s\'olo basta con hacer un análisis de casos 
sobre el \'arbol $t$:

\begin{itemize}
  \item El \'arbol vac\'io $E$, la meta a demostrar para este caso es:
\begin{minted}{coq}
nearly_redblack n E -> redblack (makeBlack E)
\end{minted}
        Como la clase \hyperref[class_rb]{$redblack$} esconde un existencial en su definici\'on, 
        para poder demostrar este caso basta con decir que existe $n$ con valor 0, esto nos da un 
        caso trivial al ser la misma definici\'on inductiva de 
        \hyperref[inductive_isRedB]{$is\_redblack$}.
  \item El segundo caso se reduce a los dos casos en los que puede caer la definici\'on inductiva
  de \hyperref[inductive_isRedB]{$nearly\_redblack$}.
\begin{minted}{coq}
H1 : nearly_redblack n (T c t1 a0 t2)
H2 : is_redblack n (T c t1 a0 t2)
______________________________________(1/2)
redblack (makeBlack (T c t1 a0 t2))
\end{minted}
        Este primer caso se reduce a hacer un análisis de casos sobre el color $c$, la soluci\'on
        de ambos colores consiste en, decir que existe $n'$ tal que su valor es \textit{S(n)},
        después de esto se simplifican las expresiones hasta obtener que las metas cumplan con las
        hip\'otesis.
\begin{minted}{coq}
H1 : nearly_redblack n (T c t1 a0 t2)
H2 : redred_tree n (T c t1 a0 t2)
______________________________________(2/2)
redblack (makeBlack (T c t1 a0 t2))
\end{minted}
        El segundo caso es m\'as corto que el primero, ya que al hacer el análisis de los colores,
        podemos ver que la definición de \hyperref[inductive_isRedB]{$redred\_tree$} no est\'a 
        definida para \'arboles negros, entonces s\'olo nos queda demostrar para \'arboles rojos. Sin 
        embargo, los pasos a seguir para este caso son los mismos que para el color rojo del caso 
        anterior.
\end{itemize}

Con este lema demostrado ya contamos con todas las herramientas para poder 
demostrar que si tenemos a 
un \'arbol que es instancia de la clase \hyperref[class_rb]{$redblack$} y eliminamos un elemento de 
\'el, el \'arbol resultante sigue siendo instancia de la clase, esta 
demostraci\'on comienza con un análisis de casos 
sobre el \'arbol $s$:

\begin{minted}{coq}
H1 : is_redblack n E
______________________________________(1/2)
redblack (makeBlack (del x E))
______________________________________(2/2)
redblack (makeBlack (del x (T c s1 c0 s2)))
\end{minted}

Podemos ver que se hace uso de la funci\'on \hyperref[raiz_negra_func]{$makeBlack$}. En la primera 
meta basta con aplicar el lema $makeBlack\_rb$, simplificar y \'esta se soluciona. En la segunda meta 
se tiene que hacer otro an\'alisis de casos, esta vez sobre el color:

\begin{minted}{coq}
H1 : is_redblack n (T R s1 c0 s2)
______________________________________(1/2)
redblack (makeBlack (del x (T R s1 c0 s2)))

\end{minted}

En la primera meta, color rojo, comenzamos por aplicar $makeBlack\_rb$, el cual despu\'es de
simplificar con la definici\'on inductiva nos resulta en la siguiente meta:

\begin{minted}{coq}
H1 : is_redblack n (T R s1 c0 s2)
______________________________________(1/1)
is_redblack n (del x (T R s1 c0 s2))
\end{minted}

La cual es un caso particular del lema \hyperref[lema_6]{$del\_rb$}, nos basta con aplicarlo, 
simplificar y esta meta queda resuelta. La \'unica meta que nos quedaría por demostrar ser\'ia el caso 
de la raíz negra:

\begin{minted}{coq}
H1 : is_redblack n (T B s1 c0 s2)
______________________________________(1/1)
redblack (makeBlack (del x (T B s1 c0 s2)))
\end{minted}

En este caso hacemos un análisis sobre $n$, los dos casos ser\'ian 0 y \textit{S(n)}. Para 0 basta
con simplificar y la meta se resuelve, pero para \textit{S(n)}, es necesario volver a aplicar el
lema $makeBlack\_rb$, una vez que hacemos esto, nos queda la siguiente meta:

\begin{minted}{coq}
H1 : is_redblack (S n) (T B s1 c0 s2)
______________________________________(1/1)
nearly_redblack n (del x (T B s1 c0 s2))
\end{minted}

La cual resulta ser un caso particular del lema \hyperref[lema_6]{$del\_arb$}, al aplicar este lema 
y simplificar nuevamente, las metas resultantes quedan resueltas. Podemos ver que las demostraciones 
cubiertas en este trabajo, en especial en la operaci\'on de eliminación, son muy repetitivas y s\'olo 
buscamos hacer que las metas empaten con las hip\'otesis que tenemos.

Con esta operaci\'on demostrada podemos decir que tenemos una estructura correcta respecto
a los invariantes descritos por las definiciones inductivas de 
\hyperref[inductive_isRedB]{$is\_redblack$} con las operaciones de borrado e inserción y de la misma 
manera que estas dos operaciones son métodos de la clase \hyperref[class_rb]{$redblack$}, por lo 
cual podemos hacer estas operaciones cuantas veces queramos y el resultado seguirá siendo instancia 
de esta clase.

\chapter{Conclusiones}
Como se ha ilustrado a lo largo de este trabajo, lo que buscamos es otra manera de demostrar la correcci\'on de una estructura de datos,
en este caso de un {\arn} usando un asistente de pruebas como lo es {\coq} con una biblioteca de
tipos y funciones que se tradujeron de Haskell.

Hemos mencionado en repetidas ocasiones que la opci\'on mas tradicional para realizar una prueba de este estilo seria
usar lápiz y papel, pero como se ha visto en capítulos anteriores el desarrollo de la prueba puede llegar a generar demasiados
casos, esto lo convierte en una tarea muy complicada y tediosa de escribir, y posteriormente de leer y
entender por alguien mas. En cambio un asistente de pruebas como lo es {\coq} da herramientas para
simplificar esta tarea y logra reducirla a álgebra ecuacional, ya que como se vio en este trabajo,
lo \'unico que se busca obtener es que las metas que queremos probar se igualen con alguna de las
hip\'otesis que se tienen, lo cual también tiene sus detrimentos ya que se deja de razonar de manera formal.

Sin embargo, el uso de una herramienta de esta naturaleza por si sola no simplifica del todo este
tipo de pruebas, ya que para poder llegar a un escenario donde se pueda desarrollar una demostraci\'on primero
tenemos que tener claro que es lo que se quiere probar y codificarlo en el lenguaje que la
herramienta comprenda.

En la vida real, esto significaría tener un programa escrito en algún lenguaje de programaci\'on
como Java, Python, Haskell, etc. y traducirlo al lenguaje de la herramienta. Esto requeriría la
implentaci\'on de un traductor o en su defecto traducir los programas a mano, esta segunda opci\'on
siendo una soluci\'on no \'optima ya que es muy susceptible a errores. En este trabajo se uso el
traductor de Haskell a Coq llamado `hs-to-coq' \cite{thrc}, que aunque nos dio algunas bibliotecas
de Haskell traducidas a Coq, esta sigue en estado de desarrollo y aunque Haskell comparte el mismo
paradigma que el lenguaje de Coq, lograr traducir en un $100\%$ un lenguaje resulta muy complicado
ya que este siempre esta evolucionando, en especial si es un lenguaje tan ampliamente usado como
lo es Haskell.

Otra restricci\'on que se tiene que establecer es que no todos los programas escritos en Haskell podrían ser traducidos
al lenguaje de Coq, este lenguaje a pesar de que entra en la categoría de lenguajes funcionales, este solo acepta funciones totales. Entonces esto introduce otras problemáticas,
la traducción un programa de un paradigma imperativo, l\'ogico, etc. a uno funcional y después
asegurar que todas las funciones de este son totales.

Supongamos que resolvemos todos estos problemas que se han presentado hasta ahora, es decir,
tenemos un programa donde todas sus funciones son totales y se logro traducir correcta y
completamente. Ahora se tienen que generar las definiciones inductivas, las cuales te ayudaran a
guardar invariantes de tu programa, y con estas escribir los lemas que se buscan probar para poder
decir que tu programa ha sido verificado formalmente, lo cual podría tomar el mismo tiempo que tomo
traducir todo el programa al lenguaje de la herramienta.

Actualmente en la industria lo que se hace para minimizar los errores en c\'odigo, es hacer que este pase por una serie de filtros, es decir, que otra persona revise tu c\'odigo para ayudarte a encontrar defectos, también en ejecutar pruebas ya existentes para asegurar que el nuevo c\'odigo no introduzca errores a componentes que funcionaban correctamente dentro del programa
y que se escriban pruebas que confirmen el correcto funcionamiento del c\'odigo a introducir.
En este momento la idea de poder probar
que un programa cualquiera puede ser probado formalmente usando un asistente de pruebas es muy
atractiva, ya que un \'unico desarrollador podr\'ia desarrollar la prueba y no depender de código ajeno que muestre que su programa es correcto. Sin embargo, esta idea resulta muy poco factible hoy en día, ya que además de los problemas expuestos con
anterioridad (las traducciones del c\'odigo implementado) se le suma el hecho de que se tendrían que traducir y probar todas las
bibliotecas ocupadas del lenguaje que se esta usando, esto antes de pensar en probar tu programa.

Otro acercamiento para poder probar este tipo de programas en la industria seria desarrollar la
mayor parte de estos en el asistente {\coq}, realizando esto con las herramientas que su lenguaje nos provee, de esta manera se pueden realizar
las demostraciones pertinentes y utilizar la funcionalidad que este posee para extraer c\'odigo en
otros lenguajes, después de haber realizado la prueba, como lo son Haskell y O'Caml. Sin embargo, esto solo nos permitiría desarrollar
programas correctos con las funcionalidades que el lenguaje de Coq nos ofrezca.

Retomando el punto anterior, otra soluci\'on seria desarrollar y probar partes clave de los programas a crear, es decir, 
m\'odulos pequeños como lo serian las estructuras de datos a usar, como lo podrian ser los {\arns}, listas doblemente 
ligadas, colas, pilas u otros tipos de \'arboles. Una vez implementados estos m\'odulos se podrían usar en cualquier 
parte de c\'odigo, el problema con hacer esto es que dependiendo del lenguaje al que se extraiga el programa 
probado, puede ser contraproducente para el desempeño del mismo. Este degradado en el desempeño se puede dar por razones
ajenas al c\'odigo y mas por asuntos relacionados a la implementaci\'on del compilador que se usar\'a para generar 
c\'odigo ejecutable y que tan optimizado es el mismo. Por ejemplo, el lenguaje C es conocido por ser muy 
usado en programas que requieren ser muy rápidos en sus operaciones.

Otro problema es que como no todo el co\'digo estaría probado formalmente, para componentes mas grandes se necesitaría caer 
nuevamente en hacer otro tipo de pruebas, como las unitarias, y como ya hemos mencionado, estas no nos aseguran que los 
programas sean correctos o completos y por lo tanto, nuestros programas solo estarían parcialmente verificados formalmente.

Como podemos apreciar el demostrar programas con un asistente de pruebas, no es el procedimiento mas amigable hoy en día, 
sin embargo, si se sigue con la actual trayectoria en el desarrollo de herramientas de traducci\'on como lo es `hs-to-coq', eventualmente la industria podría comenzar a utilizar herramientas de este estilo para mejorar la calidad de sus productos. Mientras tanto, se tendrán que seguir desarrollando pruebas unitarias de mejor calidad y lograr generar programas que tiendan a la correcci\'on y completud.

\appendix
\chapter{Conceptos básicos de {\sc Coq}}

Para una mejor comprensi\'on de las pruebas realizadas en este trabajo, daremos una breve explicación
del asistente de pruebas {\coq}. Nos enfocaremos solamente en las características de \textit{Coq} que se usaron en este trabajo. Este apéndice está basado en el manual del asistente que el lector puede consultar en \cite{coqdefs}.

Lo primero que se hace al querer probar un programa en \textit{Coq} es escribirlo en su propio lenguaje llamado Gallina. Este
contiene varias palabras reservadas para el desarrollo de funciones y de tipos especiales para el uso
de las mismas, en seguida explicaremos las que se usaron:

\begin{itemize}
  \item \textbf{\textit{Inductive}} - Esta palabra reservada se refiere a un tipo al cual se le puede aplicar inducc\'on
    estructural. Al usar esta palabra para definir a un tipo, se establecen los constructores, y en algunos casos,
    destructores. En el listado \ref{ind_example} presentamos un ejemplo de uso, definiendo los 
    n\'umeros naturales.
    \begin{listing}[!ht]
    \centering
    \captionsetup{justification=centering}
    \begin{minted}{coq}
      Inductive nat : Set :=
      | O : nat
      | S : nat -> nat. 
    \end{minted}
    \caption{Tipo Inductivo.}
    \label{ind_example}
    \end{listing}
  \item \textbf{\textit{Fixpoint}} - Esta palabra reservada se refiere a la definici\'on de una funci\'on recursiva, aplicada a, al menos, 
    un tipo inductivo definido con la palabra reservada \textbf{\textit{Inductive}}. El listado \ref{fix_example} presentamos una funci\'on recursiva usando el tipo definido arriba. 
    \begin{listing}[!ht]
      \centering
      \captionsetup{justification=centering}
      \begin{minted}{coq}
        Fixpoint add (n m:nat) {struct n} : nat :=
        match n with
        | O => m
        | S p => S (add p m)
        end. 
      \end{minted}
      \caption{Funci\'on \textit{Fixpoint}.}
      \label{fix_example}
      \end{listing}
  \item \textbf{\textit{Definition}} - Esta palabra reservada b\'asicamente se refiere a una sustituci\'on textual de 
    una expresión por otra. En el listado \ref{def_example} se hace una sustituci\'on textual para aumentar en uno el valor de $n$.
    \begin{listing}[!ht]
      \centering
      \captionsetup{justification=centering}
      \begin{minted}{coq}
        Definition add_one (n :nat) {struct n} : nat :=
        match n with
        | O => S (O)
        | S p => S (S (p))
        end. 
      \end{minted}
      \caption{Sustituci\'on textual con \textit{Definition}.}
      \label{def_example}
      \end{listing}
\end{itemize}


En los mecanismos de definición anteriores se hace uso de la coincidencia de patrones o pattern matching. Los tipos inductivos tambien los usamos para definir las invariantes o especificación formal del programa 
que se busca verificar. En el c\'odigo \hyperref[inductive_isRB]{$isRB$} definimos la especificación 
de los {\arns}.

Una vez escrito el programa y su especificación, procedemos a escribir lemas que nos ayuden a verificar
la correcci\'on con respecto a la especificación dada. Esto lo logramos con la palabra reservada 
\textbf{\textit{Lemma}}\footnote{Además de esta palabra también se puede usar Theorem, Proposition, Corolary, para indicar una proposición.} (ver \hyperref[lema_1]{\textit{ins\_rr\_rb}}). Al usar esta palabra \textit{Coq} entra en modo 
prueba, con lo cual se pueden usar las \textit{t\'acticas}.

Las t\'acticas (ver listado \ref{tacticas_example}) son palabras reservadas como: \textit{intros}, \textit{destruct}, \textit{apply}, \textit{constructor}, etc. Estas 
t\'acticas son funciones que nos ayudan a transformar el estado actual de una prueba, denominado \textit{meta}. Una
sucesión de t\'acticas eventualmente generan una prueba completa, a cual opcionalmente empieza con la palabra \textit{Proof} y siempre termina con \textit{Qed}.

\begin{listing}[!ht]
\centering
\captionsetup{justification=centering}
\begin{minted}{coq}
(* lema que prueba que un arbol con al menos un elemento no es vacio.
 *)
Lemma T_neq_E {a} `{GHC.Base.Ord a}:
  forall (c:Color) (l: RB a) (k: a) (r: RB a), T c l k r <> E.
Proof.
intros. intro Hx. inversion Hx.
Qed.
\end{minted}
\caption{Ejemplo del uso de t\'acticas y lemas.}
\label{tacticas_example}
\end{listing}

Las metas (ver listado \ref{meta_example}) consisten de dos partes: la conclusi\'on y un contexto. El segundo contiene las hip\'otesis, definiciones y variables que podemos 
usar para demostrar la conclusi\'on con el uso de las t\'acticas. 
\begin{listing}[!ht]
\centering
\captionsetup{justification=centering}
\begin{minted}{coq}
H1_ : is_redblack n l
H1_0 : is_redblack n r
IHis_redblack1 :
    ifred l (redred_tree n (ins x l)) (is_redblack n (ins x l))
IHis_redblack2 :
    ifred r (redred_tree n (ins x r)) (is_redblack n (ins x r))
______________________________________(1/2)
is_redblack (S n) (lbal (ins x l) k r)
______________________________________(2/2)
is_redblack (S n) (rbal l k (ins x r))
\end{minted}
\caption{Ejemplo de una \textit{meta}.}
\label{meta_example}
\end{listing}
En este trabajo se habla en alto nivel como se usaron las t\'acticas para verificar los {\arns}.






%aprendiendo a usar esta cosa, it is cool-aid
%\bibliographystyle{plain} % We choose the "plain" reference style
%\bibliography{refs} % Entries are in the "refs.bib" file
\bibliographystyle{apalike}
\bibliography{refs.bib}
\backmatter%@sglvgdor


\end{document}
