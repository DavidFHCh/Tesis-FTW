\chapter{Implementación de arboles roji-negros en {\coq}}

\section{Traducción de implementaciones}
Se tuvieron varias aproximaciones para la implementación de {\arns}; la primera fue obtener las
implementaciones en Haskell de estos\cite{tesisG} y pasar estas implementaciones como entrada a
hs-to-coq, la segunda aproximación, y la que se uso para esta tesis, fue obtener la implementación
de Okasaki traducida a Coq  \cite{MSetRBT}, con la diferencia de que se usaron las bibliotecas de
Haskell que contienen los tipos y comparaciones, estas obtenidas de la traducción de estas
bibliotecas a Coq, en lugar de usar las bibliotecas ya existentes de Coq.

A continuación profundizaremos acerca de estos dos métodos.

\subsection{Traducción directa de implementaciones de Haskell a Coq}
De un trabajo anterior\cite{tesisG} se obtuvieron diversas implementaciones de {\arns} la mayor
parte de estas cambiaba principalmente en las implementaciones de las operaciones de borrado,
siendo esta la operación mas compleja.
Estas diversas implementaciones van desde la implementación de Okasaki\footnote{siendo esta la
m\'as simple}, constructores inteligentes\footnote{implementaci\'on anterior con optimizaciones} y
tipos anidados\footnote{una implementaci\'on totalmente diferente a las anteriores}.

Por la compleja naturaleza de estas implementaciones\footnote{incluso Okasaki} la traducción
manual del código de Haskell resulto ser muy problemática, esto porque las implementaciones en
Haskell se aprovechan del hecho de que en este lenguaje se pueden dar funciones parciales, lo cual
resulta problemático cuando se quiere traducir a Coq, ya que este lenguaje únicamente acepta
funciones totales.

Por lo expuesto anteriormente se opto por usar la herramienta \textbf{hs-to-coq} para facilitar la
traducci\'on, pero por las mismas razones antes descritas\footnote{las funciones no eran totales},
la herramienta caia en alguna de estas dos situaciones:

\begin{itemize}
    \item El tiempo de ejecuci\'on de la herramienta era muy alto y eventualmente los recursos de
    la maquina virtual, donde esta herramienta se ejecuto, se quedaba sin recursos\footnote{en
    especial memoria}.
    \item La herramienta generaba c\'odigo en Coq pero con elementos de Haskell cuyas bibliotecas
    todavía no habían sido traducidas del todo.
\end{itemize}{}

Se busco otra acercamiento para poder verificar esta estructura; la herramienta hs-to-coq tiene
parte de las bibliotecas de Haskell traducidas a Coq, entonces se uso la implementación de {\arns}
de las bibliotecas de {\coq}\cite{MSetRBT} pero usando los tipos y comparaciones de tipos de
Haskell traducidos a Coq.

\section{Inserción de elementos en un {\arn}}

La inserci\'on de elementos a un {\arn} es la operaci\'on mas sencilla de de las dos operaciones
que se verificaran en este trabajo. La idea principal detrás de este algoritmo es que solo se
agreguen hojas al \'arbol binario y se efectúen ``giros'' para mantener las invariantes de la
estructura, ver figura 2.1 y 2.2.
\begin{figure}
\centering
\captionsetup{justification=centering}
\label{arbolRB}
 \begin{tikzpicture}[-,level/.style={sibling distance = 4cm/#1,level distance = 1cm}]
%\draw[style=help lines] (-5,-7) grid (5,0);
\node [arn_n] at (-2,0) {6}
        child{ node [arn_r] {2}
            child{ node [arn_n] {1}
                child{ node [arn_x] {}} %for a named pointer
                child{ node [arn_x] {}}
            }
            child{ node [arn_n] {4}
                child{ node [arn_r] {3}
                    child{ node [arn_x] {}}
                    child{ node [arn_x] {}}
                }
                child{ node [arn_r] {5}
                    child{ node [arn_x] {}}
                    child{ node [arn_x] {}}
                }
            }
        }
        child{ node [arn_n] {8}
            child{ node [arn_x] {}}
            child{ node [arn_r] {9}
                child{ node [arn_x] {}}
                child{ node [arn_x] {}}
            }
        }
;

{
}
;
\end{tikzpicture}
\caption{{\Arn} antes de insertar nodo 7.}
\end{figure}

\begin{figure}
\centering
\captionsetup{justification=centering}
\label{arbolRB}
 \begin{tikzpicture}[-,level/.style={sibling distance = 4cm/#1,level distance = 1cm}]
%\draw[style=help lines] (-5,-7) grid (5,0);
\node [arn_n] at (-2,0) {6}
        child{ node [arn_r] {2}
            child{ node [arn_n] {1}
                child{ node [arn_x] {}} %for a named pointer
                child{ node [arn_x] {}}
            }
            child{ node [arn_n] {4}
                child{ node [arn_r] {3}
                    child{ node [arn_x] {}}
                    child{ node [arn_x] {}}
                }
                child{ node [arn_r] {5}
                    child{ node [arn_x] {}}
                    child{ node [arn_x] {}}
                }
            }
        }
        child{ node [arn_n] {8}
            child{ node [arn_r] {7}
                child{ node [arn_x] {}}
                child{ node [arn_x] {}}
            }
            child{ node [arn_r] {9}
                child{ node [arn_x] {}}
                child{ node [arn_x] {}}
            }
        }
;

{
}
;
\end{tikzpicture}
\caption{{\Arn} después de insertar nodo 7.}
\end{figure}
\subsection{Operaciones de Balanceo}
Los ``giros'' antes mencionados están definidos en las operaciones de balanceo, se tienen dos, una
operaci\'on para los subárboles izquierdos y otra para los derechos. Estas operaciones de
balanceo, c\'odigo en figura 2.3, se encargan de solucionar los casos en los que inmediatamente
después de agregar una hoja los invariantes son violados, por ejemplo, dos nodos rojos que
resultan contiguos en algún lugar de la estructura del \'arbol. Mas a\'un solo dos de los tres
invariantes pueden ser violados cuando se agrega un nuevo nodo, siendo estos que hayan dos nodos
rojos contiguos o que la raíz del árbol resulte roja\footnote{esto no se arregla con giros, si no,
pintando la raíz de negro después de insertar}.

El giro elimina el doble nodo rojo, al crear solo un nodo rojo con dos hijos negros, de igual
manera al tener dos hijos negros nos aseguramos que el árbol crece de manera controlada en
n\'umero de nodos negros\footnote{este n\'umero de nodos negros se conoce como altura negra}, ya
que en ningún momento se están agregando dos nodos negros contiguos. Cabe mencionar que esta es la
única operación en donde se agregan nodos negros con la excepción de $makeBlack$, la cual
describiremos mas adelante.
\begin{figure}
\centering
\captionsetup{justification=centering}
\begin{minted}{coq}
Definition lbal {a} `{GHC.Base.Ord a} (l:RB a) (k:a) (r:RB a) :=
 match l with
 | T R (T R a x b) y c => T R (T B a x b) y (T B c k r)
 | T R a x (T R b y c) => T R (T B a x b) y (T B c k r)
 | _ => T B l k r
 end.

 Definition rbal {a} `{GHC.Base.Ord a} (l:RB a) (k:a) (r:RB a) :=
 match r with
 | T R (T R b y c) z d => T R (T B l k b) y (T B c z d)
 | T R b y (T R c z d) => T R (T B l k b) y (T B c z d)
 | _ => T B l k r
 end.
\end{minted}
\caption{Funciones de Balanceo.}
\end{figure}

Mas adelante se explicar\'a porque solamente nos concentramos en estos cuatro casos y porque la
mayor parte de los casos se reacomodan en un \'arbol con ra\'iz negra.

\subsection {Funci\'on de inserci\'on}
Con esta funci\'on es donde por primera vez podemos apreciar el uso de las  bibliotecas de
Haskell, podemos apreciar como los tipos que estamos agregando al \'arbol son tipos ordenados de
la biblioteca $Base$ de \textbf{GHC} y por esa misma raz\'on estamos usando las comparaciones de
esa biblioteca.
\begin{figure}
\centering
\captionsetup{justification=centering}
\begin{minted}{coq}
Fixpoint ins {a} `{GHC.Base.Ord a} (x:a) (s:RB a) :=
 match s with
 | E => T R E x E
 | T c l y r =>
    if x GHC.Base.< y : bool then
      match c with
       | R => T R (ins x l) y r
       | B => lbal (ins x l) y r
      end
    else
    if x GHC.Base.> y : bool then
      match c with
       | R => T R l y (ins x r)
       | B => rbal l y (ins x r)
      end
    else s
 end.
\end{minted}
\caption{Funci\'on ins.}
\end{figure}

Leyendo mas detenidamente la funci\'on, figura 2.4, se puede observar que las operaciones de
balanceo solo se efectúan cuando el nodo por el que se esta pasando es negro, esto sucede por la
raz\'on de que estos son los nodos  que se toman en cuenta para decidir si un \'arbol esta
balanceado. Al aplicar las operaciones de balanceo en estos nodos, nos aseguramos de no quedar con
nodos negros extras en alguno de los hijos del nodo al que se le esta aplicando la operación de
balanceo. Esto se puede apreciar si nos regresamos a las definiciones de las operaciones de
balanceo, tomemos $rbal$\footnote{Con $lbal$ la idea es an\'aloga}, tenemos dos casos:

\begin{itemize}
    \item Sean $x$, $y$ y $z$ nodos del \'arbol y sea $n$ un subárbol, $x$ es el nodo al que se le
    aplica la operaci\'on de balanceo, $n$ es el subárbol izquierdo, $y$ es el nodo derecho  de $x$
    y $z$ es hijo de $y$. Suponiendo que $y$ y $z$ son rojos\footnote{se viola una invariante, dos
    nodos rojos contiguos}, se cae en cualquiera de los dos casos de $rbal$ que no son el caso
    general. En este momento es donde se efectúa el ``giro'' del árbol y resulta lo siguiente: $x$
    se vuelve el hijo izquierdo de $y$ y $z$ se pinta de negro.

    En el momento en que $x$ se convierte en hijo izquierdo de $y$ el \'arbol se desbalancea, es
    por estoque se pinta de negro a $z$, así los dos nodos negros son hijos de $y$ y la invariante
    se conserva.
    \item En cualquier otro caso el \'arbol no sufre modificaci\'on alguna.
\end{itemize}{}

Esta función de inserción puede no cumplir con una de las invariantes, que la raíz del árbol sea
negra, es por esto que se introducen las definiciones de la figura 2.5.

\begin{figure}
\centering
\captionsetup{justification=centering}
\begin{minted}{coq}
Definition makeBlack {a} `{GHC.Base.Ord a} (t:RB a) :=
 match t with
 | E => E
 | T _ a x b => T B a x b
 end.

Definition insert {a} `{GHC.Base.Ord a} (x:a) (s:RB a) := makeBlack (ins x s).
\end{minted}
\caption{Definiciones para pintar ra\'iz de negro.}
\end{figure}

La definici\'on $makeBlack$ únicamente colorea cualquier nodo de color negro y la definición
$insert$ es una envoltura de $ins$, con la cual nos aseguramos de que la ra\'iz de los arboles
siempre sea de color negro.

Estas funciones y definiciones son suficientes para poder construir {\arns} que respeten las
invariantes que planteamos en la definici\'on 1.2.1.

\section{Eliminación de elementos en un {\arn}}

Como se menciono en la secci\'on pasada la operaci\'on de eliminaci\'on es significativamente mas
compleja que su contra parte, esto se debe al hecho de que se pueden eliminar nodos de cualquier
parte del \'arbol, mientras que en la inserci\'on solo se agregaban hojas de color rojo, es decir,
la altura no se modifica en la inserción\cite{RBTypes}, solo hasta que se aplica el balanceo.

El poder eliminar nodos de cualquier parte de un \'arbol presenta una problemática muy grande para
el balanceo del mismo, esto porque al eliminar un nodo del \'arbol, los dos subárboles de este
tienen que concatenarse de alguna manera y las invariantes de los {\arns} tienen que respetarse.

\begin{figure}
\centering
\captionsetup{justification=centering}
\label{arbolRB}
 \begin{tikzpicture}[-,level/.style={sibling distance = 4cm/#1,level distance = 1cm}]
%\draw[style=help lines] (-5,-7) grid (5,0);
\node [arn_n] at (-2,0) {6}
        child{ node [arn_r] {2}
            child{ node [arn_n] {1}
                child{ node [arn_x] {}} %for a named pointer
                child{ node [arn_x] {}}
            }
            child{ node [arn_n] {4}
                child{ node [arn_r] {3}
                    child{ node [arn_x] {}}
                    child{ node [arn_x] {}}
                }
                child{ node [arn_r] {5}
                    child{ node [arn_x] {}}
                    child{ node [arn_x] {}}
                }
            }
        }
        child{ node [arn_n] {8}
            child{ node [arn_x] {}}
            child{ node [arn_r] {9}
                child{ node [arn_x] {}}
                child{ node [arn_x] {}}
            }
        }
;

{
}
;
\end{tikzpicture}
\caption{{\Arn antes de eliminar nodo 6.}}
\end{figure}

\begin{figure}
\centering
\captionsetup{justification=centering}
\label{arbolRB}
 \begin{tikzpicture}[-,level/.style={sibling distance = 4cm/#1,level distance = 1cm}]
%\draw[style=help lines] (-5,-7) grid (5,0);
%\node [arn_n] at (-2,0) {6}

        \node [arn_r] {2}
            child{ node [arn_n] {1}
                child{ node [arn_x] {}} %for a named pointer
                child{ node [arn_x] {}}
            }
            child{ node [arn_n] {4}
                child{ node [arn_r] {3}
                    child{ node [arn_x] {}}
                    child{ node [arn_x] {}}
                }
                child{ node [arn_r] {5}
                    child{ node [arn_x] {}}
                    child{ node [arn_x] {}}
                }
            }
;
\end{tikzpicture}
\begin{tikzpicture}[-,level/.style={sibling distance = 4cm/#1,level distance = 1cm}]

            \node [arn_n] {8}
            child{ node [arn_x] {}}
            child{ node [arn_r] {9}
                child{ node [arn_x] {}}
                child{ node [arn_x] {}}
            }

;
\end{tikzpicture}

\caption{{\Arn} roto, después de eliminar nodo 6.}
\end{figure}

\begin{figure}
\centering
\captionsetup{justification=centering}
\label{arbolRB}
 \begin{tikzpicture}[-,level/.style={sibling distance = 4cm/#1,level distance = 1cm}]
%\draw[style=help lines] (-5,-7) grid (5,0);
\node [arn_n] at (-2,0) {5}
        child{ node [arn_r] {2}
            child{ node [arn_n] {1}
                child{ node [arn_x] {}} %for a named pointer
                child{ node [arn_x] {}}
            }
            child{ node [arn_n] {4}
                child{ node [arn_r] {3}
                    child{ node [arn_x] {}}
                    child{ node [arn_x] {}}
                }
                child{ node [arn_x] {}}
            }
        }
        child{ node [arn_n] {8}
            child{ node [arn_x] {}}
            child{ node [arn_r] {9}
                child{ node [arn_x] {}}
                child{ node [arn_x] {}}
            }
        }
;

{
}
;
\end{tikzpicture}
\caption{{\Arn} después de aplicar función append.}
\end{figure}

\subsection{Funci\'on de eliminaci\'on}

Para poder comprender la l\'ogica de las funciones\footnote{auxiliares y principales} de la
operaci\'on de eliminaci\'on es necesario empezar por la funci'on que elimina los nodos, ver la
figura 2.9. La idea principal es bastante simple, como los \arns son \'arboles de búsqueda, lo
primero que hacemos es buscar el nodo a eliminar, si se encuentra se elimina y se ``pegan'' los
subárboles restantes de esta operaci\'on, ver figuras 2.6, 2.7 y 2.8. A continuaci\'on se describe
mas a fondo los casos de la misma:

\begin{itemize}
    \item Si se recibe un \'arbol vacío como argumento de la funci'on se regresa el mismo, pues
    eliminar un elemento del \'arbol vacío resulta en el mismo. También este caso sirve para
    cuando un elemento no es encontrado en el \'arbol, es el caso base de la recursi\'on de
    búsqueda.
    \item En otro caso, se realiza recursivamente la búsqueda del elemento a eliminar. Si el nodo
    no contiene el elemento que buscamos, comparamos si el elemento es menor o mayor para seguir
    buscando en el \'arbol izquierdo o derecho respectivamente y si el nodo es negro se realiza la
    operaci\'on $lbalS$\footnote{Se explican mas adelante} si la operación recursiva se hace al
    árbol izquierdo o $rbalS$\footnote{Se explican mas adelante} si la operación recursiva se hace
    al árbol derecho. Si el elemento en el que estamos no es ni mayor ni menor al que buscamos, en
    ese caso eliminamos el elemento y pegamos los subárboles restantes usando la función $append$
    \footnote{Se explican mas adelante}.
\end{itemize}

Podemos ver que las funciones de balanceo $lbalS$ y $rbalS$ se aplican cuando el nodo en el que
estamos parados, sea $n$, es negro. Esto evita que después de eliminar un nodo y aplicar la
funci\'on $append$, se acabe con dos nodos rojos seguidos, es decir, que el hijo y alguno de los
nietos del nodo $n$ sean rojos.

\begin{figure}
\centering
\captionsetup{justification=centering}
\begin{minted}{coq}
Fixpoint del {a} `{GHC.Base.Ord a} (x:a) (t:RB a) :=
 match t with
 | E => E
 | T _ a y b =>
    if x GHC.Base.< y : bool then
      match a with
       | T B _ _ _ => lbalS (del x a) y b
       | _ => T R (del x a) y b
      end
    else
    if x GHC.Base.> y : bool then
      match b with
       | T B _ _ _ => rbalS a y (del x b)
       | _ => T R a y (del x b)
      end
    else append a b
 end.

Definition remove x t := makeBlack (del x t).
\end{minted}
\caption{Función de eliminación}
\end{figure}

\subsection{Funci\'on de concatenaci\'on}

La funci\'on de concatenación, figura 2.10, es usada cuando se encuentra el elemento que se busca
eliminar del {\arn}, esto es porque la acci\'on de retirar un nodo del \'arbol da como resultado
dos \'arboles que tienen que ser concatenados y tienen que respetar los invariantes de los {\arns}
. Esta funci\'on recibe como parámetros los dos \'arboles\footnote{Estos arboles pueden no cumplir
las invariantes de un {\arn}.} que estamos buscando juntar.

Sean $a$ y $b$ los dos subárboles a los que se les aplicar\'a la funci\'on append, es decir,
$append(a,b)$, tenemos los siguientes casos:

\begin{itemize}
    \item Si $a$ es el \'arbol vacío, entonces se regresa $b$.
    \item Si $b$ es el \'arbol vacío, entonces regresamos $a$.
    \item Si $a$ y $b$ son arboles con raíces rojas, entonces se aplica $append$ al subárbol
    derecho de $a$, sea $ar$, con el subárbol izquierdo de $b$, sea $bl$, es decir, $append(ar,bl)$
    . Si el resultado de esta operación es un árbol con raíz roja, sea $arbl$, los \'arboles $a$ y
    $b$ se pintan de rojo y se concatenan con la raíz de $arbl$, igual de color rojo, $ar$ se
    reemplaza por el subárbol izquierdo de $arbl$ y $bl$ se reemplaza por el subárbol derecho de
    $arbl$.
    En otro caso, si el 'arbol resultante de $append(ar,bl)$ no es rojo, tomamos $a$ y $b$, los
    pintamos de rojo, el subárbol derecho de $a$ se reemplaza por $b$ y el subárbol izquierdo de
    $b$ se reemplaza por el resultado de $append(ar,bl)$.
    \item Si $a$ y $b$ son arboles con raíces negras, entonces se aplica $append$ al subárbol
    derechode $a$, sea $ar$, con el subárbol izquierdo de $b$, sea $bl$, es decir, $append(ar,bl)$
    . Si el resultado de esta operación es un árbol con raíz roja, sea $arbl$, los \'arboles $a$ y
    $b$ se pintan de negro y se concatenan con la raíz de $arbl$, esta de color rojo, $ar$ se
    reemplaza porel subárbol izquierdo de $arbl$ y $bl$ se reemplaza por el subárbol derecho de
    $arbl$.
    En otro caso, si el 'arbol resultante de $append(ar,bl)$ no es rojo, tomamos $a$ y $b$, el
    subárbol derecho de $a$ se reemplaza por $b$ y el subárbol izquierdo de $b$ se reemplaza por
    el resultado de $append(ar,bl)$ y a este resultado le aplicamos una función de balanceo,
    $lbalS$.
    \item Si $a$ es un \'arbol de color negro y $b$ de color rojo, entonces se toma $b$, se pinta
    de rojo pero en lugar de su subárbol izquierdo, sea $bl$, se aplica una llamada recursiva a
    $bl$ conla  funci\'on embebida en $append$, llamada $append\_l$, es decir, $append\_l(bl)$, la
    cual sigue teniendo al \'arbol $a$, gracias al currying\cite{Currying}.
    \item Si $a$ es un \'arbol de color rojo y $b$ de color negro, entonces se toma $a$, se pinta
    de rojo pero en lugar de su subárbol derecho, sea $ar$, se hace una llamada recursiva con
    $append(ar,b)$.
\end{itemize}

Debemos mencionar que el \'arbol resultante de aplicar esta funci\'on no necesariamente cumple las
invariantes de un {\arn}, estas invariantes se logran conservar ya que en la funci\'on $del()$ se
realizan llamadas a las funciones extendidas de balanceo, las cuales desarrollaremos en la
siguiente sección.

\begin{figure}
\centering
\captionsetup{justification=centering}
\begin{minted}{coq}
Fixpoint append {a} `{GHC.Base.Ord a} (l:RB a) : RB a -> RB a :=
 match l with
 | E => fun r => r
 | T lc ll lx lr =>
   fix append_l (r:RB a) : RB a :=
   match r with
   | E => l
   | T rc rl rx rr =>
     match lc, rc with
     | R, R =>
       let lrl := append lr rl in
       match lrl with
       | T R lr' x rl' => T R (T R ll lx lr') x (T R rl' rx rr)
       | _ => T R ll lx (T R lrl rx rr)
       end
     | B, B =>
       let lrl := append lr rl in
       match lrl with
       | T R lr' x rl' => T R (T B ll lx lr') x (T B rl' rx rr)
       | _ => lbalS ll lx (T B lrl rx rr)
       end
     | B, R => T R (append_l rl) rx rr
     | R, B => T R ll lx (append lr r)
     end
   end
 end.
\end{minted}
\caption{Funci\'on de concatenaci\'on, append}
\end{figure}

\subsection{Extensi\'on de funciones de balanceo}

En la secci\'on 2.2 de este trabajo se trato la inserci\'on de elementos a un \arn, en donde se
describen un par de funciones llamadas ``funciones de balanceo'', tratadas en las subsecci\'on
2.2.1, estas funciones a su vez toman los nombres $rbal$ y $lbal$, figura 2.3. Estas funciones
resultan insuficientes para balancear un \'arbol al momento de eliminar un nodo y concatenar los
dos arboles restantes con la función $append$, es por eso que se implementan las extensiones de
estas funciones, llamadas $lbalS$ y $rbalS$, figuras 2.11 y 2.12 respectivamente, las cuales a su
vez llaman a las funciones $rbal'$\footnote{La funci\'on $rbal'$ es una variación de la función
$rbal$, solo cambia el orden de la caza de patrones.}, figura 2.13, y $lbal$. Estas extensiones
agregan mas casos de manejo de subarboles negros, esto porque existen casos en los que se puede
llegar a eliminar un nodo negro intermedio y se tiene que asegurar que las invariantes no se
violen después de concatenar los subárboles resultantes de aplicar la función $del$.

\begin{figure}
\centering
\captionsetup{justification=centering}
\begin{minted}{coq}

Definition lbalS {a} `{GHC.Base.Ord a} (l:RB a) (k:a) (r:RB a) :=
 match l with
 | T R a x b => T R (T B a x b) k r
 | _ =>
   match r with
   | T B a y b => rbal' l k (T R a y b)
   | T R (T B a y b) z c => T R (T B l k a) y (rbal' b z (makeRed c))
   | _ => T R l k r
   end
 end.

\end{minted}
\caption{Funci\'on de balanceo de lado izquierdo extendida.}
\end{figure}

Las funciones $rbalS$ y $lbalS$ son usadas en la funci\'on $del$, figura 2.9, cuando el caso en el
que se cae es un nodo de color negro, y al aplicar la funci\'on en estos nodos podemos asegurar
que los dos subárboles de este nodo no se van a desequilibrar, es decir, que un subárbol tenga mas
nodos que el otro, después de aplicar la función de balanceo se aplica otra llamada recursiva a
$del$.


\begin{figure}
\centering
\captionsetup{justification=centering}
\begin{minted}{coq}

Definition rbalS {a} `{GHC.Base.Ord a} (l:RB a) (k:a) (r:RB a) :=
 match r with
 | T R b y c => T R l k (T B b y c)
 | _ =>
   match l with
   | T B a x b => lbal (T R a x b) k r
   | T R a x (T B b y c) => T R (lbal (makeRed a) x b) y (T B c k r)
   | _ => T R l k r
   end
 end.

\end{minted}
\caption{Funciones de balanceo de lado derecho extendida.}
\end{figure}

Hay otra función donde se utiliza una de estas funciones, específicamente $lbalS$, esta es $append$
en el caso de que los arboles que se le están dando como parámetros sean negros, esto es por la
misma razón por la cual se aplican las funciones de balanceo en $del$ sobre los nodos de color
negro; para que sus subárboles no se desbalanceen.

\begin{figure}
\centering
\captionsetup{justification=centering}
\begin{minted}{coq}
Definition rbal' {a} `{GHC.Base.Ord a} (l:RB a) (k:a) (r:RB a) :=
 match r with
 | T R b y (T R c z d) => T R (T B l k b) y (T B c z d)
 | T R (T R b y c) z d => T R (T B l k b) y (T B c z d)
 | _ => T B l k r
 end.

\end{minted}
\caption{Funci\'on de balanceo de lado derecho alternativa.}
\end{figure}

Estas definiciones y funciones son suficientes para poder eliminar nodos de un {\arn} y que el
resultado no viole las invariantes de los mismos, aunque realmente nos falta demostrar que esta
afirmaci\'on es cierta, esto se tratar\'a en el siguiente cap\'itulo.   
