\chapter{Implementación de arboles roji-negros en {\coq}}

\section{Traducción de implementaciones}
Se tuvieron un par de aproximaciones para la implementación de {\arns}: la primera fue obtener las
implementaciones de estos \cite{tesisG} en Haskell, que fueron utilizadas como entrada para la
herramienta \textit{hs-to-coq}, es decir, una traducci\'on directa. La segunda aproximación y la que
se uso para este trabajo, fue obtener de \cite{MSetRBT} la implementaci\'on de los {\arns} que se
usan en la biblioteca estándar de {\coq}, los cuales son una versi\'on de los {\arns} de Okasaki\cite{Okasaki}; en este caso se usaron
las bibliotecas traducidas de Haskell a {\coq}, las cuales contienen los tipos y operaciones sobre ellos de Haskell.
Esta traducción se obtuvo con la ayuda del traductor \textit{hs-to-coq} y estas
sustituyeron a los tipos y operaciones de {\coq}. A continuación profundizaremos de estos 
dos casos.

\subsection{Traducción directa de implementaciones de Haskell a {\coq}}
De un trabajo anterior \cite{tesisG} se obtuvieron diversas implementaciones de {\arns}; estas
variaban en su mayor parte en las operaciones de borrado, es por ello que dicha operación es
significativamente mas compleja que su contraparte, es decir, la operación de inserci\'on. Estas 
variantes son: la implementación de Okasaki, siendo esta la m\'as simple,
los constructores inteligentes (implementaci\'on anterior con optimizaciones) y los 
tipos anidados (una implementaci\'on totalmente diferente a las anteriores y mas elegante).

Por la compleja naturaleza de estas implementaciones, la traducción
manual del código de Haskell result\'o ser muy problemática, esto porque las implementaciones en
Haskell se aprovechan del hecho de que en este lenguaje se pueden declarar funciones parciales, lo
cual representa un reto al momento de intentar traducir a {\coq}, ya que este lenguaje únicamente 
acepta funciones totales. Se buscaron soluciones para `totalizar' estas funciones, sin embargo,
solo traerían problemas al intentar realizar las demostraciones, ya que al totalizar se generar\'ian 
casos inalcanzables en la ejecuci\'on, pero tendrían que ser demostrados para poder completar la verificaci\'on.

A pasear de ello, se intent\'o totalizar las funciones de Haskell y así poder 
usar la herramienta \textit{hs-to-coq} y de esta manera facilitar la traducci\'on, pero por las mismas
razones antes descritas, la herramienta caía en alguna de estas dos situaciones:

\begin{itemize}
    \item El tiempo de ejecuci\'on de la herramienta era muy alto y eventualmente los recursos de
    la maquina virtual, donde esta herramienta se ejecut\'o, se quedaba sin recursos (en
    especial memoria). Esto probablemente se deba a la falta de totalidad en alguna función.
    \item La herramienta generaba c\'odigo en {\coq} pero con elementos de Haskell cuyas bibliotecas
    todavía no habían sido traducidas del todo. Esto porque las implementaciones en Haskell pod\'ian
    llegar a ser muy complejas y utilizar módulos de GHC, a los cuales todav\'ia no se les hab\'ia
    traducido con la herramienta.
\end{itemize}{}

Por estas razones se busc\'o otro acercamiento para poder verificar esta estructura, entonces,
sabemos que el equipo de desarrollo de la herramienta \textit{hs-to-coq} ha traducido exitosamente una
fracci\'on de las bibliotecas de Haskell a {\coq}\footnote{El nombre de esta es \textit{GHC.Base}, nos referiremos a las funciones de la biblioteca con ese mismo prefijo.}, por esta raz\'on, se opt\'o por el uso de la
implementación de {\arns} de las bibliotecas de {\coq}, \cite{MSetRBT}, pero usando los tipos y
operaciones obtenidos de las traducciones con la herramienta.

\section{Inserción de elementos en un {\arn}}

La inserci\'on de elementos a un {\arn} es la operaci\'on mas sencilla de las dos que se
verificar\'an en este trabajo. La idea principal detrás de este algoritmo es que \'unicamente se 
agreguen hojas al \'arbol binario y se efectúen ``giros'' o funciones de balanceo para 
mantener las invariantes de la estructura (ver figura \ref{arbolRB_2} y \ref{arbolRB_3}).

\begin{figure}[!ht]
\centering
\captionsetup{justification=centering}
 \begin{tikzpicture}[-,level/.style={sibling distance = 4cm/#1,level distance = 1cm}]
%\draw[style=help lines] (-5,-7) grid (5,0);
\node [arn_n] at (-2,0) {6}
        child{ node [arn_r] {2}
            child{ node [arn_n] {1}
                child{ node [arn_x] {}} %for a named pointer
                child{ node [arn_x] {}}
            }
            child{ node [arn_n] {4}
                child{ node [arn_r] {3}
                    child{ node [arn_x] {}}
                    child{ node [arn_x] {}}
                }
                child{ node [arn_r] {5}
                    child{ node [arn_x] {}}
                    child{ node [arn_x] {}}
                }
            }
        }
        child{ node [arn_n] {8}
            child{ node [arn_x] {}}
            child{ node [arn_r] {9}
                child{ node [arn_x] {}}
                child{ node [arn_x] {}}
            }
        }
;

{
}
;
\end{tikzpicture}
\caption{{\Arn} antes de insertar nodo 7.}
\label{arbolRB_2}
\end{figure}

\begin{figure}[!ht]
\centering
\captionsetup{justification=centering}
 \begin{tikzpicture}[-,level/.style={sibling distance = 4cm/#1,level distance = 1cm}]
%\draw[style=help lines] (-5,-7) grid (5,0);
\node [arn_n] at (-2,0) {6}
        child{ node [arn_r] {2}
            child{ node [arn_n] {1}
                child{ node [arn_x] {}} %for a named pointer
                child{ node [arn_x] {}}
            }
            child{ node [arn_n] {4}
                child{ node [arn_r] {3}
                    child{ node [arn_x] {}}
                    child{ node [arn_x] {}}
                }
                child{ node [arn_r] {5}
                    child{ node [arn_x] {}}
                    child{ node [arn_x] {}}
                }
            }
        }
        child{ node [arn_n] {8}
            child{ node [arn_r] {7}
                child{ node [arn_x] {}}
                child{ node [arn_x] {}}
            }
            child{ node [arn_r] {9}
                child{ node [arn_x] {}}
                child{ node [arn_x] {}}
            }
        }
;

{
}
;
\end{tikzpicture}
\caption{{\Arn} después de insertar nodo 7.}
\label{arbolRB_3}
\end{figure}
\subsection{Operaciones de Balanceo}
Los giros antes mencionados están definidos en las operaciones de balanceo, las cuales son dos, una para
los subárboles izquierdos y otra para los derechos. Estas funciones (ver figura \ref{func_balanceo})
se encargan de solucionar los casos en los que inmediatamente después de agregar una hoja alguna de 
las invariantes sean violadas, por ejemplo, dos nodos rojos que resultan contiguos en algún lugar 
de la estructura del \'arbol.

\begin{figure}[!ht]
\centering
\captionsetup{justification=centering}
\begin{minted}{coq}
Definition lbal {a} `{GHC.Base.Ord a} (l:RB a) (k:a) (r:RB a) :=
 match l with
 | T R (T R a x b) y c => T R (T B a x b) y (T B c k r)
 | T R a x (T R b y c) => T R (T B a x b) y (T B c k r)
 | _ => T B l k r
 end.

 Definition rbal {a} `{GHC.Base.Ord a} (l:RB a) (k:a) (r:RB a) :=
 match r with
 | T R (T R b y c) z d => T R (T B l k b) y (T B c z d)
 | T R b y (T R c z d) => T R (T B l k b) y (T B c z d)
 | _ => T B l k r
 end.
\end{minted}
\caption{Funciones de Balanceo.}
\label{func_balanceo}
\end{figure}

El balanceo elimina el doble nodo rojo al crear \'unicamente un nodo rojo con dos hijos negros, de 
igual manera esto nos asegura que el árbol crece de forma controlada en n\'umero de nodos negros, es decir,
mantiene la invariante de la altura negra, lo cual se debe a que en ningún
momento se están agregando dos nodos negros contiguos\footnote{Nodos padre e hijo negros después de
balancear.}; cabe mencionar que esta es la única operación en donde se agregan nodos negros, con la
excepción de $makeBlack$, la cual describiremos m\'as adelante.

En puntos posteriores se explicar\'an los casos de uso de esta función, se desarrollar\'a el 
porqu\'e los \'unicos casos a los que se les da un trato especial es a los de nodos rojos contiguos
y en el resto s\'olo se regresa un \'arbol con ra\'iz negra sin hacer mayor acomodo.

\subsection {Funci\'on de inserci\'on}
Esta funci\'on es donde se presenta por primera vez el uso de las bibliotecas traducidas de
Haskell, podemos apreciar como los tipos \footnote{El tipo que se usa en los \arns es representado 
con la letra \textbf{\textit{a}}.} de los elementos que se est\'an agregando al \'arbol son tipos
ordenados de la biblioteca $Base$ del compilador de GHC y por esa misma raz\'on estamos usando las
comparaciones de esa biblioteca.

\begin{figure}[!ht]
\centering
\captionsetup{justification=centering}
\begin{minted}{coq}
Fixpoint ins {a} `{GHC.Base.Ord a} (x:a) (s:RB a) :=
 match s with
 | E => T R E x E
 | T c l y r =>
    if x GHC.Base.< y : bool then
      match c with
       | R => T R (ins x l) y r
       | B => lbal (ins x l) y r
      end
    else
    if x GHC.Base.> y : bool then
      match c with
       | R => T R l y (ins x r)
       | B => rbal l y (ins x r)
      end
    else s
 end.
\end{minted}
\caption{Funci\'on ins.}
\label{func_ins}
\end{figure}

Analizando m\'as detenidamente la funci\'on (figura \ref{func_ins}) se puede observar que las
operaciones de balanceo solo se efectúan cuando el nodo por el que se esta pasando es negro, esto
sucede por la raz\'on de que los nodos de este color son los que se toman en cuenta para decidir si
un \'arbol cumple con el balanceo adecuado. Al aplicar el balanceo en estos nodos, podemos 
garantizar que no quedar\'an con nodos negros extras alguno de los hijos de este nodo, es decir, que
ninguno de los caminos de la ra\'iz a las hojas tenga mas nodos negros que los demás. Esto se puede
apreciar si recordamos lo que se menciono en las definiciones de las operaciones de balanceo, tomemos $rbal$ (figura
\ref{func_balanceo})\footnote{Con $lbal$ la idea es an\'aloga}, tenemos dos casos:

\begin{itemize}
    \item Sean \textbf{\textit{x}}, \textbf{\textit{y}} y \textbf{\textit{z}} nodos del \'arbol y 
    sea \textbf{\textit{t}} un subárbol, \textbf{\textit{x}} es el nodo al que se le aplica la 
    operaci\'on de balanceo y este es de color negro, \textbf{\textit{t}} es el subárbol izquierdo,
    \textbf{\textit{y}} es el nodo derecho de \textbf{\textit{x}} y \textbf{\textit{z}} es hijo de
    \textbf{\textit{y}} (Es irrelevante si es derecho o izquierdo, el resultado es el mismo.
    ). Suponiendo que \textbf{\textit{y}} y \textbf{\textit{z}} son rojos\footnote{se viola una 
    invariante, dos nodos rojos contiguos}, se cae en cualquiera de los dos casos de $rbal$ que no 
    sean el caso general. En este momento es donde se efectúa el \textit{balanceo} del árbol y 
    resulta lo siguiente: \textbf{\textit{x}} se vuelve el hijo izquierdo de \textbf{\textit{y}} y
    \textbf{\textit{z}} se pinta de negro \footnote{El hijo se vuelve padre y el padre se vuelve 
    hijo.}, todas las dem\'as estructuras del \'arbol permanecen igual.

    En el momento en que \textbf{\textit{x}} se convierte en hijo izquierdo de \textbf{\textit{y}}
    el \'arbol se desbalancea, es por esto que se pinta de negro a \textbf{\textit{z}}, así los dos
    nodos negros son hijos de \textbf{\textit{y}} y la invariante se conserva.
    \item En cualquier otro caso el \'arbol no sufre modificaci\'on alguna.
\end{itemize}

Este balanceo es necesario en esta funci\'on, ya que todos los elementos nuevos que se agregan al 
\'arbol son hojas rojas, esto puede traer consigo violaciones a los invariantes, en especial al de
que existan dos nodos rojos contiguos y esta opearci\'on ayuda a mitigar este problema.

A pesar de que las operaciones de balanceo cuidan la mayoría invariantes en el cuerpo del \'arbol,
la función $ins$ no necesariamente cumple con uno de los invariantes, espec\'ificamente en el que la
raíz del árbol es negra, es por ello que se introducen las definiciones de la figura 
\ref{raiz_negra_func}.

\begin{figure}[!ht]
\centering
\captionsetup{justification=centering}
\begin{minted}{coq}
Definition makeBlack {a} `{GHC.Base.Ord a} (t:RB a) :=
 match t with
 | E => E
 | T _ a x b => T B a x b
 end.

Definition insert {a} `{GHC.Base.Ord a} (x:a) (s:RB a) :=
                                          makeBlack (ins x s).
\end{minted}
\caption{Definiciones para pintar ra\'iz de negro.}
\label{raiz_negra_func}
\end{figure}

La definici\'on $makeBlack$ únicamente colorea un nodo de color negro y la definición
$insert$ es una envoltura de $ins$, con la cual nos aseguramos de que la ra\'iz de los \'arboles
siempre sea de color negro, esto se logra con ayuda de $makeBlack$.

Estas funciones y definiciones son suficientes para poder construir {\arns} que respeten las
invariantes que planteamos en la definici\'on 1.2.1.

\section{Eliminación de elementos en un {\arn}}

Como se menciono en la secci\'on anterior, la operaci\'on de eliminaci\'on es significativamente 
m\'as compleja que su contra parte, esto se debe al hecho de que pueden ser eliminados cualesquiera 
nodos en un {\arn}, mientras que en la inserci\'on s\'olo se agregan hojas de color rojo, es decir,
la altura \'unicamente se modifica en la inserción cuando se aplica el balanceo.

La acci\'on de eliminar nodos de cualquier parte de un {\arn} presenta una problemática muy grande 
para el balanceo del mismo, esto se suscita al eliminar un nodo del \'arbol, los dos subárboles de 
este tienen que ser concatenados de alguna forma y los invariantes de los mismos tienen que ser
respetados.

\subsection{Funci\'on de eliminaci\'on}

Para poder comprender la l\'ogica de las funciones que conforman a la operaci\'on de eliminaci\'on
es necesario comenzar por la funci'on que retira el nodo del \'arbol (ver la figura \ref{func_del}).
La idea central de esta operaci\'on es bastante simple: como los {\arns} son \'arboles de búsqueda,
lo primero que hacemos es buscar el nodo a eliminar, si se encuentra se elimina y se concatenan los
subárboles restantes de esta operaci\'on (ver figuras \ref{arbolRB_4}, \ref{arbolRB_5} y
\ref{arbolRB_6}). A continuaci\'on se describen m\'as a fondo los casos de la misma:

\begin{itemize}
    \item Si se recibe un \'arbol vacío como argumento de la funci\'on, se regresa este mismo; pues
    eliminar un elemento del \'arbol vacío termina siendo vacío. También este caso sirve para
    cuando un elemento no es encontrado en el \'arbol, es el caso base de la recursi\'on de
    búsqueda del nodo a eliminar.
    \item En otro caso, se realiza recursivamente la búsqueda del elemento a eliminar. Si el nodo
    actual no contiene el elemento que buscamos, se compara si es menor o mayor para seguir
    buscando en el \'arbol izquierdo o derecho respectivamente, describimos los casos:
    \begin{itemize}
        \item Si el nodo es menor y el nodo del \'arbol izquierdo es negro, entonces aplicamos la funci\'on $lbalS$ \footnote{Funci\'on de balanceo extendida para subárboles
    izquierdos.} al \'arbol resultante de seguir buscando el nodo a eliminar por el subárbol izquierdo. 
        En otro caso solo seguimos buscando por el subárbol izquierdo.
        \item Si el nodo es mayor y el nodo del \'arbol derecho es negro, entonces aplicamos la funci\'on $rbalS$ \footnote{Funci\'on de balanceo extendida para subárboles
    derechos.} al \'arbol resultante de seguir buscando el nodo a eliminar por el subárbol derecho. 
        En otro caso solo seguimos buscando por el subárbol derecho.
        \item Si el elemento en el que estamos no es ni mayor ni menor al que buscamos, en ese
    caso eliminamos el elemento y concatenamos los subárboles restantes usando la función $append$
    \footnote{Funci\'on donde se juntan lo arboles restantes de esta operaci\'on}.
    \end{itemize}
    
\end{itemize}

\begin{figure}[!ht]
\centering
\captionsetup{justification=centering}
\begin{minted}{coq}
Fixpoint del {a} `{GHC.Base.Ord a} (x:a) (t:RB a) :=
 match t with
 | E => E
 | T _ a y b =>
    if x GHC.Base.< y : bool then
      match a with
       | T B _ _ _ => lbalS (del x a) y b
       | _ => T R (del x a) y b
      end
    else
    if x GHC.Base.> y : bool then
      match b with
       | T B _ _ _ => rbalS a y (del x b)
       | _ => T R a y (del x b)
      end
    else append a b
 end.

Definition remove x t := makeBlack (del x t).
\end{minted}
\caption{Función de eliminación}
\label{func_del}
\end{figure}

\begin{figure}[!ht]
\centering
\captionsetup{justification=centering}
 \begin{tikzpicture}[-,level/.style={sibling distance = 4cm/#1,level distance = 1cm}]
%\draw[style=help lines] (-5,-7) grid (5,0);
\node [arn_n] at (-2,0) {6}
        child{ node [arn_r] {2}
            child{ node [arn_n] {1}
                child{ node [arn_x] {}} %for a named pointer
                child{ node [arn_x] {}}
            }
            child{ node [arn_n] {4}
                child{ node [arn_r] {3}
                    child{ node [arn_x] {}}
                    child{ node [arn_x] {}}
                }
                child{ node [arn_r] {5}
                    child{ node [arn_x] {}}
                    child{ node [arn_x] {}}
                }
            }
        }
        child{ node [arn_n] {8}
            child{ node [arn_x] {}}
            child{ node [arn_r] {9}
                child{ node [arn_x] {}}
                child{ node [arn_x] {}}
            }
        }
;

{
}
;
\end{tikzpicture}
\caption{{\Arn antes de eliminar nodo 6.}}
\label{arbolRB_4}
\end{figure}

\begin{figure}[!ht]
\centering
\captionsetup{justification=centering}
 \begin{tikzpicture}[-,level/.style={sibling distance = 4cm/#1,level distance = 1cm}]
%\draw[style=help lines] (-5,-7) grid (5,0);
%\node [arn_n] at (-2,0) {6}

        \node [arn_r] {2}
            child{ node [arn_n] {1}
                child{ node [arn_x] {}} %for a named pointer
                child{ node [arn_x] {}}
            }
            child{ node [arn_n] {4}
                child{ node [arn_r] {3}
                    child{ node [arn_x] {}}
                    child{ node [arn_x] {}}
                }
                child{ node [arn_r] {5}
                    child{ node [arn_x] {}}
                    child{ node [arn_x] {}}
                }
            }
;
\end{tikzpicture}
\begin{tikzpicture}[-,level/.style={sibling distance = 4cm/#1,level distance = 1cm}]

            \node [arn_n] {8}
            child{ node [arn_x] {}}
            child{ node [arn_r] {9}
                child{ node [arn_x] {}}
                child{ node [arn_x] {}}
            }

;
\end{tikzpicture}

\caption{{\Arn} roto, después de eliminar nodo 6.}
\label{arbolRB_5}
\end{figure}

\begin{figure}[!ht]
\centering
\captionsetup{justification=centering}
 \begin{tikzpicture}[-,level/.style={sibling distance = 4cm/#1,level distance = 1cm}]
%\draw[style=help lines] (-5,-7) grid (5,0);
\node [arn_n] at (-2,0) {5}
        child{ node [arn_r] {2}
            child{ node [arn_n] {1}
                child{ node [arn_x] {}} %for a named pointer
                child{ node [arn_x] {}}
            }
            child{ node [arn_n] {4}
                child{ node [arn_r] {3}
                    child{ node [arn_x] {}}
                    child{ node [arn_x] {}}
                }
                child{ node [arn_x] {}}
            }
        }
        child{ node [arn_n] {8}
            child{ node [arn_x] {}}
            child{ node [arn_r] {9}
                child{ node [arn_x] {}}
                child{ node [arn_x] {}}
            }
        }
;

{
}
;
\end{tikzpicture}
\caption{{\Arn} después de aplicar función append.}
\label{arbolRB_6}
\end{figure}

Podemos ver que las funciones de balanceo $lbalS$ y $rbalS$ se aplican cuando el nodo en el que
estamos parados, llamémoslo \textbf{\textit{n}}, es negro; esto evita que después de eliminar un 
nodo y aplicar la funci\'on $append$ se acabe con dos nodos rojos seguidos, es decir, que el hijo y 
alguno de los nietos del nodo \textbf{\textit{n}} sean rojos.

\subsection{Funci\'on de concatenaci\'on ($append$)}

La funci\'on de concatenación (figura \ref{func_app}) es usada cuando se encuentra el elemento que
se busca eliminar de un {\arn}, esto porque la acci\'on de retirar un nodo del \'arbol resulta
en dos \'arboles que tienen que ser concatenados, los cuales deben de respetar los invariantes de
los {\arns}. Esta funci\'on recibe como parámetros los dos \'arboles\footnote{Estos arboles pueden
no cumplir las invariantes de ls {\arns}.} que estamos buscando unir. Esta operación se describe
con mayor detalle en seguida.

\begin{figure}[!ht]
\centering
\captionsetup{justification=centering}
\begin{minted}{coq}
Fixpoint append {a} `{GHC.Base.Ord a} (l:RB a) : RB a -> RB a :=
 match l with
 | E => fun r => r
 | T lc ll lx lr =>
   fix append_l (r:RB a) : RB a :=
   match r with
   | E => l
   | T rc rl rx rr =>
     match lc, rc with
     | R, R =>
       let lrl := append lr rl in
       match lrl with
       | T R lr' x rl' => T R (T R ll lx lr') x (T R rl' rx rr)
       | _ => T R ll lx (T R lrl rx rr)
       end
     | B, B =>
       let lrl := append lr rl in
       match lrl with
       | T R lr' x rl' => T R (T B ll lx lr') x (T B rl' rx rr)
       | _ => lbalS ll lx (T B lrl rx rr)
       end
     | B, R => T R (append_l rl) rx rr
     | R, B => T R ll lx (append lr r)
     end
   end
 end.
\end{minted}
\caption{Funci\'on de concatenaci\'on, append}
\label{func_app}
\end{figure}


Sean \textbf{\textit{a}} y \textbf{\textit{b}} los dos subárboles a los que se les aplicar\'a la 
funci\'on append, es decir, \textit{append a b}, tenemos los siguientes casos:

\begin{itemize}
    \item Si \textbf{\textit{a}} es el \'arbol vacío, entonces se regresa \textbf{\textit{b}}.
    \item Si \textbf{\textit{b}} es el \'arbol vacío, entonces regresamos \textbf{\textit{a}}.
    \item Si \textbf{\textit{a}} y \textbf{\textit{b}} son \'arboles con raíces rojas, entonces se 
    aplica $append$ al subárbol derecho de \textbf{\textit{a}}, sea este \textbf{\textit{ar}}, 
    junto con el subárbol izquierdo de \textbf{\textit{b}}, sea \textbf{\textit{bl}}, es decir, 
    \textit{append ar bl}. Tenemos subcasos:
    \begin{itemize}
      \item Si el resultado de esta operación es un árbol con raíz roja, sea \textbf{\textit{arbl}},
      los \'arboles \textbf{\textit{a}} y \textbf{\textit{b}} se pintan de rojo y se concatenan con 
      la raíz de \textbf{\textit{arbl}}, igual de color rojo; \textbf{\textit{ar}} se reemplaza por 
      el subárbol izquierdo de \textbf{\textit{arbl}} y \textbf{\textit{bl}} se reemplaza por el
      subárbol derecho de \textbf{\textit{arbl}}.
      \item En otro caso, si el \'arbol resultante de \textit{append ar bl} no es rojo, tomamos 
      \textbf{\textit{a}} y \textbf{\textit{b}}, los pintamos de rojo, el subárbol derecho de
      \textbf{\textit{a}} se reemplaza por \textbf{\textit{b}} y el subárbol izquierdo de 
      \textbf{\textit{b}} se reemplaza por el resultado de \textit{append ar bl}.
    \end{itemize}
    \item Si \textbf{\textit{a}} y \textbf{\textit{b}} son arboles con raíces negras, entonces se 
    aplica $append$ al subárbol derecho de \textbf{\textit{a}}, sea \textbf{\textit{ar}}, con el
    subárbol izquierdo de \textbf{\textit{b}}, sea \textbf{\textit{bl}}, es decir, \textit{append
    ar bl}. Tenemos casos:
    \begin{itemize}
      \item Si el resultado de esta operación es un árbol con raíz roja, sea \textbf{\textit{arbl}},
      los \'arboles \textbf{\textit{a}} y \textbf{\textit{b}} se pintan de negro y se concatenan con
      la raíz de \textbf{\textit{arbl}}, esta de color rojo; \textbf{\textit{ar}} se reemplaza por 
      el subárbol izquierdo de \textbf{\textit{arbl}} y \textbf{\textit{bl}} se reemplaza por el 
      subárbol derecho de \textbf{\textit{arbl}}.
      \item En otro caso, si el \'arbol resultante de \textit{append ar bl} no es rojo, tomamos 
      \textbf{\textit{a}} y \textbf{\textit{b}}, el subárbol derecho de \textbf{\textit{a}} se 
      reemplaza por \textbf{\textit{b}} y el subárbol izquierdo de \textbf{\textit{b}} se reemplaza 
      por el resultado de \textit{append ar bl} y a este resultado le aplicamos una función de 
      balanceo, $lbalS$.
    \end{itemize}
    \item Si \textbf{\textit{a}} es un \'arbol de color negro y \textbf{\textit{b}} de color rojo,
    entonces se toma \textbf{\textit{b}}, se pinta de rojo pero en lugar de su subárbol izquierdo, 
    sea \textbf{\textit{bl}}, se aplica una llamada recursiva a \textbf{\textit{bl}} con la 
    funci\'on embebida en $append$, llamada $append\_l$, es decir: \textit{append\_l bl}, esta 
    llamada tambi\'en carga al \'arbol \textbf{\textit{a}} gracias al currying\cite{Currying}.
    \item Si \textbf{\textit{a}} es un \'arbol de color rojo y \textbf{\textit{b}} de color negro,
    entonces se toma \textbf{\textit{a}}, se pinta de rojo pero en lugar de su subárbol derecho, sea
    \textbf{\textit{ar}}, se hace una llamada recursiva con $append(ar,b)$.
\end{itemize}

Debemos mencionar que el \'arbol resultante de aplicar esta funci\'on no necesariamente cumple los
invariantes de un {\arn}, estos invariantes se logran conservar ya que en la funci\'on $del$ se
realizan llamadas a las funciones extendidas de balanceo, las cuales desarrollaremos en la
siguiente sección.

\subsection{Extensi\'on de funciones de balanceo}

En la secci\'on 2.2 de este trabajo se trat\'o la inserci\'on de elementos a un {\arn}, en donde se
describen un par de funciones llamadas `de balanceo', tratadas en las subsecci\'on
2.2.1; estas funciones a su vez toman los nombres $rbal$ y $lbal$ (figura \ref{func_balanceo}).
Estas operaciones resultan insuficientes para balancear un \'arbol al momento de eliminar un nodo y
concatenar los dos \'arboles restantes con la función $append$, es por esto que se implementan las
extensiones de estas funciones, llamadas $lbalS$ y $rbalS$ (figuras \ref{lbalS} y \ref{rbalS}
respectivamente) las cuales a su vez llaman a las funciones $rbal'$\footnote{La funci\'on $rbal'$
es una variación de la función $rbal$, solo cambia el orden de la caza de patrones.} (figura
\ref{rbal_2}) y $lbal$. Estas extensiones agregan mas casos de manejo de subárboles negros, esto
sucede porque existen casos en los que se puede llegar a eliminar un nodo negro intermedio del \'arbol,
inclusive la misma ra\'iz y tenemos que poder asegurar que las invariantes no se violen después 
de concatenar estos subárboles restantes.

\begin{figure}[!ht]
\centering
\captionsetup{justification=centering}
\begin{minted}{coq}

Definition lbalS {a} `{GHC.Base.Ord a} (l:RB a) (k:a) (r:RB a) :=
 match l with
 | T R a x b => T R (T B a x b) k r
 | _ =>
   match r with
   | T B a y b => rbal' l k (T R a y b)
   | T R (T B a y b) z c => T R (T B l k a) y (rbal' b z (makeRed c))
   | _ => T R l k r
   end
 end.

\end{minted}
\caption{Funci\'on de balanceo de lado izquierdo extendida.}
\label{lbalS}
\end{figure}

Las funciones $rbalS$ y $lbalS$ son usadas en la funci\'on $del$ (figura \ref{func_del}) cuando el
caso en el que se cae es un nodo de color negro, al aplicar la funci\'on en estos nodos podemos
asegurar que los dos subárboles de este nodo no se van a desequilibrar, es decir, que un subárbol
tenga mayor altura negra que el otro, para asi seguir buscando el nodo a eliminar. Estas funciones
b\'asicamente hacen un reacomodo de los nodos de los subarboles para despu\'es llamar a las 
funciones $rbal'$ y $lbal$, es por eso que decimos que son extensiones de estas.


\begin{figure}[!ht]
\centering
\captionsetup{justification=centering}
\begin{minted}{coq}

Definition rbalS {a} `{GHC.Base.Ord a} (l:RB a) (k:a) (r:RB a) :=
 match r with
 | T R b y c => T R l k (T B b y c)
 | _ =>
   match l with
   | T B a x b => lbal (T R a x b) k r
   | T R a x (T B b y c) => T R (lbal (makeRed a) x b) y (T B c k r)
   | _ => T R l k r
   end
 end.

\end{minted}
\caption{Funciones de balanceo de lado derecho extendida.}
\label{rbalS}
\end{figure}

Existe otra función donde se utiliza una de estas operaciones de balanceo, específicamente $lbalS$,
esta funci\'on es $append$. Esto sucede en el caso de que los \'arboles que se le estén pasando como parámetros
sean negros, es decir, la misma razón por la que se aplican las funciones de balanceo en $del$
sobre los nodos de color negro: para que sus subárboles no se desbalanceen.

\begin{figure}[!ht]
\centering
\captionsetup{justification=centering}
\begin{minted}{coq}
Definition rbal' {a} `{GHC.Base.Ord a} (l:RB a) (k:a) (r:RB a) :=
 match r with
 | T R b y (T R c z d) => T R (T B l k b) y (T B c z d)
 | T R (T R b y c) z d => T R (T B l k b) y (T B c z d)
 | _ => T B l k r
 end.

\end{minted}
\caption{Funci\'on de balanceo de lado derecho alternativa.}
\label{rbal_2}
\end{figure}

Las definiciones y funciones presentadas son suficientes para poder eliminar nodos de un {\arn} y que el
resultado no viole los invariantes de estos. En esta etapa del trabajo nos agradar\'ia llegar a esta 
conclusi\'on, sin embargo, esta sentencia tiene que ser demostrada, es decir, tenemos que
probar que nuestros {\arns} cumplen con la  definici\'on de los mismos.
