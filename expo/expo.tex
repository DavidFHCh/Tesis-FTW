%%% Presentaciones para Lenguajes de programacion y sus paradigmas

%\documentclass[xcolor=dvipsnames,table,handout]{beamer}
%\documentclass[xcolor=dvipsnames,table]{beamer}
\documentclass[xcolor=dvipsnames,table,spanish]{beamer}

\usepackage[utf8]{inputenc}
\usepackage[spanish]{babel}
\usepackage{hyperref}
\usepackage{lmodern}
\usepackage[T1]{fontenc}

%%%% paquetes matematicas
\usepackage{amssymb,amsmath,amscd}
\usepackage{extarrows}
\usepackage{stmaryrd}
\usepackage{mathabx}
\usepackage{mathrsfs}
% \usepackage{mathabx}
\usepackage{amsthm}

%%%%%
\usepackage{hyperref}
\usepackage{graphicx}
\usepackage{multicol}
\usepackage{pifont}
\usepackage{xcolor}
\usepackage{etex}
\usepackage{tikz}
\usepackage{array}
%\usepackage{pgfplots}


\newtheorem{prop}{Proposición}

%%%% cosmetics
% D.Remy package for pretty display of rules
\usepackage{mathpartir}

% para insertar codigo con formato particular
\usepackage{listings}

% comillas
\usepackage[autostyle=true,spanish=mexican]{csquotes}

% codigo
\usepackage{verbatim}
\usepackage{alltt}

% footnotes
\usepackage[bottom]{footmisc}
\usepackage{setspace}

\usepackage{wrapfig}
\usepackage{caption}


\hfuzz=5.002pt %parameter to allow hbox overfulled by length before error!

% Options for presentation
% ------------------------
% \definecolor{mycolor}{RGB}{255,192,3}
\definecolor{mycolor}{RGB}{17,132,221}
\mode<presentation>
{
% \usetheme[secheader]{Boadilla}
% \usecolortheme{orchid}
\useoutertheme{infolines}
\useinnertheme{rectangles}
\setbeamertemplate{itemize items}[square]
\setbeamertemplate{enumerate items}[square]
\setbeamersize{text margin left=6mm, text margin right=6mm}

\setbeamercolor{alerted text}{fg=red,bg=red!70!white}
\setbeamercolor{background canvas}{bg=white}
\setbeamercolor{frametitle}{bg=mycolor,fg=white}
\setbeamercolor{normal text}{bg=white,fg=black}
\setbeamercolor{structure}{bg=black,fg=mycolor}
\setbeamercolor{title}{bg=mycolor,fg=white}
\setbeamercolor{subtitle}{bg=mycolor,fg=white}
\setbeamercolor{titlelike}{bg=white,fg=mycolor}

\setbeamercovered{invisible}

\setbeamercolor*{palette primary}{fg=mycolor,bg=white}
\setbeamercolor*{palette secondary}{bg=white,fg=white}
\setbeamercolor*{palette tertiary}{fg=mycolor,bg=white}
\setbeamercolor*{palette quaternary}{fg=white,bg=white}

\setbeamercolor{separation line}{bg=mycolor,fg=mycolor}
\setbeamercolor{fine separation line}{bg=white,fg=red}
\setbeamercolor{author in head/foot}{bg=mycolor!30!white,fg=mycolor!80!black}
\setbeamercolor{title in head/foot}{bg=mycolor!30!white,fg=mycolor!80!black}
\setbeamercolor{date in head/foot}{bg=mycolor!30!white,fg=mycolor!80!black}
\setbeamercolor{institute in head/foot}{bg=mycolor!30!white,fg=mycolor!80!black}
\setbeamercolor{section in head/foot}{bg=mycolor!60!white, fg=Red}
\setbeamercolor{subsection in head/foot}
{bg=mycolor!50!white,fg=mycolor!50!white}


\setbeamertemplate{headline}
{
  \leavevmode%
  \hbox{%
  \begin{beamercolorbox}[wd=.5\paperwidth,ht=2.65ex,dp=1.5ex,center]{section in
head/foot}%
    \usebeamerfont{section in head/foot}\insertsectionhead\hspace*{2ex}
  \end{beamercolorbox}%
  \begin{beamercolorbox}[wd=.5\paperwidth,ht=2.65ex,dp=1.5ex,center]{subsection
in head/foot}%
    \usebeamerfont{subsection in head/foot}\hspace*{2ex}\insertsubsectionhead
  \end{beamercolorbox}}%
  \vskip0pt%
}
% \beamerdefaultoverlayspecification{<+->}
\beamertemplatenavigationsymbolsempty
% \setbeamertemplate{footline}[frame number]
}

\input{macroslc}

\title[]{Verificación formal de arboles rojinegros en Haskell con Coq}
\author[]{David Felipe Hernández Chiapa}
\institute[UNAM-FC]{Facultad de Ciencias\\
Universidad Nacional Aut\'onoma de M\'exico}
\date[]{\small{\today}}



\beamerdefaultoverlayspecification{<+->}

\titlegraphic{\includegraphics[width=16mm]{fc2.png}
}

% Opciones extras
% L5: beamer en español
\usepackage[all]{xy}
\decimalpoint
% Counter para enumerates en varios frames
\newcounter{saveenumi}
\newcommand{\savei}{\setcounter{saveenumi}{\value{enumi}}}
\newcommand{\conti}{\setcounter{enumi}{\value{saveenumi}}}
\resetcounteronoverlays{saveenumi}
\definecolor{light-gray}{gray}{0.75}
\newcommand{\IF}{\operatorname{if}}
\newcommand{\THEN}{\operatorname{then}}
\newcommand{\ELSE}{\operatorname{else}}
\newcommand{\qn}{\operatorname{qn}}
\newcommand{\shift}{\operatorname{shift}}

\DeclareFontFamily{U}{mathb}{\hyphenchar\font45}
\DeclareFontShape{U}{mathb}{m}{n}{
<-6> mathb5 <6-7> mathb6 <7-8> mathb7
<8-9> mathb8 <9-10> mathb9
<10-12> mathb10 <12-> mathb12
}{}
\DeclareSymbolFont{mathb}{U}{mathb}{m}{n}
\DeclareMathSymbol{\llcurly}{\mathrel}{mathb}{"CE}
\DeclareMathSymbol{\ggcurly}{\mathrel}{mathb}{"CF}

\begin{document}

\frame{\titlepage}

\begin{frame}
  \frametitle{1. Verificación formal en Haskell}
    Haskell es un lenguaje con una base muy grande de desarrolladores que constantemente están generando mas programas escritos en el.

\end{frame}

\begin{frame}
  \frametitle{1. Verificación formal en Haskell}
    Una de las cosas que se dice de Haskell es que la verificación de su código es bastante sencilla.
\end{frame}

\begin{frame}
  \frametitle{1. Verificación formal en Haskell}
    \centering¿Pero que tan cierto y escalable es esto?
\end{frame}

\begin{frame}
  \frametitle{2. Verificación formal en Coq.}
	A diferencia de Haskell, Coq es un asistente de pruebas, con el cual tu puedes escribir un programa en el y después verificarlo formalmente. 
\end{frame}

\begin{frame}
  \frametitle{2. Verificación formal en Coq.}
    Una de las diferencias mas grandes entre la escritura de programas entre Haskell y Coq, es que Coq solo acepta funciones totales. 
\end{frame}

\begin{frame}
  \frametitle{3. Problema.}
       Nos gustaría una manera de traducir módulos de Haskell con funciones totales a Coq para poder verificarlas formalmente de una manera mas sencilla, ordenada y escalable.
\end{frame}

\begin{frame}
  \frametitle{4. hs-to-coq}
       Es una herramienta en desarrollo por un equipo de la Universidad de Pensilvania.
       
       En esta herramienta ya existen bibliotecas de Haskell traducidas a Coq y tambien te da la facilidad de traducir tus propios programas de Haskell.
\end{frame}

\begin{frame}
  \frametitle{4. hs-to-coq}
        Esta herrmienta es creada para facilitar la verificación, siguiendo los siguientes pasos:
        
\end{frame}

\begin{frame}
  \frametitle{4. hs-to-coq}
        \begin{enumerate}
            \item Escribir un modulo de Haskell, digamos un modulo de Arboles Rojinegros.
            \item Probar ese codigo en Haskell, generar ejemplos.
            \item Utilizar hs-to-coq para traducir el codigo a Coq.
            \item ¡A verificar!
        \end{enumerate}
        
\end{frame}

\begin{frame}
  \frametitle{4. hs-to-coq}
        Esto simplifica mucho la verificación en varios frentes:
        
\end{frame}

\begin{frame}
  \frametitle{4. hs-to-coq}
        \begin{itemize}
            \item La traducción no se hace a mano.
            \item La cooperación en un equipo de trabajo se hace mas sencilla.
        \end{itemize}
        
\end{frame}

\begin{frame}
  \frametitle{5. Ejemplos}
       404 not found
        
\end{frame}

\end{document}
