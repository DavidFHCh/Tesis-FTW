%%% Presentaciones para Lenguajes de programacion y sus paradigmas

%\documentclass[xcolor=dvipsnames,table,handout]{beamer}
%\documentclass[xcolor=dvipsnames,table]{beamer}
\documentclass[xcolor=dvipsnames,table,spanish]{beamer}

\usepackage[utf8]{inputenc}
\usepackage[spanish]{babel}
\usepackage{hyperref}
\usepackage{lmodern}
\usepackage[T1]{fontenc}

%%%% paquetes matematicas
\usepackage{amssymb,amsmath,amscd}
\usepackage{extarrows}
\usepackage{stmaryrd}
\usepackage{mathabx}
\usepackage{mathrsfs}
% \usepackage{mathabx}
\usepackage{amsthm}

%%%%%
\usepackage{hyperref}
\usepackage{graphicx}
\usepackage{multicol}
\usepackage{pifont}
\usepackage{xcolor}
\usepackage{etex}
\usepackage{tikz}
\usepackage{array}
%\usepackage{pgfplots}

%%%% cosmetics
% D.Remy package for pretty display of rules
\usepackage{mathpartir}

% para insertar codigo con formato particular
\usepackage{listing}

% comillas
\usepackage[autostyle=true,spanish=mexican]{csquotes}

% codigo
\usepackage{verbatim}
\usepackage{alltt}

% footnotes
\usepackage[bottom]{footmisc}
\usepackage{setspace}

\usepackage{wrapfig}
\usepackage{caption}


\hfuzz=5.002pt %parameter to allow hbox overfulled by length before error!

% Options for presentation
% ------------------------
% \definecolor{mycolor}{RGB}{255,192,3}
\definecolor{mycolor}{RGB}{17,132,221}
\mode<presentation>
{
% \usetheme[secheader]{Boadilla}
% \usecolortheme{orchid}
\useoutertheme{infolines}
\useinnertheme{rectangles}
\setbeamertemplate{itemize items}[square]
\setbeamertemplate{enumerate items}[square]
\setbeamersize{text margin left=6mm, text margin right=6mm}

\setbeamercolor{alerted text}{fg=red,bg=red!70!white}
\setbeamercolor{background canvas}{bg=white}
\setbeamercolor{frametitle}{bg=mycolor,fg=white}
\setbeamercolor{normal text}{bg=white,fg=black}
\setbeamercolor{structure}{bg=black,fg=mycolor}
\setbeamercolor{title}{bg=mycolor,fg=white}
\setbeamercolor{subtitle}{bg=mycolor,fg=white}
\setbeamercolor{titlelike}{bg=white,fg=mycolor}

\setbeamercovered{invisible}

\setbeamercolor*{palette primary}{fg=mycolor,bg=white}
\setbeamercolor*{palette secondary}{bg=white,fg=white}
\setbeamercolor*{palette tertiary}{fg=mycolor,bg=white}
\setbeamercolor*{palette quaternary}{fg=white,bg=white}

\setbeamercolor{separation line}{bg=mycolor,fg=mycolor}
\setbeamercolor{fine separation line}{bg=white,fg=red}
\setbeamercolor{author in head/foot}{bg=mycolor!30!white,fg=mycolor!80!black}
\setbeamercolor{title in head/foot}{bg=mycolor!30!white,fg=mycolor!80!black}
\setbeamercolor{date in head/foot}{bg=mycolor!30!white,fg=mycolor!80!black}
\setbeamercolor{institute in head/foot}{bg=mycolor!30!white,fg=mycolor!80!black}
\setbeamercolor{section in head/foot}{bg=mycolor!60!white, fg=Red}
\setbeamercolor{subsection in head/foot}
{bg=mycolor!50!white,fg=mycolor!50!white}


\setbeamertemplate{headline}
{
  \leavevmode%
  \hbox{%
  \begin{beamercolorbox}[wd=.5\paperwidth,ht=2.65ex,dp=1.5ex,center]{section in
head/foot}%
    \usebeamerfont{section in head/foot}\insertsectionhead\hspace*{2ex}
  \end{beamercolorbox}%
  \begin{beamercolorbox}[wd=.5\paperwidth,ht=2.65ex,dp=1.5ex,center]{subsection
in head/foot}%
    \usebeamerfont{subsection in head/foot}\hspace*{2ex}\insertsubsectionhead
  \end{beamercolorbox}}%
  \vskip0pt%
}
% \beamerdefaultoverlayspecification{<+->}
\beamertemplatenavigationsymbolsempty
% \setbeamertemplate{footline}[frame number]
}

\input{macroslc}

\title[]{Lenguajes de programación y sus paradigmas}
\subtitle{Tema: Excepciones}
\author[]{}
\institute[UNAM-FC]{Facultad de Ciencias\\
Universidad Nacional Aut\'onoma de M\'exico}
\date[]{\small{\today}
\newline{\tiny{Material desarrollado bajo el proyecto UNAM-PAPIME PE102117.}}}


\beamerdefaultoverlayspecification{<+->}

\titlegraphic{\includegraphics[width=16mm]{fc2.png}
}

% Opciones extras
% L5: beamer en español
\usepackage[all]{xy}
\decimalpoint
% Counter para enumerates en varios frames
\newcounter{saveenumi}
\newcommand{\savei}{\setcounter{saveenumi}{\value{enumi}}}
\newcommand{\conti}{\setcounter{enumi}{\value{saveenumi}}}
\resetcounteronoverlays{saveenumi}
\definecolor{light-gray}{gray}{0.75}
\newcommand{\IF}{\operatorname{if}}
\newcommand{\THEN}{\operatorname{then}}
\newcommand{\ELSE}{\operatorname{else}}
\newcommand{\qn}{\operatorname{qn}}
\newcommand{\shift}{\operatorname{shift}}

\begin{document}

\frame{\titlepage}

\begin{frame}
  \frametitle{Idea}
	En la programación existen situaciones donde una función o 		procedimiento no pueda terminar su ejecución y deba de 			reportar este hecho a su emisor.

	\begin{examples}
	\begin{itemize}
		\item División entre cero.
		\item Desborde del rango aritmético.
		\item Índice de un arreglo fuera de rango.
        \item Archivo no accesible.
        \item Sistema sin memoria.
        \item Proceso terminado por usuario.
	\end{itemize}
    \end{examples}
\end{frame}

\begin{frame}
  \frametitle{Idea}
	Algunas de las condiciones anteriores pueden ser manejadas al cambiar el rango de la función que las provoca.

  	\begin{example}
    La función predecesor regresaria lo siguiente:\newline
    \begin{itemize}
        \item \texttt{trivial()}, en caso de ser 0 el argumento.\newline
  	    \item \texttt{Maybe Nat} $=_{def}$ 1 + \texttt{Nat}, en otro caso.
    \end{itemize}
  	\end{example}

\end{frame}

\begin{frame}
  \frametitle{Excepciones sin manejo}
  La manera mas simple de introducir excepciones es mediante un costructor \textbf{error}, el cual causa la detención total del programa.\newline

  La sintaxis se extiende con una constante de error:
  \begin{center}
      $e::=...|error$
  \end{center}
  La evaluacion se extiende de la siguiente manera:
  \begin{center}
      \begin{equation*}
         \frac{}{app(error,e_2)\imp error}
	  \end{equation*}
      \begin{equation*}
         \frac{}{app(v_1,error)\imp error}
	  \end{equation*}
  \end{center}
Notese que \textbf{error} no es un valor, entonces el programa $(\lambda x: Nat.0)error$, por un lado se reduciria a 0 y por el otro a error.
\end{frame}


\begin{frame}

\end{frame}

\begin{frame}

\end{frame}

\begin{frame}

\end{frame}

\begin{frame}

\end{frame}

\begin{frame}

\end{frame}

\begin{frame}

\end{frame}

\begin{frame}

\end{frame}

\begin{frame}

\end{frame}

\begin{frame}

\end{frame}

\begin{frame}

\end{frame}

\begin{frame}

\end{frame}

\begin{frame}

\end{frame}

\begin{frame}

\end{frame}

\begin{frame}

\end{frame}

\end{document}
